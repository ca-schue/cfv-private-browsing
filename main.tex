\documentclass[
11pt,					% Schriftgröße
paper=a4,
DIV=13,				% Seitenlayout (Satzspiegel)
parskip=half,			% Abstand zwischen Absätzen
oneside,				% Doppelseitig
%  cleardoublepage,
bibtotoc,				% Literaturverzeichis in Inhaltsverzeichnis
headsepline,			% Kopfzeilentrennlinie
headings,	
%  draft,				% Korrekturfassung
table,
xcdraw
]{scrreprt}		% scrartcl	

% Eingabecodierung
\usepackage[utf8]{inputenc}

% Schriftcodierung
\usepackage[T1]{fontenc}

% Sprachraum
\usepackage[ngerman]{babel}

% Blindtext
\usepackage{blindtext}

% Schrifteinstellungen
\usepackage{lmodern} 		% Vektorschrift
\renewcommand{\familydefault}{\sfdefault} % Serifenlose Schrift
\usepackage{sansmath}  	% Mathe-Schrift ohne Serifen
\sansmath 							% aktiviert serifenlose Matheschrift
\usepackage{microtype}	% harmonische Typenverteilung

% Literatur einbinden
\usepackage[
backend=biber,
style=numeric-comp,
block=ragged
]{biblatex}

\addbibresource{ref/citavi.bib}
\addbibresource{ref/manual.bib}
\addbibresource{ref/carl.bib}
\addbibresource{ref/christoph.bib}

%\usepackage{hyperref}
\usepackage[bookmarks,bookmarksopen,bookmarksnumbered]{hyperref}
\hypersetup{breaklinks=true, %Zeilenumbruch in den Verzeichnissen
	colorlinks=true, %false: boxed links; true: colored links
	citecolor=black,  %color of links to bibliography
	linkcolor=black,   %color of links to internal links
	urlcolor=black
}

% Mathemodus
\usepackage{amsmath,amssymb}

% Trennung
\hyphenation{Crash-zo-ne}

% Bilder einbinden
\usepackage{graphicx}
\graphicspath{{bilder/}}
\usepackage{svg}

% Abbildungen nebeneinander
\usepackage{subfig}
\usepackage{booktabs}

% Kopf- und Fußzeile
\usepackage[
headsepline,	% Kopfzeilen-Sepparationslinie
automark,		% Lebende Kolumnentitel
]
{scrlayer-scrpage}
\pagestyle{scrheadings}	
\ohead{\headmark}

% Anhang
\usepackage[toc,page]{appendix}

% Comments
\usepackage{comment}

% Tables
% \usepackage[table,xcdraw]{xcolor}
\usepackage{adjustbox}
\usepackage{multirow}

% Balkendiagramme
\usepackage{pgfplots, pgfplotstable}

% Für zurätzliche pgfplot Features
\pgfplotsset{compat=1.9}

% Abbildungen nebeneinander
\usepackage{subcaption}

% Diagramme
\usepackage{tikz}
\usepackage{pgfplots}
\usepackage{bchart}

% Tabelle as Figure referenzieren
\usepackage{caption}

% Fußnote in Tabelle
\usepackage{tablefootnote}

% Codebeispiele
\usepackage{minted}
\usepackage{multicol} % added package

\usepackage{chngcntr}

% Trennung von texttt Sektionen
\usepackage[htt]{hyphenat}

% Appendix -> Anhang
\renewcommand\appendixname{Anhang}
\renewcommand\appendixpagename{Anhänge}
\renewcommand\appendixtocname{Anhänge}

%%%%%%%%%%%%%%%%%%%%%%%%%%%%%%%%%%%%%%%%%%%%%%%%%
%%%%%%%%%%%%%%%%%%%%%%%%%%%%%%%%%%%%%%%%%%%%%%%%%

% Titelseite
\titlehead{
	\hfil
	\includegraphics[width=0.3\textwidth]{thi_logo}
	\hfil	
}

\title{\vspace{2ex} Vergleich und Analyse des privaten Modus verschiedener Browser}

\subtitle{ \vspace{8ex} \LARGE Computer-Forensik und Vorfallsbehandlung}

\author{\\ \\Carl Schünemann \\ \\ \vspace{2ex} Christoph Sell}

\date{\vspace{3ex}29.08.2025}



% Extra-Titelseite

\begin{document}
	
	% Titelseite anzeigen
	\maketitle
	
	\pagenumbering{Roman}
	
	
	% Inhaltsverzeichnis
	\tableofcontents
	
	\cleardoubleoddpage
	\pagenumbering{arabic}
	
	
	% Kapitel einbinden
	\include{kapitel/01_Einführung.tex}
	
	\chapter{Theoretischer Hintergrund}\label{chap:theorie}

%Einleitend werden Struktur, Motivation und die abgeleiteten Forschungsfragen diskutiert.
Nachfolgend werden die für das Verständnis dieser Arbeit relevanten Begriffe \textit{Private Browsing}, \textit{Angreifermodell} sowie \textit{Private-Browsing-Artefakte} erläutert.

\section{Private Browsing}\label{chap:theorie-private-browsing}

Ein \textit{Web Browser}, kurz \textit{Browser}, ist eine Softwareanwendung zum Abrufen und Durchsuchen von Informationsquellen im Internet \cite{Rochmadi.2017}. Izuati und Ab Rahman \cite{Izzati.2022} bezeichnen ihn als eine Software, die es Benutzern ermöglicht, das Internet über den von ihrem Dienstanbieter bereitgestellten Zugang zu nutzen. Sie werden für alltägliche Aktivitäten wie das Anschauen von Videos, das Durchsuchen von Websiten, das Posten von Bildern oder Videos in sozialen Medien und das Herunterladen von Dateien genutzt. \cite{Izzati.2022}
%Die bekanntesten Webbrowser sind dabei Google Chrome, Mozilla Firefox, Microsoft Edge und Brave \cite{Izzati.2022}.\\
Beim \textit{normalen} Browsen speichert der Browser alle entstehenden Dateien wie Cache, Cookies, sowie den Suchverlauf auf dem Computer \cite{Izzati.2022}. Um das zu verhindern wurde der sogenannte \textit{private Modus} bei Webbrowsern eingeführt. Diese Funktion ermöglicht das \textit{Private Browsing}, was den Internetnutzern eine größere Kontrolle über ihre Privatsphäre gibt, ohne Rückstände von Datenspuren auf dem Computer zu hinterlassen \cite{Said.2011}. Der private Modus wurde erstmals 2005 mit Apple Safari 2.0 eingeführt \cite{Said.2011}. Drei Jahre später folgte in Google Chrome der \textit{Incognito-Modus} sowie \textit{InPrivate Browsing Modus} für den Internet Explorer. Im Jahr 2009 führte Mozilla Firefox 3.5 mit dem \textit{Privaten Modus} seine Version des privaten Modus ein \cite{Montasari.2015}.

Das genaue Ziel des privaten Modus unterscheidet sich je nach Browser. Beispielsweise sollen bei Mozilla Firefox und Google Chrome besuchte Webseiten ausschließlich auf dem lokalen Computer des Benutzers hinterlassen. \cite{MozillaWiki.05.06.2023,GoogleChrome.} Manche Browser erweitern diesen Schutz, indem sie verhindern, dass beispielsweise Webseitenbetreiber auf Informationen des Private-Browsing Nutzers zugreifen. \cite{Tor.24.05.2023}

Die Nutzer des privaten Modus lassen sich in unterschiedliche Interessensgruppen aufteilen. 
Forensische Ermittler versuchen über Rückstände durchgeführter Browsing-Sessions, Kriminelle mithilfe forensischer Tools und Techniken gezielt zu überführen. \cite{Montasari.2015}. Kriminelle versuchen hingegen gezielt, ihre illegalen Aktivitäten mithilfe des privaten Modus zu verbergen \cite{Mahlous.2020}. 
Herkömmliche Nutzer verwenden den privaten Modus zum Schutz der Privatsphäre. Dieser stellt zusammen mit dem Löschen des Verlaufes die beliebteste Online-Datenschutzmaßnahme dar \cite{Horsman.2019}. 

\section{Browser-Forensik und Browsing-Artefakte}\label{chap:theorie-browser-forensics-artefakte} % Browser Forensiks mit Artefakte 

%Bevor der Begriff der Browser-Forensik eingeführt wird, ist es zunächst wichtig, die digitale Forensik einzuführen. \\
Die \textit{digitale Forensik} konzentriert sich auf die Anwendung forensischer Techniken und Methoden zur Untersuchung von digitalen Geräten, Netzwerken und elektronischen Daten, um digitale Beweise für die strafrechtliche Verfolgung von kriminellen Onlineaktivitäten zu sammeln.
Die gilt es Beweise vollständig und in ihrem Originalzustand zu sichern, um vor Gericht zulässig zu sein. Dazu muss insbesondere der Prozess der Erwerbung, Untersuchung, Analyse sowie Berichterstattung digitaler Beweise forensisch einwandfrei durchgeführt werden.  \cite{Izzati.2022}.

Die \textit{Browser-Forensik} sammelt und identifiziert Beweise und Informationen im Zusammenhang mit einem Verbrechen aus wiederhergestellten Spuren von Browser-Sitzungen. 
%Diese Art der Forensik wird für Ermittler immer wichtiger, da der Suchverlauf, die Download-Aktivität und Seitenaufrufe das Verständnis für das kriminelle Motiv verbessern können. 
Dabei werden nach Informationen über die durchgeführten Browsing-Aktivitäten gesucht, die sogenannten \textit{Browsing-Artefakte}. 
Dies umfasst beispielsweise lokale Dateien, die Informationen wie den Suchverlauf, Cookies, Caches und andere sensiblen Daten enthält. Auch Webseitenbetreiben können Browsing-Artefakte speichern, indem Informationen über die Eigenschaften und das Verhalten der Webseitenbesucher gespeichert werden. Im Rahmen dieser Arbeit werden ausschließlich lokale Browsing-Artefakte untersucht.  \cite{Izzati.2022}
Browsing-Artefakte, die sensible Informationen während einer Private-Browsing-Sitzung speichern, werden als \textit{Private-Browsing-Artefakte}, kurz \textit{PB-Artefakte} bezeichnet.

Aufgrund der \textit{Berweisauthentifizierung} muss ein gefundenes Browsing-Artefakt eindeutig einer Browsing-Aktivität zugeordnet werden, um einen Verdächtigen aufgrund seiner Online-Aktivitäten zu überführen \cite{Mahlous.2020}. 



%Unterschieden kann bei der forensischen Untersuchung dann zwischen der Live und Dead Forensik. Bei der Live Forensik wird im Gegensatz zur traditionellen (dead) Forensik versucht, flüchtige Daten aufzubewahren (RAM, Caches) und Gegenmaßnahmen für verschlüsselte Dateien auf einem Live-System zu ergreifen. \cite{Gupta.2013}. Eine große Herausforderung dabei ist das Kontaminieren von Beweismitteln, was während des Datenerfassungsprozesses geschieht \cite{Gupta.2013}, wie das Ausführen der Software zum Speichern des Arbeitsspeichers, welche selbst im RAM ausgeführt wird. Bei der Dead Forensik hingegen wird der Computer oder das Gerät, das untersucht werden soll, zuerst heruntergefahren, bevor das Speicherabbild erstellt wird bzw. die Datenextraktion beginnt \cite{Izzati.2022}.
%
%Da nun die wichtigsten Begriffe erklärt wurden, folgt anschließend das Ziel dieser Arbeit.
%

\section{Angreifermodell}\label{chap:theorie-angreifermodell}

In der Browser-Forensik bezieht sich das \textit{Angreifermodell} auf die spezifischen Annahmen und Eigenschaften eines potenziellen Angreifers, der auf Browsing-Artefakte abzielt. Aggrawal et al. \cite{Aggarwal.2010} definierten zwei anzunehmenden Angreifer in der Browser-Forensik.

Der sogenannte \textit{Web Attacker} versucht Onlineaktivitäten des Benutzers außerhalb des lokalen Computers zu verfolgen und zu identifizieren. Zum Beispiel kann mittels Tracking-Tools oder durch das Sammeln von Informationen über die IP-Adressen versucht werden, einzelne Benutzer sowie deren Aktivitäten zu identifizieren und nachzuverfolgen. So kann der Internetdienstanbieter den Datenverkehr der Kunden verfolgen, um die Daten unter Zustimmung der Kunden für Marketingzwecke zu monetarisieren \cite{Aggarwal.2010}. 

%Ein Web Attacker hat jedoch im Gegensatz zum lokalen Angreifer keinen tatsächlichen physischen Zugriff auf den Computer, von welchem aus das Browsing durchgeführt wurde.

Im Gegensatz dazu hat der \textit{Local Attacker} physischen Zugriff auf den Computer, von welchem aus das Browsing durchgeführt wurde. Dies kann ein forensischer Prüfer, ein Familienmitglied oder Freund sein, der  beispielsweise versucht, auf den Browserverlauf zuzugreifen. 
Wie im Ziel dieser Arbeit definiert, wird für dieses forensische Experiment der Local Attacker als Angreifermodell angenommen. \cite{Aggarwal.2010}

In einem realistischen Forensischen Szenario hätte der Local Attacker erst Zugriff auf den Computer, nachdem der Benutzer den privaten Modus verlässt, was beispielsweise die Aufzeichnung von Browsing-Aktivitäten ausschließt \cite{Aggarwal.2010}
In der Literatur der Browsing-Forensik wird bei forensischen Versuchen zur Untersuchung des Browser-Verhaltens an dieser Stelle von der Definition des Local Attackers abgewichen. 
Wie im Ziel der Arbeit beschrieben, hat der Local Attacker im Kontext der \textit{transparenten Versuchsdurchführung} vor, während und nach der Browsing-Sitzung vollständigen Zugriff auf den lokalen Computer. \cite{Fayyad.2021, Rochmadi.2017}





	 
	\chapter{Ziel der Arbeit}

Wichtig: White-Box Ansatz gemäß local Attacker in \cite{Aggarwal.2010}
	-	Das Ziel des Angreifers besteht darin, für eine bestimmte Menge von HTTP-Anfragen, die er wählt, festzustellen, ob der Browser eine dieser Anfragen im privaten Browsing-Modus ausgeführt hat oder nicht. Wenn der lokale Angreifer dieses Ziel nicht erreichen kann, gilt die Implementierung des privaten Browsings als sicher.
	- Local Attacker weiß, wonach er sucht!
	


Forensiker müssen Funktionsweise von Private Browsing kennen \cite{Horsman.2019}
	•	Die Kenntnis der Erfolgsrate der PB-Technologie unterstützt die Strafverfolgungsbehörden bei digitalen Untersuchungen von Internetinhalten
	•	Internetbeweise sind oft entscheidend für Untersuchungen
	•	Bestimmung des Umfangs und des Erfolgs von PB-Technologie unterstützt die Strafverfolgungsbehörden bei digitalen Untersuchungen von Internetinhalten
	•	Durch die Bestimmung des Umfangs und des Erfolgs von PB-Technologie können sie unnötige Datenverarbeitung und Zeitverschwendung vermeiden, die Untersuchungseffizienz verbessern und sicherstellen, dass keine wichtigen Inhalte übersehen werden. Daher können diese Punkte dazu beitragen, die Effektivität und Effizienz von Untersuchungen zu verbessern, insbesondere in Fällen, in denen Vor-Ort-Triage stattfindet oder in denen eine SHPO angeordnet wurde. Drei Punkte wichtig:
	•	Wenn der Verdacht besteht, dass PB stattgefunden hat, hilft es zu wissen, wie erfolgreich die PB-Funktion eines bestimmten Browsers ist, um unnötige Datenverarbeitung (und Zeitverschwendung) zu vermeiden, wenn tatsächlich keine Browserdaten auf einem Gerät vorhanden sind.
	•	Die Kenntnis darüber, wo PB möglicherweise Informationen zu Browsing-Sitzungen preisgibt, verbessert die Effizienz von Untersuchungen und verhindert, dass wichtige Inhalte übersehen werden. Dies ist besonders wichtig bei Vor-Ort-Triage, wie sie in einigen Fällen mit einer SHPO angeordnet wird.
	


Ziel der Arbeit:
================
-	Welche Browsing Artefakte werden beim private Browsing auf einem Rechner hinterlassen, welche zeigen, dass eine Browsing Aktion vom Browser durchgeführt wurde?
-	Das heißt: 
	o	Es wird nach Browsing Artefakten gesucht, welche die Zuordnung „Durchgeführte Browsing Aktion“ <-> Browser ermöglichen
	o	Vor, während und nach private Browsing Session nach Browsing Artefakten suchen, welche dem Browser zugeordnet werden können
-	Negativbeispiel: Suche in Hexdump nach im Browser gesuchtem String nicht als Beweis ausreichend, dass private Browsing Artefakte gefunden wurde.
- Kategorisierung nach \cite{Ohana.2013}: 
	> Browsing History
	> Usernames/Email Accounts
	> Images

=> Thematisiert in \cite{Ohana.2013}:
	o	It appeared that the overall best way to recover residual data was to obtain the evidence from RAM or working memory,
	o	Kritik: Oft nur String Match in RAM-Hex als Nachweis für PB genannt -> ausreichend? (Evtl. Gegenexperiment mit Editor)


Warum muss String-Artefakt Browser zugeordnet werden können? \cite{Izzati.2022}
	•	Die Artefakte, die von den Browsing-Aktivitäten eines Kriminellen zurückgelassen wurden, können mit forensischen Tools extrahiert werden, um die Untersuchung des Ermittlers zu unterstützen.
	•	Die erlangten Beweise müssen vor Gericht zugelassen werden, insbesondere digitale Beweise, da sie ohne ordnungsgemäße Verfahren leicht manipuliert werden können.
	•	Es gibt bestimmte Merkmale von digitalen Beweisen, die Gerichte nach folgenden Kriterien akzeptieren:
	1.	Durchsuchungsbefehle - Beweise, die ohne Genehmigung erlangt wurden, können vor Gericht nicht anerkannt werden.
	2.	Berichte - Alle Prozesse, Werkzeuge, Methoden, Techniken, spezifischen Zeit- und Datumsangaben sowie die Beweiskette müssen formell dokumentiert werden, um die Authentizität der digitalen Beweise vor Gericht zu demonstrieren und zu unterstützen.
	3.	Beweisauthentifizierung - Der ursprünglich erhaltene Beweis sollte durch Vergleich der Hash-Werte mit dem Kopiebeweis übereinstimmen. Der erworbene Beweis muss unverändert bleiben, um die Gerichte mit genauen Informationen zu überzeugen. Gerichte akzeptieren Kopien von Beweisen, wenn der ursprüngliche Beweis verloren gegangen oder zerstört wurde, die Kopie jedoch noch intakt ist.
	


Ziele anderer Arbeiten:
=======================
> \cite{Izzati.2022}
	-	Die Art der extrahierbaren Daten zu untersuchen
	-	den Unterschied zwischen privatem und normalem Surfen zu vergleichen 
	-	zu analysieren, welcher Browser die vollständigeren residualen Daten liefert.
> \cite{Montasari.2015}
	•	ob bestimmte Arten von Browser-Daten gefunden werden konnten (Webseiten, Verlauf, Download-Verlauf, besuchte URLs und Suchbegriffe)	
> \cite{Rochmadi.2017}
	•	Das Ziel dieser Studie ist es, eine Rahmenbedingung für die Analysephasen des Webbrowsers im privaten Modus und Anti-Forensik vorzuschlagen, um eine effektive und effiziente forensische Untersuchung zu ermöglichen.
	•	Die Studie nutzt Live-Forensik, um detailliertere Informationen über den Computer zu erhalten, während er noch in Betrieb ist, und eignet sich daher besser für die schnelle Datenerfassung in Echtzeit.
> \cite{Satvat.2014}
	•	umfassende Analyse der privaten Browsing-Funktion in den vier beliebtesten Webbrowsern (IE, Firefox, Chrome und Safari) vorgestellt.
> \cite{Izzati.2022}
	-	digitalen Forensikern helfen, Artefakte von Geräten zu verfolgen, die Live-Memory-Erfassung verwenden
> \cite{Muir.2019}
	-	Methodik entwerfen, um folgende Fragen zu beantworten:
	1.	Kann Tor den Benutzer schützen, indem es Beweise für dessen Nutzung aus dem RAM löscht, wenn die Browsing-Sitzung geschlossen wird?
	2.	Kann die Tor-Nutzung zu vier Schlüsselmomenten erkannt werden: während das Browser-Fenster geöffnet ist, nach Schließen des Browser-Fensters, nach dem Löschen des Installationsverzeichnisses/ zugehöriger Dateien und nach dem Ausloggen des Benutzers?
	3.	Können Dateien aus dem Browsing-Protokoll in der Live-Forensik mit Tor 7.5.2 wiederhergestellt werden, der zum Zeitpunkt der Schreibens aktuellsten Version?
	-	Die Experimente wurden im mobilen Modus mit Tor wiederholt, d.h. von einem USB-Stick ausgeführt.
	(!!!) zu bestätigen, dass die Existenz und Nutzung des Tor-Browsers in Windows 10 nachweisbar ist.
	(!!!) nachweisen, dass Artefakte des Tor-Browsing-Protokolls auf dem Zielcomputer wiederhergestellt werden können.
> \cite{Izzati.2022}
	> In dieser Studie werden die residualen Daten zwischen Google Chrome und Mozilla Firefox Webbrowsern im normalen und privaten Browsermodus mithilfe eines forensischen Tools analysiert und verglichen.
> \cite{Montasari.2015}
	•	In dem Projekt wurden die Effektivität der "privaten" Modus von vier weit verbreiteten Webbrowsern analysiert.
> \cite{Satvat.2014}
	•	Ziel: Bewertung der Sicherheit des privaten Surfens in den Browsern Chrome, Safari, Firefox und IE
	•	Die Autoren haben eine umfassende forensische Analyse durchgeführt, die sowohl Live-Memory-Analyse als auch Post-Mortem-Analyse umfasste.
> \cite{Montasari.2015}
	•	Vier getestet: Firefox, IE, Safari und Chrome
	

Keine Ziele der Arbeit:
=======================
-	zeitl. Kontext nicht wichtig
- 	Private Browsing "Indicators": Entering/Leaving Private Browsing \cite{Ohana.2013}
-	Zeigen, dass ein Browser gestartet/geschlossen wurde
-	Zeigen, dass ein Browser im privaten Modus gestartet wurde
-	Zeigen, wann ein Browser gestartet/geschlossen wurde
- Browser-Erweiterungen: \cite{Satvat.2014}
	> Browser-Erweiterungen und ihre Auswirkungen auf das private Surfen wurden in einer Studie von Aggarwal et al. Im Jahr 2010 untersucht. --> Siehe Punkt „Add-Ons als Leck”
	> Die Chrome-Erweiterung „Incognito Inspector“ kann im privaten Modus genutzt werden, um detaillierte Informationen über die Nutzeraktivitäten zu sammeln und in Echtzeit an einen Remote-Server zu senden.
	> Firefox-Erweiterungen sind standardmäßig im privaten Modus aktiviert und können genutzt werden, um Nutzeraktivitäten aufzuzeichnen.
	> Internet Explorer-Erweiterungen sind in der Regel deaktiviert und erfordern die manuelle Aktivierung im privaten Modus. Die von den Autoren entwickelte Erweiterung funktionierte jedoch nicht, da sie aufgrund eingeschränkter Privilegien nicht auf die BHO-Klasse zugreifen konnte
- \cite{Aggarwal.2010}
	> Unterschiedliche Handhabung durch Browser: Gefährliche Leckage für private Browsing Artefakte
	> Entwickler von Add-Ons haben möglicherweise den privaten Browsing-Modus bei der Entwicklung ihrer Software nicht berücksichtigt, und ihr Quellcode wird nicht derselben rigorosen Überprüfung unterzogen wie die Browser selbst.
	> Gegenmaßnahme: \cite{Aggarwal.2010} 
		•	Es wurde eine Firefox-Erweiterung namens ExtensionBlocker entwickelt, um unsichere Erweiterungen im privaten Modus zu blockieren




	
	\chapter{Methodik}
\label{chapter:methodik}
In der Browserforensik ist eine definierte Methodik notwendig, um die Komplexität moderner Browser zu bewältigen. Sie bildet bildet eine wissenschaftliche Basis für den durchgeführten Versuch sowie einen Leitfaden für Ermittler bei zukünftigen Untersuchungen. \cite{Aggarwal.2010, Izzati.2022, Horsman.2019}	
Ein oft verwendetes Vorgehensmodell in der Computer Forensik ist das "Generic Model Computer Forensics Investigations", kurz GCFIM. \cite{Yusoff.2011}
%Ähnlich zum "abschnittsbasierter Verlauf einer forensischen Untersuchung" des Bundesamt für Sicherheit in der Informationstechnik ist es in Phasen mit definierten Abläufen gegliedert.
% https://www.bsi.bund.de/SharedDocs/Downloads/DE/BSI/Cyber-Sicherheit/Themen/Leitfaden_IT-Forensik.pdf?__blob=publicationFile&v=2

Izzati et al. haben diese Phasen auf Browserforensik abgebildet: \cite{Izzati.2022}
\begin{itemize}
	\item Vorbereitung: Versuchsplanung und Konfiguration der Versuchsumgebung.
	\item Datensammlung: Speicherabbilder identifizieren und während des Browsing Szenarios erstellen. 
	\item Datenanalyse: Suche nach Browsing Artefakten in gesammelten Daten.
	\item Dokumentation: Vorgehensweise und gefundene Artefakte dokumentieren.
\end{itemize}

Die Dokumentationsphase entspricht in dieser Arbeit dem Kapitel "Vergleich der Browser" (TODO!). Die Methodik der anderen Phasen wird nachfolgend beschrieben.

\section{Vorbereitung}
\label{section:methodik-vorbereitung}
In der Vorbereitungsphase wird der durchgeführte Versuch geplant sowie die Versuchsumgebung konfiguriert. \cite{Izzati.2022} Die Versuchsplanung umfasst die Auswahl von Browsern und Tools sowie die Definition der durchzuführenden Schritte zur Kontaminierung des Rechners. Die Konfiguration der Versuchsumgebung umfasst die Installation und Konfiguration der notwendigen Software und Hardware.

\subsection{Browserauswahl}
\label{section:methodik-vorbereitung-browserauswahl}
Für diese Arbeit dazu entschieden: zwei weit verbreitete Brower verwenden + zwei Browser mit verstärkem Schutz der Privatsphäre.

Dazu: Google Chrome und Mozilla Firefox ausgewählt.

Weiterhin werden zwei Browser mit verstärkem Schutz der Privatsphäre ausgewählt.
Basierend auf Chromium wird der Browser "Brave" gewählt.
Für Firefox wird der Tor-Browser gewählt, eine modifizierte Version von Firefox.

\subsubsection*{Firefox}
\label{subsubsection:methodik-vorbereitung-browserauswahl-firefox}
Der Browser Mozilla Firefox, kurz Firefox, ist ein open-source Webbrowser der gemeinnützigen Organisation Mozilla. 
*** TODO: Hier Verweis auf Literatur **
Firefox hat die Funktion des "privaten Modus". Diese ermöglicht es, ohne Speicherung von Verlaufsdaten und Cookies zu im Internet zu Browsen.
Laut Firefox wird mit dem privaten Modus vor dem "Lokalen Angreifer" geschützt, wie er in Kapitel X (TODO!) definiert ist, jedoch nicht vor dem Webangreifer.
Es wird ausdrücklich darauf hingewiesen, dass die besuchten Webseiten und Ihr Internetanbieter (ISP)  weiterhin anhand Ihrer IP-Adresse Informationen über die von Ihnen besuchten Seiten sammeln.
% https://www.mozilla.org/de/firefox/
% https://www.mozilla.org/de/about/history/

% Für diesen Versuch: Firefox Version 112.0.2 (64 Bit)

\subsubsection*{Tor}
\label{subsubsection:methodik-vorbereitung-browserauswahl-tor}
Der Tor Browser, ist ein auf Firefox basierender Webbrowser, der das Tor-Netzwerk nutzt.
Im Gegensatz zu Firefox wird hier mit Schutzmaßnahmen gegen den Webangreifer geworben.
Internetaktivitäten aufgrund des Tor-Netzwerks nicht verfolgt werden.
Der Tor Browser wirbt mit folgendem Schutzmaßnahmen gegen lokalen Angreifer:
- Tor Browser does not keep any browsing history. Cookies are only valid for a single Browsing session.
Zusätzliche Funktion: "Neue Identität"
\begin{figure}[h!]
	\resizebox{\linewidth}{!}{\includegraphics{bilder/tor-new-identity.png}}
%	\label{...}
	\caption{Tabelle mit wiederherstellbaren Dateien: Logfile 1 vs. Logfile 2}
\end{figure}
Die Funktion "Neue Identität" im Tor Browser ermöglicht es, alle aktuellen Tabs und Fenster zu schließen, sämtliche private Informationen wie Cookies und Verlauf zu löschen sowie die Verbindung mit dem Tor-Netzwerk neu zu konfigurieren.
%Für diesen Versuch: Tor Version 12.0.4 (64 Bit)

\subsubsection*{Chrome}
\label{subsubsection:methodik-vorbereitung-browserauswahl-chrome}

\subsubsection*{Brave}
\label{subsubsection:methodik-vorbereitung-browserauswahl-brave}

\subsection{Browsing Szenario}
\label{subsection:methodik-vorbereitung-browsing-szenario}
%*** TODO: Edge vorinstalliert erwähnen ***
% > Image als Hex: \cite{Ohana.2013}
Im Falle der Browser-Forensik werden eine Reihe von Aktivitäten definiert, die für jeden zu untersuchenden Browser durchgeführt werden, das sog. "Browsing Szenario".
Es wird somit definiert, mit welchen Daten der Rechner kontaminiert werden soll.

- Ziel: PB Artefakte, die ausschließlich in Browsing-Szenario vorkommen, bspw. "twitter" oder "facebook" bereits in vielen Windows-Standardanwendungen enthalten.

Folgende Schritte werden in jedem Browser durchgeführt: 

\begin{enumerate}
\item  www.google.com aufrufen
	\begin{enumerate}[label*=\arabic*.]
	\item Alle Cookies akzeptieren 
	\item Google-Suche nach "pfaffenhofen"
	\end{enumerate}
\item www.google.com aufrufen
	\begin{enumerate}[label*=\arabic*.]
	\item Cookies alle akzeptieren 
	\item Google-Suche nach "nanoradar" 
	\end{enumerate}
\item www.google.com aufrufen
	\begin{enumerate}[label*=\arabic*.]
	\item Cookies alle akzeptieren 
	\item Google-Suche nach "mallofamerica"
	\item Auf Suchergebnis "mallofamerica.com" klicken
	\end{enumerate}
\item www.google.com aufrufen
	\begin{enumerate}[label*=\arabic*.]
	\item Cookies alle akzeptieren 
	\item Google-Suche nach "mooserliesl"
	\item Auf Suchergebnis "mooserliesl.de" klicken
	\end{enumerate}
\item "www.unitree.com" über URL-Leiste öffnen
\item "www.donaukurier.de" über URL-Leiste öffnen
	\begin{enumerate}[label*=\arabic*.]
	\item Donaukurier Logo in neuem Tab öffnen
	\end{enumerate}
\item "mail.google.com" über URL-Leiste öffnen
	\begin{enumerate}[label*=\arabic*.]
	\item Mit google Account anmelden: 
			\begin{enumerate}[label*=\arabic*.]
			\item E-Mail = "computerforensikvl@gmail.com"
			\item Passwort = "Vorlesung23!"
			\end{enumerate}
	\item Neue E-Mail schreiben:
			\begin{enumerate}[label*=\arabic*.]
			\item Empfänger: "cas0597@thi.de" und "chs3702@thi.de"
			\item Betreff: "Betrefftext"
			\item Mailinhalt: "Mailinhalt"
			\end{enumerate}			
	\end{enumerate}
\end{enumerate}

Aus diesem Browsing-Szenario lassen sich die in Tabelle X dargestellten "private Browsing Artefakte", kurz PB Artefakte ableiten. Dabei handelt es sich dabei um Strings oder reguläre Ausdrücke, die eindeutig einem Schritt im Browsing-Szenario zugeordnet werden können. Diese sind von zentraler Bedeutung in der Analysephase: Nur nach diesen Strings wird gesucht.

\begin{table}[h!]
\centering
\begin{tabular}{|c|l|c|}
\hline
\textbf{Kategorie}           & \multicolumn{1}{c|}{\textbf{Private Browsing Artefakt}}                                      & \textbf{\begin{tabular}[c]{@{}c@{}}Schritt im\\  Browsing Szenario\end{tabular}} \\ \hline
\multirow{4}{*}{Suchbegriff} & "pfaffenhofen"                                                                               & 1.2                                                                              \\ \cline{2-3} 
                             & "nanoradar"                                                                                  & 2.2                                                                              \\ \cline{2-3} 
                             & "mallofamerica"                                                                              & 3.2                                                                              \\ \cline{2-3} 
                             & "mooserliesl"                                                                                & 4.2                                                                              \\ \hline
\multirow{4}{*}{URL}         & "mooserliesl.de"                                                                             & 3.3                                                                              \\ \cline{2-3} 
                             & "mallofamerica.com"                                                                          & 4.3                                                                              \\ \cline{2-3} 
                             & "unitree.com"                                                                                & 5.                                                                               \\ \cline{2-3} 
                             & "donaukurier.de"                                                                             & 6.                                                                               \\ \hline
Bild                         & \begin{tabular}[c]{@{}l@{}}0x89 0x50 0x4E 0x47 ...\\ (PNG als Hexadezimalwerte)\end{tabular} & 6.1                                                                              \\ \hline
\multirow{4}{*}{E-Mail}      & "computerforensikvl@gmail.com"                                                               & 7.1.1                                                                            \\ \cline{2-3} 
                             & "Vorlesung23!"                                                                               & 7.1.2                                                                            \\ \cline{2-3} 
                             & "cas0597@thi.de"                                                                             & 7.2.1                                                                            \\ \cline{2-3} 
                             & "chs3702@thi.de"                                                                             & 7.2.1                                                                            \\ \hline
\end{tabular}
\end{table}


\subsection*{VM Konfiguration}
\label{subsection:methodik-vorbereitung-vmkonfiguration}
Best Practice in Browser Forensik: Versuche sowie Analysen in virtualisierter Umgebung durchführen. (TODO: Quellen)
- Dadurch Reproduzierbarkeit der Ergebnisse sichergestellt
- Keine Vermischung der PB Artefakte, wenn gleicher Rechner zur Analyse verwendet
- Trennung der Versuchsumgebung von der Analyseumgebung
- Ergebnisse sind transportabel -> Zustände von Virtuellen Maschinen exportierbar

Als Virtualisierungssoftware für Versuch verwendet: Kostenlose Oracle VirtualBox VM.

Pro Browser wird eine VM erstellt, auf der Browsing Szenario durchgeführt wird. Alle VMs basieren auf gleicher OVA, deren Konfiguration in Tabelle X dargestellt ist.

Um Programme auf VM zu installieren: Gemeinsamer Order zwischen VM und Rechner eingerichtet, auf dem VM läuft. Ordner wird in VM als Netzwerklaufwerk angezeigt. Programme auf Analyserechner heruntergeladen, dann in VM offline installiert, um VM nicht vor Browsing-Szenario bereits mit Browsing Artefakten zu kontaminieren.

Auf VM zwei Werkzeuge der Sysinternal-Abteilung von Microsoft installiert, um in Analysephase Browserverhalten vollständig zu untersuchen: "Process Monitor", zur Aufzeichnung aller Prozessaktivitäten sowie "Process Explorer", um die Funktionen des Windows Task Managers zu erweitern. 
% https://learn.microsoft.com/de-de/sysinternals/downloads/procmon
% https://learn.microsoft.com/de-de/sysinternals/downloads/process-explorer

\begin{table}[h!]
\centering
\begin{tabular}{|l|l|}
\hline
\textbf{Betriebssystem}         & Windows 10 Pro, 64 Bit, Build: 19045.2006                                  \\ \hline
\textbf{Festplatte}             & 30 GB, VDI-Format, kein SSD Laufwerk                                       \\ \hline
\textbf{RAM}                    & 6 GB                                                                       \\ \hline
\textbf{Netzwerk}               & Netzwerkbrücke                                                             \\ \hline
\textbf{Verbindung zu Host-PC}  & Gemeinsamer Ordner                                                         \\ \hline
\textbf{Installierte Programme} & \begin{tabular}[c]{@{}l@{}}Process Monitor (Version 3.93)\\ Process Explorer (Version 17.04)\end{tabular} \\ \hline
\end{tabular}
\end{table}

\subsubsection*{Browserinstallation}
Nachdem OVA mit Standardkonfiguration erstellt: Für jeden Browser dupliziert und Browser über gemeinsamen Ordner installiert. Folgende Installationsverzeichnisse verwendet:
\begin{enumerate}
\item[\textbf{Firefox}] \texttt{C:$\backslash$Program Files$\backslash$Mozilla Firefox$\backslash$firefox.exe}
\item[\textbf{Tor}] \texttt{C:$\backslash$Program Files$\backslash$Tor Browser$\backslash$Browser$\backslash$firefox.exe}
\item[\textbf{Chrome}] ***TODO!***
\item[\textbf{Brave}] ***TODO!***
\end{enumerate}

\subsection*{Verwendete Software}
\label{subsection:methodik-vorbereitung-verwendetesoftware}
Neben Konfiguration der VM muss Analyseumgebung vorbereitet werden.
Als Analyseumgebung dient der Rechner, auf dem die VM läuft.
Spezifikationen: *** TODO ***
Dazu: Diverse Tools zur Analyse installieren

\subsubsection*{Autopsy}
\label{subsubsection:methodik-vorbereitung-verwendetesoftware-autopsy}
Um erstellte Festplattenabbilder zu untersuchen: Tool "Autopsy" verwendet.
= Source-Digital-Forensik-Tool, das auf der Sleuthkit-Bibliothek basiert, diese mit zusätzlichen Funktionen erweitert und eine grafische Benutzeroberfläche für die forensische Analyse bietet.  
= Sammlung Befehlszeilen-Tools für die forensische Analyse von Dateisystemen. 
 
%Für diesen Versuch verwendet: Version 4.20.0

\subsubsection*{Volatility}
\label{subsubsection:methodik-vorbereitung-verwendetesoftware-volatility}
Um Abbilder des Arbeitsspeichers zu untersuchen: "Volatility"-Framework verwendet.
= Open-Source-Tool, das speziell darauf ausgerichtet ist, Informationen und Artefakte aus dem physischen oder virtuellen Arbeitsspeicher eines Computers zu extrahieren.
Geschrieben in Python, frei auf GitHub verfügbar:	
%Version: Volatility3, Version 2.4.1 (aktuellster Release)
Für diesen Versuch verwendet: "Volatility3"
= vollständige Neuschreibung des Volatility Memory Forensics Frameworks, wird seit 2020 entwickelt.
%Behebt technische und Performanceprobleme der vorherigen Version.
%Oft beworbender Vorteil von Volatility3: kein "Profile" mehr notwendig. Volatility3 erstellt Symboltabellen für Windows-Speicherabbilder basierend auf dem Speicherabbild selbst.
%Dabei handelt es sich um ein Konfigurationseinstellung, welche die Speicherstruktur und die Verhaltensweisen des Betriebssystems definiert.
	% https://volatility3.readthedocs.io/en/latest/vol2to3.html
	% https://www.volatilityfoundation.org/
	% https://github.com/volatilityfoundation/volatility3
Volatility basiert auf Plugins, welche spezifische Funktionen und Analysen für verschiedene Aspekte des Systems bereitstellen. Für diesen Versuch werden folgende Plugins verwendet. 
\begin{itemize}
\item pslist	
\item yarascan		
\item memmap	 	
\item filescan
\item svcscan
\end{itemize}
Genaue Beschreibung der PlugIns und deren Zusammenhang: Siehe Analysephase in Kapitel X (TODO!).

\subsubsection*{Sonstige Tools}
\label{subsubsection:methodik-vorbereitung-verwendetesoftware-sonstigetools}
Tabelle X listet zusammenfassend alle in diesem Versuch verwendeten Software-Programme, deren Zweck sowie Version auf.
Darunter diverse zusätzliche unterstützende Tools, welche zur vollständigen Analyse benötigt werden

\begin{table}[]
\resizebox{\linewidth}{!}{
\begin{tabular}{|l|l|l|}
\hline
\multicolumn{1}{|c|}{\textbf{Software}} & \multicolumn{1}{c|}{\textbf{Zweck}}                                              & \multicolumn{1}{c|}{\textbf{Version}} \\ \hline
Oracle VirtualBox                       & Virtualisierung                                                                  & 7.0.8 r156879                         \\ \hline
Windows 10 Pro                          & VM Betriebssystem                                                                & Build: 19045.2006                     \\ \hline
Process Monitor                         & Aufzeichnung Prozessaktivitäten                                                  & 3.93                                  \\ \hline
Process Explorer                        & Darstellung der Eigenschaften aktueller Prozesse                                 & 17.04                                 \\ \hline
Autopsy                                 & Analyse Festplattenabbilder                                                      & 4.20.0                                \\ \hline
Volatiltiy                              & Analyse RAM-Abbilder                                                             & Volatility3 Version 2.4.1             \\ \hline
HxD                                     & Analyse Binärdateien in hexadezimaler und ASCII-Darstellung                      & 2.5.0.0                               \\ \hline
Notepad++                               & Analyse strukturierter Dateiformate, z.B. JSON, XML                              & 8.4.5                                 \\ \hline
Registry Explorer                       & Grafischer Oberfläche zur Untersuchung von Windows-Registry Hives                & 2.0.0.0                               \\ \hline
DB Browser for SQLite                   & Grafische Oberfläche zur Verwaltung und Untersuchung von SQLite-Datenbanken      & 3.12.2                                \\ \hline
sqldiff.exe                             & Befehlszeilen-Programm zur Anzeige von Unterschieden zwischen SQLite-Datenbanken & 3.42.0                                \\ \hline
ChromeCacheView                         & Einlesen von Chrome Cache-Dateien und visuelle Aufbereitung des Inhalts          & 2.46                                  \\ \hline
MZCacheView                             & Einlesen von Firefox Cache-Dateien und visuelle Aufbereitung des Inhalts         & 2.21                                  \\ \hline
FirefoxCache2                           & Erweitert MZCacheView, um Firefox "index"-Cachedatei zu analysieren              & Commit b50ab4f                        \\ \hline
dejsonlz4                               & Dekomprimierung von .jsonlz4-Dateien                                             & Commit c4305b8                        \\ \hline
\end{tabular}
}
\end{table}


\section{Datensammlung}
\label{section:methodik-datensammlung}
In der Phase der Datensammlung werden alle potenziellen Beweismittel identifiziert und in einem analysierbaren Format gesichert.\cite{Izzati.2022}

Für diesen Versuch umfasst dies die Durchführung des Browsing Szenarios sowie die Sammlung von Ressoucen, die potentielle private Browsing Artefakte enthalten.

\subsection*{Process Monitor Logfiles}
\label{subsection:methodik-datensammlung-processmonitorlogfiles}
Um das Verhalten von privaten Browsingmodi möglichst vollständig zu untersuchen, schlagen Fayyad-Kazan et al. \cite{Fayyad.2021} vor, alle Aktivitäten des Browsers während Browsing-Szenarios aufzeichnen.
Dazu werden mit dem Tool "Process Monitor" alle Prozess-Aktivitäten zwischen zwei Zeitpunkten als "Process Monitor Logfile" (PML) oder CSV-Datei gespeichert. \cite{Fayyad.2021, Rochmadi.2017}
Die PML-Dateien werden mithilfe des gemeinsamen Ordners auf den Analyserechner transportiert.

\subsection*{Speicherabbilder}
\label{subsection:methodik-datensammlung-speicherabbilder}
Eine der Hauptaufgaben eines Computer-Forensischen-Ermittlers ist die Erstellung und Analyse von direkten Kopien der Speichermedien des untersuchten Rechners. \cite{Hassan.2019}
Im Falle der Browser Forensik werden Abbilder der Festplatten und des Arbeitsspeicher erstellt und analysiert.

\paragraph*{Festplatten-Image}
Da in diesem Versuch die Festplatten virtualisiert wurden, wird ein Abbild aus einem sogenannten "VM-Snapshot" gewonnen, eine Momentaufnahme der virtuellen Maschine.  % https://docs.oracle.com/en/virtualization/virtualbox/6.0/user/snapshots.html
VM-Snapshots können "aufgetaut" werden, wodurch der Zeitpunkt der Momentaufnahme des Betriebssystems wiederhergestellt wird.
Bei Oracle VirutalBox kann ein VM Snapshot über die grafische Oberfläche erstellt werden.
Durch den Snapshot wird ein "Virtual Disk Image", eine VDI-Datei, im Snapshot-Ordner der VM erzeugt. Diese Laufwerksdatei enthält nur differentielle Daten zum vorherigen Snapshot.
Um aus den differentiellen Daten ein vollständiges Festplatten-Image zu erzeugen muss ein "vollständiger Klon" des Snapshots erstellt werden. Die VDI-Datei der geklonten VM entspricht einem vollständigem Abbild der Festplatte zum Zeitpunkt des durchgeführten Snapshots.

Da Autopsy nicht das VDI-Format unterstützt, müssen die Laufwerksdateien der geklonten Snapshots in ein das generische Image-Format (.img) umgewandelt werden.
Durch Nutzung des VirtualBox Befehlszeilen-Tool "vboxmanage" wird mit dem Befehl \texttt{vboxmanage clonehd <VDI\_File>.vdi <IMG\_File>.img --format raw} die VDI-Datei ein eine IMG-Datei umgewandelt.

*** TODO: Hier Einlesen der Festplatten-Images => Zeitaufwändig! ***

\paragraph*{RAM-Dump}
Ein "RAM-Dump" erfasst den Zustand des Arbeitsspeichers, einschließlich der im Speicher befindlichen Daten, Programme und Prozesse zu einem bestimmten Zeitpunkt.
% https://kollee.github.io/posts/memory-forensics-of-a-virtualbox-vm/
VirtualBox empfiehlt, Abbilder des RAMs ebenfalls über das "vboxmanage" Befehlszeilen-Tool durchzuführen.
Im Unterschied zu Festplatten-Images können RAM-Dumps nur im angeschalteten Zustand der virtellen Maschine mithilfe des Befehls \texttt{vboxmanage debugvm <VM Name> dumpvmcore --filename "<RAM Dump Dateiname>.elf} durchgeführt werden. RAM-Dumps im .elf Format können direkt vom Analysetool Volatility eingelesen werden.		

\paragraph*{Zeitpunkte zur Datensammlung}
Wichtig für die Qualität der Versuchsergebnisse sind die Zeitpunkte während des Browsing Szenarios zum Sammeln der Daten.
Dieses Thema wird in der Literatur kaum thematisiert. Die Autoren wählen die Zeitpunkte meist ohne Begründung \cite{Sajan.2021, Nalawade.2016, Montasari.2015, Satvat.2014, Said.2011, Aggarwal.2010}.

Dieses Problem haben Muir, Leimich und Buchanan erkannt und Zeitpunkte zur Datensammlung vorgeschlagen, um das Browserverhalten während des Browsing-Szenarios vollständig analysieren zu können \cite{Muir.2019}. Wie in Abbildung X (TODO!) dargestellt, wurde sich an diesen Zeitpunkten für diesen Versuch orientiert.
\begin{figure}[h!]
	\centering
	\small
	\centerline{\resizebox{\linewidth}{!}{\input{bilder/datensammlung-zeitpunkte-Latex.pdf_tex}}}
	\caption{Datensammlung Zeitpunkte}
	\label{fig:jes}
\end{figure}
*** TODO: Kürzen ***
Nach der Browser-Installation, vor Beginn des Browsing-Szenarios wird der erste RAM-Dump sowie der erste VM-Snapshot erstellt.
Diese Speicherabbilder dienen als Benchmark für die Analyse, da in diesen Speicherabbildern kein PB Artefakt gefunden werden darf.

Nachdem der Private Modus im Browser geöffnet wird, bevor das Browsing Szenario beinnt wird die Aufnahme des ersten Process Monitor Logfiles gestartet.
Die Aufzeichnung beginnt erst zu diesem Zeitpunkt, da beim erstmaligem Öffnen der Browser einige Dateien initial angelegt werden. Um ausschließlich Schreiboperationen aufzuzeichnen, die auf das private Browsing zurückzuführen sind, wird die Aufzeichnung erst nach dem erstmaligen Öffnen des Browsers im privaten Modus gestartet.

Nach Durchführung des Browsing-Szenarios, während der Browser noch geöffnet ist, wird die Aufnahme des ersten Process Monitor Logfiles beendet. Weiterhin wird ein zweiter RAM-Dump sowie VM-Snapshot erstellt. Anschließend wird das eine zweite Process Monitor Aufzeichnung gestartet. 
Somit enthält das erste Logfile zu diesem Zeitpunkt die Prozessaktivitäten während des Browsing Szenarios. 

Nachdem der Browser geschlossen wurde, wird die Aufzeichnung des zweiten Process Monitor Logfiles beendet. Zusätzlich wird ein dritter RAM-Dump sowie  VM-Snapshot erstellt. Somit enthält das zweite Logfile wird alle Prozessaktivitäten vom Schließen der Browsers.

Nach Herunterfahren der VM wird ein vierter VM-Snapshot erstellt, der für die für Post-Mortem Analyse relevant ist.

\paragraph*{Sonderfälle}
Dieses Vorgehen zur Datensammlung wird bei allen Browsern durchgeführt. Einzig der Tor-Browser weicht davon ab. Um die "Neue Identität"-Funktion des Tor-Browsers zu berücksichten, werden zusätzlich Daten vor und nach der Erstellung einer "Neuen Identiät" gesammelt. Wie in Abbildung X (TODO!) dargestellt, umfasst dies einen zusätzlichen RAM-Dump sowie VM-Snapshot und ein weiteres Process Monitor Logfile.
\begin{figure}[h!]
	\centering
	\small
	\centerline{\resizebox{\linewidth}{!}{\input{bilder/datensammlung-zeitpunkte-tor-Latex.pdf_tex}}}
	\caption{Datensammlung Zeitpunkte Tor}
	\label{fig:jes}
\end{figure}	
Bei Durchführung des Browsing-Szenarios für den Firefox-Browser wurde nach erstmaligem Öffnen des Browsers automatisch die Firefox Datenschutz-Webseite \texttt{https://www.mozilla.org/de/privacy/firefox/} im nicht-privaten Modus geöffnet. 

\section{Datenanalyse}
\label{section:methodik-datenanalyse}
Nachdem Daten in Form von Process Monitor Logfiles und Festplatten- sowie RAM-Speicherabbildern gesammelt: in gesammelten Daten nach Strings in Tabelle X aus Kapitel X (TODO!) suchen. 

Um die Analyse zu vereinfachen: die gesammelten Daten des Versuchs lassen sich in drei Kategorien aufteilen:
> Common Locations
> Uncommon Locations	
> Registry

\subsection{Common Locations}
\label{subsection:methodik-datenanalyse-commonlocations}
*** TODO: Wichtig: NUR FESTPLATTE, NICHT RAM ***
Die sogenannten "Common Locations", (dt. "gängige Speicherorte") beziehen im Zusammenhang der Browserforensik auf die standardmäßigen Verzeichnisse eines Browsers, beispielsweise Ordner von Browsern zur Verwaltung von Nutzerdaten.
Untersucht werden Common Locations mittels "Whitebox-Analyse" \cite{Bonetti.2014}
Der Fokus liegt darauf, das System vollständig zu verstehen und alle relevanten Beweise zu sammeln.

Bei diesem Versuch werden die Browser Speicherorte über die Schreiboperationen der Process Monitor Logfiles identifiziert.
Anschließend wird für jede Datei in den Speicherorten geprüft, ob PB Artefakte enthalten sind.
Dazu sind zwei Schritte notwendig:
\begin{enumerate}
\item Dateiextraktion: Extrakion der Datei aus dem Speicherabbild. Wenn die Datei nicht mehr vorhanden ist, werden dazu ggf. Tools zur Dateiwiederherstellung benötigt.
\item Dateianalyse: Um zu überprüfen ob die Datei PB Artefakte enthält, werden ggf. Tools für spezielle Dateiformate benötigt, beispielsweise Dekomprimierungstools.
\end{enumerate}

\subsubsection*{Process Monitor Logfiles}
\label{subsubsection:methodik-datenanalyse-commonlocations-processmonitorlogfiles}
\paragraph*{Identifikation der Common Locations}
Um die gängigen Browserpfade und -dateien zu identifizieren, werden die in den Process Monitor Logfiles aufgezeichneten Schreibaktivitäten der Browserprozesse ausgewertet.

Dazu wird jede Process Logfile mit dem Process Explorer eingelesen. Anschließend werden die Aktivitäten gefiltert.
\begin{figure}[h!]
	\centerline{\resizebox{0.7\linewidth}{!}{\includegraphics{bilder/process-monitor-filter.png}}}
%	\label{...}
	\caption{Tabelle mit wiederherstellbaren Dateien: Logfile 1 vs. Logfile 2}
\end{figure}
Wie in Abbildung X dargestellt werden dazu ausschließlich die Option "File System Activity" ausgewählt.
Anschließend wird als Prozessname der Browserprozess gesetzt:
\begin{itemize}
\item[\textbf{Firefox}] firefox.exe
\item[\textbf{Tor-Browser}] firefox.exe und tor.exe
\item[\textbf{Chrome}] chrome.exe
\item[\textbf{Brave}] brave.exe
\end{itemize}
Da PB Artefakte nur über Schreiboperationen Operationen entstehen können, wird als Prozessoperation "WriteFile" gesetzt.
Die gefilterte Logfile wird als CSV exportiert, um sie dann in Excel zu öffnen und irrelevante Spalten sowie Duplikate zu löschen.
Die geschriebenen Dateien werden anschließend browserspezifisch gruppiert.

\paragraph*{Prüfung auf PB Artefakte}
Nachdem die geschriebenen Browserdateien identifiziert und kategorisiert wurden, wird für jede Datei geprüft, ob PB Artefakte enthalten sind. Folgende, in Abbildung X (TODO!) dargestellte Schritte sind zur Dateiextraktion und Dateianalyse notwendig:
Die Datei befindet sich entweder im entsprechenden Festplatten-Image oder ist im RAM-Dump gespeichert. 
Wenn die Datei nicht mit Autopsy aus dem Festplatten-Image extrahiert werden kann und sich der Dateiname in der Ausgabe des Volatiltiy Plugins "filescan" (\texttt{vol.py -f ram\_dump.img windows.filescan > filescan.txt}) befindet, wird diese mit dem Volatility Plugin "dumpfiles" aus dem RAM extrahiert.
Wenn auch dies nicht möglich ist und es sich um eine temporäre Datei (.tmp) handelt, wird versucht die enstprechende nicht-temporäre Datei zu extrahieren. 
Im Falle der Datei "some-file.json.tmp" wird beispielsweise geprüft, ob die Datei "some-file.json" existiert.
Nachdem die Datei extrahiert wurde und ggf. mit einem Tool zu Analyse vorberarbeitet wurde, wird geprüft, ob die Datei PB Artefakte enthält.
\begin{figure}[h!]
	\centering
	\small
	\centerline{\resizebox{\linewidth}{!}{\input{bilder/process_monitor_to_exce-Latexl.pdf_tex}}}
	\caption{TODO: Process Monitor Write Operation to Excel Spreadsheet}
	\label{fig:jes}
\end{figure}

\subsubsection*{SQLite-Datenbänke}
\label{subsubsection:methodik-datenanalyse-commonlocations-sqlitedbs}
Eine besondere Rolle unter den Common Locations bei Browsern nehmen SQLite Datenbänke ein. 
Sie ermöglichen Browsern das Speichern und Verwalten von Nutzerinformationen, wie Lesezeichen, Browserverlauf, Caches, Cookies in Datenbankdateien zu speichern, ohne einen separaten Datenbankserver zu benötigen.

Wie in Abbildung X dategestellt erfolgt die Dateiextraktion analog zur Vorgehensweise bei den Schreiboperationen der Process Monitor Logfiles.

Um die SQLite Datenbänke zu analysieren wird jede Datenbank mit der gleichen Datenbank aus dem vorherigem Snapshot mithilfe des Befehlszeilentools "sqldiff.exe" (\texttt{sqldiff.exe database1.sqlite database2.sqlite}) verglichen. Die Inhaltsunterschiede werden für jede Datei in jedem Snapshot untersucht und in einer Excel Tabelle festgehalten.
Datenbankänderungen einer SQLite-Datei werden zuerst im "Write-Ahead Log", kurz WAL, vorübergehend protokolliert. 
Um potentielle PB Artefakte zu berücksichtigen, wird der WAL mithilfe der sqlite3 Befehlszeile (\texttt{sqlite3> PRAGMA wal\_checkpoint;}) in die SQLite Hauptdatenbank geschrieben.

\begin{figure}[h!]
	\centering
	\small
	\centerline{\resizebox{\linewidth}{!}{\input{bilder/sqlite-methodology-Latex.pdf_tex}}}
	\caption{TODO: Process Monitor Write Operation to Excel Spreadsheet}
	\label{fig:jes}
\end{figure}

\subsection{Uncommon Locations}
\label{subsection:methodik-datenanalyse-uncommonlocations}
Ungewöhnliche Speicherorte beziehen sich auf Verzeichnisse, die nicht zu den gängigen Speicherorten gehören. 
Bei Festplatten-Images handelt es sich dabei meist um Dateien des Betriebssystems oder andere Festplattenbereiche, wie beispielsweise unallokierte Speicherbereiche oder der Arbeitsspeicher.
Uncommon Locations werden ohne Vorwissen über das Browserverhalten sowie ohne Vorverarbeitung der Dateien mithilfe der "Blackbox-Analyse" untersucht \cite{Bonetti.2014}:
Im Kontext der Browser Forensik werden dazu Stringsuchen nach PB Artefakten über die gesamten Speicherabbilder durchgeführt.
Dies ist nur durch Unterstützung mit Forensik-Tools möglich. Somit wird bei der Analyse der Uncommon Locations in die Vollständigkeit der Tools vertraut.

\subsubsection*{Analyse mit Autopsy}
\label{subsubsection:methodik-datenanalyse-uncommonlocations-analysemitautopsy}
Bei den Uncommon Locations wird Autopsy als forensisches Werkzeug zur Analyse der Festplatten-Images verwendet.
Dazu wird eine Stichwortsuche mit den in Tabelle X definierten PB Artefakten über das gesamte eingelesene Festplatten-Image durchgeführt.

Autopsy bietet dazu die in Abbildung X dargestellte Funktion an, zur Suche nach Strings, Teilstrings oder regulären Ausdrücken in Dateinamen und Dateiinhalten.
\begin{figure}[h!]
	\centerline{\resizebox{0.85\linewidth}{!}{\includegraphics{bilder/autopsy-search.png}}}
%	\label{...}
	\caption{Tabelle mit wiederherstellbaren Dateien: Logfile 1 vs. Logfile 2}
\end{figure}

Zusätzlich kategorisiert Autopsy automatisch die Dateien eines Festplatten-Images. Für diesen Versuch sind folgende Dateikategorien von Interesse:
\begin{itemize}
\item Web Bookmarks
\item Web Cookies
\item Web History
\item Web Categories
\end{itemize}


\subsubsection*{Analyse mit Volatility}
\label{subsubsection:methodik-datenanalyse-uncommonlocations-analysemitvolatility}
Bei der Analyse des Arbeitsspeichers als Uncommon Location ist es kritisch, dass ein gefundener String eindeutig einem Browserprozess zugeordnet werden kann. 

In der Literatur wird der Arbeitsspeicher oft unzureichend analysiert, indem eine Stringsuche im RAM-Dump durchgeführt ist, der als Binärdatei in einem Hexadezimaleditor geöffnet ist. \cite{Rochmadi.2017, Md.2018, Montasari.2015}
Wie beispielhaft in Abbildung X und Y gezeigt, wird ein String, der in einer Textdatei auf dem Desktop gespeichert ist ebenfalls im Hexadezimaleditor HxD angezeigt, obwohl kein Browsing Szenario durchgeführt wurde.
\begin{figure}[h!]
	\centerline{\resizebox{0.5\linewidth}{!}{\includegraphics{bilder/ram-editor1.png}}}
%	\label{...}
	\caption{Tabelle mit wiederherstellbaren Dateien: Logfile 1 vs. Logfile 2}
\end{figure}
\begin{figure}[h!]
	\resizebox{\linewidth}{!}{\includegraphics{bilder/ram-editor2.png}}
%	\label{...}
	\caption{Tabelle mit wiederherstellbaren Dateien: Logfile 1 vs. Logfile 2}
\end{figure}

Um einen im RAM gefundenen String einem Browserprozess zuordnen zu können, wird deshalb das forensiche Analysetool Volatility mit dem PlugIn "Yarascan" verwendet.

Mithilfe sogenannter "YARA-Regeln" wird bestimmten Mustern im Arbeitsspeicher gesucht.
Die für diesen Versuch verwendeten Yara-Regeln entsprechen den Strings der PB Artefakte in Tabelle X in Kapitel Y. Zusätzlich sucht eine Regel nach HTML-Fragmenten, die eindeutig einer besuchten Seite des Browsing-Szenarios zuzuordnen sind. \cite{Said.2011}
Alle verwendeten Yara-Regeln sind im Anhang X (TODO!) aufgelistet.

Um den RAM-Dump nach den Yara-Regeln zu durchsuchen wird folgender Befehl ausgeführt: \texttt{vol.py -f ram\_dump.img windows.vadyarascan --yara-file yara\_rules.yara > yarascan.txt}
Nachdem der RAM-Dump nach den Regeln durchsucht wurde, gibt die Yarascan-Ausgabe für jeden gefundenen String die PID des Prozesses, in dem der String gefunden wurde sowie die virtuelle Speicheradresse des gefundenen Strings an.

Wie in Abbildung X dargestellt, wird davon ausgehend mit dem Plugin "pslist" (\texttt{vol.py -f ram\_dump.img windows.pslist --pid <PID> > pslist.txt}) der Prozessname der PID ermittelt, in dem der String gefunden wurde.
\begin{figure}[h!]
	\centering
	\small
	\centerline{\resizebox{\linewidth}{!}{\input{bilder/yarascan_plugin_tree-Latex.pdf_tex}}}
	\caption{TODO: Process Monitor Write Operation to Excel Spreadsheet}
	\label{fig:jes}
\end{figure}

Oft ist für bei einem gefundenen String von Interesse, ob in den Speicheradressen vor und nach dem Treffer weitere Zusammenhänge erkennbar sind.
Mithilfe des Plugins "memmap" (\texttt{vol.py -f ram\_dump.img windows.memmap --pid <PID> > memmap.txt}) wird die Abbildung der virtuellen Speicheradressen eines Prozesses auf die Byte-Offset der extrahierten Speicherseite des Prozesses ermittelt.
Diese Seite kann mithilfe des "--dump" Flags extrahiert werden: \texttt{vol.py -f ram\_dump.img -o $\backslash$dump\_dir$\backslash$ windows.memmap --pid <PID> --dump}.
In einem Hexadezimaleditor, wie HxD, kann der String-Treffer anhand des ermittelten Byte-Offsets in der Speicherseite untersucht werden.

\subsection{Registry}
\label{subsection:methodik-datenanalyse-registry}
Die letzte Kategorie analysierter Daten umfasst die Artefakte der Registry.
Diese zählen sowohl zu den Common als auch Uncommon Locations und werden deshalb eigene Kategorie aufgeführt.

\paragraph*{Common Locations}
*** TODO: Common location: Shellactivities Key ***
	existiert nicht mehr --> Nicht mehr vorhanden in aktueller Version (Verweis auf E-Mail)

Als Teil der Common Locations werden die Registry-Aktivitäten in den Process Monitor Logfiles analysiert.
\begin{figure}[h!]
	\centerline{\resizebox{0.7\linewidth}{!}{\includegraphics{bilder/process-monitor-filter-registry.png}}}
%	\label{...}
	\caption{Tabelle mit wiederherstellbaren Dateien: Logfile 1 vs. Logfile 2}
\end{figure}
Wie in Abbildung X gezeigt, wird zunächst die Logfile ausschließlich nach "Registry Activity" sowie Einträgen mit der Operation "SetValue" sowie dem Browser-Prozessnamen gefiltert.
Als CSV Datei wird das Logfile in Excel weiter verarbeitet, indem Duplikate gelöscht werden und die geschriebenen Registry Keys browserspezifisch gruppiert werden.
	
\paragraph*{Uncommon Locations}
Als Uncommon Location werden alle Registry Hives in jedem Festplatten-Image mit dem "Registry Explorer" untersucht.
Dabei wird zwischen Hives zur Speicherung von Systemeinstellungen (System-Hives) und individuellen Benutzerkonfigurationen (User-Hives) unterschieden. Diese in Tabelle X dargestellten Hives werden von Windows beim Start geladen und dienen Systemkomponenten und Anwendungen als Quelle für Einstellungen und Informationen.	
 % https://medium.com/@haircutfish/tryhackme-windows-forensics-1-task-3-accessing-registry-hives-offline-task-4-data-acquisition-b440f5be2a13
Zur Analyse wird jeder Hive aus den Fesplatten-Images extrahiert und in eine Registry Explorer Sitzung geladen und eine Stringsuche nach PB Artefakten durchgeführt.

\begin{table}[h!]
\resizebox{\linewidth}{!}{
\begin{tabular}{lllll}
\cline{1-2}
\multicolumn{2}{|c|}{\textbf{System-Hives (C:\textbackslash{}\textbackslash{}Windows\textbackslash{}\textbackslash{}System32\textbackslash{}\textbackslash{}Config)}} &  &  &  \\ \cline{1-2}
\multicolumn{1}{|l|}{\textbf{Dateiname}}             & \multicolumn{1}{l|}{\textbf{Inhalt}}                                                                           &  &  &  \\ \cline{1-2}
\multicolumn{1}{|l|}{\textit{DEFAULT}}               & \multicolumn{1}{l|}{Standardkonfigurationseinstellungen für neue Benutzerprofile.}                             &  &  &  \\ \cline{1-2}
\multicolumn{1}{|l|}{\textit{SAM}}                   & \multicolumn{1}{l|}{Sicherheitskontensdaten, einschließlich der Benutzerkonten und deren Kennwörter.}          &  &  &  \\ \cline{1-2}
\multicolumn{1}{|l|}{\textit{SECURITY}}              & \multicolumn{1}{l|}{Sicherheitsinformationen für die Zugriffssteuerung und Authentifizierung.}                 &  &  &  \\ \cline{1-2}
\multicolumn{1}{|l|}{\textit{SOFTWARE}}              & \multicolumn{1}{l|}{Konfigurationsdaten für installierte Software und Anwendungen.}                            &  &  &  \\ \cline{1-2}
\multicolumn{1}{|l|}{\textit{SYSTEM}}                & \multicolumn{1}{l|}{Systemkonfigurationseinstellungen und Gerätetreiberinformationen.}                         &  &  &  \\ \cline{1-2}
                                                     &                                                                                                                &  &  &  \\ \cline{1-2}
\multicolumn{2}{|c|}{\textbf{User-Hives (C:\textbackslash{}\textbackslash{}Users\textbackslash{}\textbackslash{}\textless{}username\textgreater{})}}                  &  &  &  \\ \cline{1-2}
\multicolumn{1}{|l|}{\textbf{Dateiname}}             & \multicolumn{1}{l|}{\textbf{Inhalt}}                                                                           &  &  &  \\ \cline{1-2}
\multicolumn{1}{|l|}{\textit{NTUSER.DAT}}            & \multicolumn{1}{l|}{Individuelle Einstellungen und Konfigurationen für den angemeldeten Benutzer}              &  &  &  \\ \cline{1-2}
\multicolumn{1}{|l|}{\textit{USRCLASS.DAT}}          & \multicolumn{1}{l|}{Dateizuordnungen und Registrierungseinstellungen für den angemeldeten Benutzer}            &  &  &  \\ \cline{1-2}
\end{tabular}
}
\label{tab:windows-registry-hives}
\caption{Windows Registry Hives}
\end{table}
	


	
	\chapter{Ergebnisse}
\label{chapter:ergebnisse}

In diesem Kapitel werden die vier Browser Mozilla Firefox, Tor-Browser, Google Chrome sowie Brave gemäß definierter Methodik in Kapitel \ref{chapter:methodik} analysiert. Dabei wird für jeden Browser untersucht, ob private Browsing Artefakte in den Common Locations, Uncommon Locations und der Registry hinterlassen wurden. Diese Analyse wird an entsprechenden Stellen durch ausführliche Untersuchungen im Anhang ergänzt.

% Hier evtl. noch schreiben, dass Ausführliche Analyse im Anhang zu finden ist

\section{Firefox}
\label{section:ergebnisse-firefox}
Im nachfolgenden Abschnitt werden die Ergebnisse der Datenanalyse für den Webbrowser Firefox beschrieben. Die Analyse ist in drei Hauptkategorien unterteilt: Common Locations, Uncommon Locations und Registry.

\subsection*{Common Locations}
\label{subsection:ergebnisse-firefox-commonlocations}
Zunächst werden die Common Locations nach potentiellen privaten Browsing Artefakten untersucht. Diese standardmäßigen Speicherorte für Browserartefakte beziehen sich ausschließlich auf die Festplatte geschriebene Dateien. In diesem Versuch wurde gemäß Methodik in Kapitel \ref{subsection:methodik-datenanalyse-commonlocations} zwischen Datei-Schreiboperationen aus den Process Monitor Logfiles und SQLite Datenbänken zur Verwaltung von Nutzerdaten unterschieden. Weder in den Schreiboperationen der Process Monitor Logfiles noch in den SQLite-Datenbänken konnten PB Artefakte gefunden werden. 

Eine detaillierte Analyse der untersuchten Datein im Anhang \ref{subsection:appendix-firefox-common-locations} beschrieben.

\subsection*{Uncommon Locations}
\label{subsection:ergebnisse-firefox-uncommonlocations}
Nachfolgend werden die Analyseergebnisse der Firefox Uncommon Locations beschrieben.
Im Gegensatz zu den Common Locations benötigt ein Forensiker dabei kein Wissen über das Browserverhalten. Stattdessen wird sich auf die Vollständigkeit der Funktionen von Forensik-Tools verlassen. Im Rahmen dieses Versuchs werden die Tools Autopsy und Volatility verwendet.

\subsubsection*{Analyse mit Autopsy}
\label{subsubsection:ergebnisse-firefox-uncommonlocations-analysemitautopsy}
Zur Analyse der Common Locations in Kapitel \ref{subsection:ergebnisse-firefox-commonlocations} wird Autopsy zur Dateiextraktion genutzt. Im Falle der Uncommon Locations dient Autopsy zusätzlich als forensisches Werkzeug zur Datenanalyse.

Eine Autopsy Stichwortsuche gemäß Methodik in Kapitel \ref{subsubsection:methodik-datenanalyse-uncommonlocations-analysemitautopsy} lieferte in allen Snapshots keine Treffer. Dabei wurde zusätzlich das \texttt{\$Carved} Verzeichnis durchsucht, in dem Autopsy alle wiederhergestellten Dateien speichert.

Ebenso wurden in den von Autopsy automatisch kategorisierten Dateien keine PB Artefakte gefunden. Eine detaillierte Analyse der Kategorien ``Web Bookmarks``, ``Web Cookies``, ``Web History`` sowie ``Web Categories`` ist im Anhang \ref{subsubsection:appendix-firefox-uncommon-locations-autopsy} beschrieben.

\subsubsection*{Analyse mit Volatility}
\label{subsubsection:ergebnisse-firefox-uncommonlocations-analysemitvolatility}
Zur Untersuchung des RAM als Uncommon Location wurde eine Stringsuche in den gesamten Arbeitsspeicherabbildern nach PB Artefakten durchgeführt.
Wie in Kapitel \ref{subsubsection:methodik-datenanalyse-uncommonlocations-analysemitvolatility} ausführlich beschrieben, muss ein gefundener String eindeutig einem Browser zugeordnet werden können. 
Dazu wird mit dem Volatility PlugIn \textit{Yarascan} nach den in Anhang \ref{appendix:yara-regeln} aufgeführten Yara-Regeln im RAM gesucht. Davon ausgehend wird das PlugIn \textit{pslist} verwendet, um den Prozessnamen anhand PID zu identifizieren.
Die Ergebnisse dieser Stringsuche sind nachfolgend nach den Yara-Regeln geordnet.

\paragraph*{Yara-Regel ``HTML``}
In keinem der Firefox RAM Dumps wurden HTML Fragemente der besuchten Seiten gefunden. Somit wird diese Yara-Regel nicht weiter betrachtet.

\paragraph*{Yara-Regel ``Suchbegriffe``}
Wie in Tabelle \ref{chart:firefox-volatility-keywords} gezeigt, wurden alle Suchbegriffe ``pfaffenhofen``, ``nanoradar``, ``mooserliesl`` sowie ``mallofamerica`` ausschließlich nach dem Browsing Szenario mit geöffnetem Browser (RAM Dump 2) identifiziert. Die Suchbegriffe wurden größtenteils in den Speicherbereichen von Firefox-Prozessen gefunden. Nur in elf Fällen wurden Suchbegriffe in anderen Prozessen identifiziert. Am häufigsten wurde der Suchbegriff ``pfaffenhofen`` mit 1301 Artefakten gefunden. Dies ist vermutlich auf den  visuellen Google Maps Kartenausschnitt zurückzuführen, welcher bei der Google-Suche relevante Informationen über die gesuchten Stadt zeigt. 

\begin{table}[h!]
	\resizebox{\linewidth}{!}{
	\begin{tabular}{l}	
		\begin{tikzpicture}
			\begin{axis}[
			xbar,
			width=12cm, 
			height=3cm, 
			ylabel style={align=center}, ylabel=\textbf{paffenhofen},
			y=0.75cm,
			symbolic y coords={RAM-Dump 3, RAM-Dump 2, RAM-Dump 1},
			label style={font=\small},
			tick label style={font=\small},
			ytick=data,
			xticklabels={,,},
            xmin = 0,
            xmax = 1400,
			nodes near coords, 
			nodes near coords align={horizontal},
			nodes near coords style={font=\tiny},
   			nodes near coords={\pgfmathfloatifflags{\pgfplotspointmeta}{0}{}{\pgfmathprintnumber{\pgfplotspointmeta}}},
			bar width=.17cm,
			enlarge y limits={abs=2*\pgfplotbarwidth},
			scaled x ticks=false,
			legend style={
				at={(0.5,-0.1)},
				anchor=north
			},
			legend columns=3,
    		yminorgrids = true,minor tick num=1
			]
				\addplot coordinates {
				(0,RAM-Dump 3) (1301,RAM-Dump 2) (0,RAM-Dump 1)
				};
				\addplot coordinates {
				(0,RAM-Dump 3) (0,RAM-Dump 2) (0,RAM-Dump 1)
				};
			\end{axis}
		\end{tikzpicture}
		\\[-7pt]
		\begin{tikzpicture}
			\begin{axis}[
			xbar,
			width=12cm, 
			height=3cm, 
			ylabel style={align=center}, ylabel=\textbf{nanoradar},
			y=0.75cm,
			symbolic y coords={RAM-Dump 3, RAM-Dump 2, RAM-Dump 1},
			label style={font=\small},
			tick label style={font=\small},
			ytick=data,
			xticklabels={,,},
            xmin = 0,
            xmax = 1400,
			nodes near coords, 
			nodes near coords align={horizontal},
			nodes near coords style={font=\tiny},
   			nodes near coords={\pgfmathfloatifflags{\pgfplotspointmeta}{0}{}{\pgfmathprintnumber{\pgfplotspointmeta}}},
			bar width=.17cm,
			enlarge y limits={abs=2*\pgfplotbarwidth},
			scaled x ticks=false,
			legend style={
				at={(0.5,-0.1)},
				anchor=north
			},
			legend columns=3,
    		yminorgrids = true,minor tick num=1
			]
				\addplot coordinates {
				(0,RAM-Dump 3)  (541,RAM-Dump 2) (0,RAM-Dump 1)
				};
				\addplot coordinates {
				(0,RAM-Dump 3)  (11,RAM-Dump 2) (0,RAM-Dump 1)
				};
			\end{axis}
		\end{tikzpicture}
		\\[-7pt]
		\begin{tikzpicture}
			\begin{axis}[
			xbar,
			width=12cm, 
			height=3cm, 
			ylabel style={align=center}, ylabel=\textbf{mooserliesl},
			y=0.75cm,
			symbolic y coords={RAM-Dump 3, RAM-Dump 2, RAM-Dump 1},
			label style={font=\small},
			tick label style={font=\small},
			ytick=data,
			xticklabels={,,},
            xmin = 0,
            xmax = 1400,
			nodes near coords, 
			nodes near coords align={horizontal},
			nodes near coords style={font=\tiny},
   			nodes near coords={\pgfmathfloatifflags{\pgfplotspointmeta}{0}{}{\pgfmathprintnumber{\pgfplotspointmeta}}},
			bar width=.17cm,
			enlarge y limits={abs=2*\pgfplotbarwidth},
			scaled x ticks=false,
			legend style={
				at={(0.5,-0.1)},
				anchor=north
			},
			legend columns=3,
    		yminorgrids = true,minor tick num=1
			]
				\addplot coordinates {
				(0,RAM-Dump 3)  (25,RAM-Dump 2) (0,RAM-Dump 1)
				};
				\addplot coordinates {
				(0,RAM-Dump 3)  (1,RAM-Dump 2) (0,RAM-Dump 1)
				};
			\end{axis}
		\end{tikzpicture}
		\\[-7pt]
		\begin{tikzpicture}
			\begin{axis}[
			xbar,
			width=12cm, 
			height=3cm, 
			ylabel style={align=center}, ylabel=\textbf{mallofamerica},
			y=0.75cm,
			symbolic y coords={RAM-Dump 3, RAM-Dump 2, RAM-Dump 1},
			label style={font=\small},
			tick label style={font=\small},
			ytick=data,
			xticklabels={,,},
            xmin = 0,
            xmax = 1400,
			nodes near coords, 
			nodes near coords align={horizontal},
			nodes near coords style={font=\tiny},
   			nodes near coords={\pgfmathfloatifflags{\pgfplotspointmeta}{0}{}{\pgfmathprintnumber{\pgfplotspointmeta}}},
			bar width=.17cm,
			enlarge y limits={abs=2*\pgfplotbarwidth},
			scaled x ticks=false,
			legend style={
				at={(0.5,-0.1)},
				anchor=north
			},
			legend columns=3,
    		yminorgrids = true,minor tick num=1
			]
				\addplot coordinates {
				(0,RAM-Dump 3)  (104,RAM-Dump 2) (0,RAM-Dump 1)
				};
				\addplot coordinates {
				(0,RAM-Dump 3)  (0,RAM-Dump 2) (0,RAM-Dump 1)
				};
				\legend{firefox.exe, Andere Prozesse}
			\end{axis}
		\end{tikzpicture}
	\end{tabular}
	}
	\caption{Anzahl gefundener Suchbegriffe im Firefox RAM}
	\label{chart:firefox-volatility-keywords}
\end{table}
				

%\begin{figure}[h!]
%	\centerline{\resizebox{\linewidth}{!}{\includegraphics{bilder/volatility/firefox/keywords.png}}}
%	\label{chart:final-criteria}  
%	\caption{Keywords}
%\end{figure}

\paragraph*{Yara-Regel ``URLs``}

\begin{table}[h!]
	\resizebox{\linewidth}{!}{
	\begin{tabular}{l}	
		\begin{tikzpicture}
			\begin{axis}[
			xbar,
			width=12cm, 
			height=3cm, 
			ylabel style={align=center}, ylabel=\textbf{unitree.com},
			y=0.75cm,
			symbolic y coords={RAM-Dump 3, RAM-Dump 2, RAM-Dump 1},
			label style={font=\small},
			tick label style={font=\small},
			ytick=data,
			xticklabels={,,},
            xmin = 0,
            xmax = 4000,
			nodes near coords, 
			nodes near coords align={horizontal},
			nodes near coords style={font=\tiny},
   			nodes near coords={\pgfmathfloatifflags{\pgfplotspointmeta}{0}{}{\pgfmathprintnumber{\pgfplotspointmeta}}},
			bar width=.17cm,
			enlarge y limits={abs=2*\pgfplotbarwidth},
			scaled x ticks=false,
			legend style={
				at={(0.5,-0.1)},
				anchor=north
			},
			legend columns=3,
    		yminorgrids = true,minor tick num=1
			]
				\addplot coordinates {
				(0,RAM-Dump 3) (1900,RAM-Dump 2) (0,RAM-Dump 1)
				};
				\addplot coordinates {
				(32,RAM-Dump 3) (34,RAM-Dump 2) (0,RAM-Dump 1)
				};
			\end{axis}
		\end{tikzpicture}
		\\[-7pt]
		\begin{tikzpicture}
			\begin{axis}[
			xbar,
			width=12cm, 
			height=3cm, 
			ylabel style={align=center}, ylabel=\textbf{mooserliesl.de},
			y=0.75cm,
			symbolic y coords={RAM-Dump 3, RAM-Dump 2, RAM-Dump 1},
			label style={font=\small},
			tick label style={font=\small},
			ytick=data,
			xticklabels={,,},
            xmin = 0,
            xmax = 4000,
			nodes near coords, 
			nodes near coords align={horizontal},
			nodes near coords style={font=\tiny},
   			nodes near coords={\pgfmathfloatifflags{\pgfplotspointmeta}{0}{}{\pgfmathprintnumber{\pgfplotspointmeta}}},
			bar width=.17cm,
			enlarge y limits={abs=2*\pgfplotbarwidth},
			scaled x ticks=false,
			legend style={
				at={(0.5,-0.1)},
				anchor=north
			},
			legend columns=3,
    		yminorgrids = true,minor tick num=1
			]
				\addplot coordinates {
				(0,RAM-Dump 3) (382,RAM-Dump 2) (0,RAM-Dump 1)
				};
				\addplot coordinates {
				(15,RAM-Dump 3) (8,RAM-Dump 2) (0,RAM-Dump 1)
				};
			\end{axis}
		\end{tikzpicture}	
		\\[-7pt]
		\begin{tikzpicture}
			\begin{axis}[
			xbar,
			width=12cm, 
			height=3cm, 
			ylabel style={align=center}, ylabel=\textbf{mallofamerica.com},
			y=0.75cm,
			symbolic y coords={RAM-Dump 3, RAM-Dump 2, RAM-Dump 1},
			label style={font=\small},
			tick label style={font=\small},
			ytick=data,
			xticklabels={,,},
            xmin = 0,
            xmax = 4000,
			nodes near coords, 
			nodes near coords align={horizontal},
			nodes near coords style={font=\tiny},
   			nodes near coords={\pgfmathfloatifflags{\pgfplotspointmeta}{0}{}{\pgfmathprintnumber{\pgfplotspointmeta}}},
			bar width=.17cm,
			enlarge y limits={abs=2*\pgfplotbarwidth},
			scaled x ticks=false,
			legend style={
				at={(0.5,-0.1)},
				anchor=north
			},
			legend columns=3,
    		yminorgrids = true,minor tick num=1
			]
				\addplot coordinates {
				(0,RAM-Dump 3) (2272,RAM-Dump 2) (0,RAM-Dump 1)
				};
				\addplot coordinates {
				(15,RAM-Dump 3) (364,RAM-Dump 2) (0,RAM-Dump 1)
				};
			\end{axis}
		\end{tikzpicture}
		\\[-7pt]
		\begin{tikzpicture}
			\begin{axis}[
			xbar,
			width=12cm, 
			height=3cm, 
			ylabel style={align=center}, ylabel=\textbf{donaukurier.de},
			y=0.75cm,
			symbolic y coords={RAM-Dump 3, RAM-Dump 2, RAM-Dump 1},
			label style={font=\small},
			tick label style={font=\small},
			ytick=data,
			xticklabels={,,},
            xmin = 0,
            xmax = 4000,
			nodes near coords, 
			nodes near coords align={horizontal},
			nodes near coords style={font=\tiny},
   			nodes near coords={\pgfmathfloatifflags{\pgfplotspointmeta}{0}{}{\pgfmathprintnumber{\pgfplotspointmeta}}},
			bar width=.17cm,
			enlarge y limits={abs=2*\pgfplotbarwidth},
			scaled x ticks=false,
			legend style={
				at={(0.5,-0.1)},
				anchor=north
			},
			legend columns=3,
    		yminorgrids = true,minor tick num=1
			]
				\addplot coordinates {
				(0,RAM-Dump 3) (3657,RAM-Dump 2) (0,RAM-Dump 1)
				};
				\addplot coordinates {
				(36,RAM-Dump 3) (38,RAM-Dump 2) (0,RAM-Dump 1)
				};
				\legend{firefox.exe, Andere Prozesse}
			\end{axis}
		\end{tikzpicture}		
	\end{tabular}
	}
	\caption{Anzahl gefundener URLs im Firefox RAM}
	\label{chart:firefox-volatility-urls}
\end{table}

Es konnten in den Arbeitsspeicherabbildern alle besuchten URLs ``unitree.com``, ``mooserliesl.de``, ``mallofamerica.com`` sowie ``donaukurier.de`` identifiziert werden.
Wie in Tabelle \ref{chart:firefox-volatility-urls} gezeigt, wurden die meisten Artefakte nach dem Browsing Szenario mit geöffnetem Browser (RAM Dump 2) gefunden. Die besuchten URLs wurden hauptsächlich in Firefox-Prozessen gefunden. Die URL ``mooserliesl.de`` wurde mit insgesamt 390 Artefakten am wenigsten gefunden, ``donaukurier.de`` mit über 3600 Artefakten am häufigsten.

Bemerkenswert ist, dass URL-Artefakte gefunden wurden, selbst nachdem der Browser geschlossen wurde (RAM Dump 3). Dabei wurde kein URL-Artefakt in einem Firefox Prozess gefunden.
Anhand der PID 2252 wurde festgestellt, dass sich alle URL-Artefakte nach Schließen des Browsers (RAM-Dump 3) in einem \textit{svchost.exe} Prozess mit der gleichen PID befinden. Unter dem \textit{Service Host} Prozess laufen gruppierte Windows-Dienste, um Ressourcen zu sparen und die Systemleistung zu verbessern.
Volatility bietet das Plugin \textit{svcscan} an, mit dem alle laufenden Dienste ausgegeben werden können.
Bei Anwendung auf den dritten RAM Dump wurde jedoch zu keinem Dienst eine PID angegeben, wordurch der Dienst mit den URL Artefakten nicht im RAM identifiziert werden konnte. \cite{Nicholasswhite.05.06.2023}
Stattdessen wurde der dritte Snapshot aufgetaut, um im laufenden Windowsbetrieb den Dienst mithilfe des Process Explorers zu identifizieren.
Wie in Abbildung \ref{chart:svchost-dnscache} gezeigt, wurde anhand der PID $2252$ der Dienst \textit{DNSCache} ermittelt.
\begin{figure}[h!]
	\centerline{\resizebox{\linewidth}{!}{\includegraphics{bilder/firefox-dnscache.png}}}
	\caption{Unter dem SVChost-Prozess PID $2252$ läuft der DNSCache-Dienst.}
	\label{chart:svchost-dnscache}  
\end{figure}
Der DNSCache-Dienst unter Windows ist ein Teil des Betriebssystems, der für die Übersetzung von Domainnamen in IP-Adressen verantwortlich ist. Der DNSCache-Dienst speichert DNS-Anfragen und Antworten temporär, um wiederholte DNS-Anfragen zu beschleunigen. \cite{MicrosoftLearn.05.06.2023}
Nach Löschen des DNSCaches mit dem Kommandozeilenbefehl \texttt{ipconfig /flushdns}, dem Schließen aller Process Monitor Instanzen sowie Beenden des DNSCaches Services wurde erneut ein RAM-Dump durchgeführt. Dort wurden keine URL Artefakte mehr gefunden.

\paragraph*{Yara-Regel ``E-Mail``}

\begin{table}[h!]
	\resizebox{\linewidth}{!}{
	\begin{tabular}{r}	
		\begin{tikzpicture}
			\begin{axis}[
			xbar,
			width=12cm, 
			height=3cm, 
			ylabel style={align=center}, ylabel=\textbf{Betrefftext},
			y=0.75cm,
			symbolic y coords={RAM-Dump 3, RAM-Dump 2, RAM-Dump 1},
			label style={font=\small},
			tick label style={font=\small},
			ytick=data,
			xticklabels={,,},
            xmin = 0,
            xmax = 80,
			nodes near coords, 
			nodes near coords align={horizontal},
			nodes near coords style={font=\tiny},
   			nodes near coords={\pgfmathfloatifflags{\pgfplotspointmeta}{0}{}{\pgfmathprintnumber{\pgfplotspointmeta}}},
			bar width=.17cm,
			enlarge y limits={abs=2*\pgfplotbarwidth},
			scaled x ticks=false,
			legend style={
				at={(0.5,-0.1)},
				anchor=north
			},
			legend columns=3,
    		yminorgrids = true,minor tick num=1
			]
				\addplot coordinates {
				(0,RAM-Dump 3) (22,RAM-Dump 2) (0,RAM-Dump 1)
				};
				\addplot coordinates {
				(0,RAM-Dump 3) (0,RAM-Dump 2) (0,RAM-Dump 1)
				};
%				\legend{firefox.exe, Andere Prozesse}
			\end{axis}
		\end{tikzpicture}
		\\[-7pt]
		\begin{tikzpicture}
			\begin{axis}[
			xbar,
			width=12cm, 
			height=3cm, 
			ylabel style={align=center}, ylabel=\textbf{Mailtext},
			y=0.75cm,
			symbolic y coords={RAM-Dump 3, RAM-Dump 2, RAM-Dump 1},
			label style={font=\small},
			tick label style={font=\small},
			ytick=data,
			xticklabels={,,},
            xmin = 0,
            xmax = 80,
			nodes near coords, 
			nodes near coords align={horizontal},
			nodes near coords style={font=\tiny},
   			nodes near coords={\pgfmathfloatifflags{\pgfplotspointmeta}{0}{}{\pgfmathprintnumber{\pgfplotspointmeta}}},
			bar width=.17cm,
			enlarge y limits={abs=2*\pgfplotbarwidth},
			scaled x ticks=false,
			legend style={
				at={(0.5,-0.1)},
				anchor=north
			},
			legend columns=3,
    		yminorgrids = true,minor tick num=1
			]
				\addplot coordinates {
				(0,RAM-Dump 3) (24,RAM-Dump 2) (0,RAM-Dump 1)
				};
				\addplot coordinates {
				(0,RAM-Dump 3) (0,RAM-Dump 2) (0,RAM-Dump 1)
				};
%				\legend{firefox.exe, Andere Prozesse}
			\end{axis}
		\end{tikzpicture}	
		\\[-7pt]
		\begin{tikzpicture}
			\begin{axis}[
			xbar,
			width=12cm, 
			height=3cm, 
			ylabel style={align=center}, ylabel=\textbf{Vorlesung23!}\\\textbf{(Passwort)},
			y=0.75cm,
			symbolic y coords={RAM-Dump 3, RAM-Dump 2, RAM-Dump 1},
			label style={font=\small},
			tick label style={font=\small},
			ytick=data,
			xticklabels={,,},
            xmin = 0,
            xmax = 80,
			nodes near coords, 
			nodes near coords align={horizontal},
			nodes near coords style={font=\tiny},
   			nodes near coords={\pgfmathfloatifflags{\pgfplotspointmeta}{0}{}{\pgfmathprintnumber{\pgfplotspointmeta}}},
			bar width=.17cm,
			enlarge y limits={abs=2*\pgfplotbarwidth},
			scaled x ticks=false,
			legend style={
				at={(0.5,-0.1)},
				anchor=north
			},
			legend columns=3,
    		yminorgrids = true,minor tick num=1
			]
				\addplot coordinates {
				(0,RAM-Dump 3) (4,RAM-Dump 2) (0,RAM-Dump 1)
				};
				\addplot coordinates {
				(0,RAM-Dump 3) (0,RAM-Dump 2) (0,RAM-Dump 1)
				};
%				\legend{firefox.exe, Andere Prozesse}
			\end{axis}
		\end{tikzpicture}
		\\[-7pt]
		\begin{tikzpicture}
			\begin{axis}[
			xbar,
			width=12cm, 
			height=3cm, 
			ylabel style={align=center}, ylabel=\textbf{computerforensik}\\\textbf{@gmail.com},
			y=0.75cm,
			symbolic y coords={RAM-Dump 3, RAM-Dump 2, RAM-Dump 1},
			label style={font=\small},
			tick label style={font=\small},
			ytick=data,
			xticklabels={,,},
            xmin = 0,
            xmax = 80,
			nodes near coords, 
			nodes near coords align={horizontal},
			nodes near coords style={font=\tiny},
   			nodes near coords={\pgfmathfloatifflags{\pgfplotspointmeta}{0}{}{\pgfmathprintnumber{\pgfplotspointmeta}}},
			bar width=.17cm,
			enlarge y limits={abs=2*\pgfplotbarwidth},
			scaled x ticks=false,
			legend style={
				at={(0.5,-0.1)},
				anchor=north
			},
			legend columns=3,
    		yminorgrids = true,minor tick num=1
			]
				\addplot coordinates {
				(0,RAM-Dump 3) (66,RAM-Dump 2) (0,RAM-Dump 1)
				};
				\addplot coordinates {
				(0,RAM-Dump 3) (6,RAM-Dump 2) (0,RAM-Dump 1)
				};
%				\legend{firefox.exe, Andere Prozesse}
			\end{axis}
		\end{tikzpicture}	
		\\[-7pt]
		\begin{tikzpicture}
			\begin{axis}[
			xbar,
			width=12cm, 
			height=3cm, 
			ylabel style={align=center}, ylabel=\textbf{chs3702@thi.de},
			y=0.75cm,
			symbolic y coords={RAM-Dump 3, RAM-Dump 2, RAM-Dump 1},
			label style={font=\small},
			tick label style={font=\small},
			ytick=data,
			xticklabels={,,},
            xmin = 0,
            xmax = 80,
			nodes near coords, 
			nodes near coords align={horizontal},
			nodes near coords style={font=\tiny},
   			nodes near coords={\pgfmathfloatifflags{\pgfplotspointmeta}{0}{}{\pgfmathprintnumber{\pgfplotspointmeta}}},
			bar width=.17cm,
			enlarge y limits={abs=2*\pgfplotbarwidth},
			scaled x ticks=false,
			legend style={
				at={(0.5,-0.1)},
				anchor=north
			},
			legend columns=3,
    		yminorgrids = true,minor tick num=1
			]
				\addplot coordinates {
				(0,RAM-Dump 3) (34,RAM-Dump 2) (0,RAM-Dump 1)
				};
%				\legend{firefox.exe, Andere Prozesse}
			\end{axis}
		\end{tikzpicture}
		\\[-7pt]
		\begin{tikzpicture}
			\begin{axis}[
			xbar,
			width=12cm, 
			height=3cm, 
			ylabel style={align=center}, ylabel=\textbf{cas0597@thi.de},
			y=0.75cm,
			symbolic y coords={RAM-Dump 3, RAM-Dump 2, RAM-Dump 1},
			label style={font=\small},
			tick label style={font=\small},
			ytick=data,
			xticklabels={,,},
            xmin = 0,
            xmax = 80,
			nodes near coords, 
			nodes near coords align={horizontal},
			nodes near coords style={font=\tiny},
   			nodes near coords={\pgfmathfloatifflags{\pgfplotspointmeta}{0}{}{\pgfmathprintnumber{\pgfplotspointmeta}}},
			bar width=.17cm,
			enlarge y limits={abs=2*\pgfplotbarwidth},
			scaled x ticks=false,
			legend style={
				at={(0.5,-0.1)},
				anchor=north
			},
			legend columns=3,
    		yminorgrids = true,minor tick num=1
			]
				\addplot coordinates {
				(0,RAM-Dump 3) (28,RAM-Dump 2) (0,RAM-Dump 1)
				};
				\addplot coordinates {
				(0,RAM-Dump 3) (0,RAM-Dump 2) (0,RAM-Dump 1)
				};
				\legend{firefox.exe, Andere Prozesse}
			\end{axis}
		\end{tikzpicture}
				%	\begin{axis}[]
		%	\legend{Logfile 1, Logfile 2}
		%	\end{axis}

	\end{tabular}
	}
	\caption{Anzahl gefundener E-Mail Artefakte im Firefox RAM}
	\label{chart:firefox-volatility-mail}
\end{table}

%\begin{figure}[h!]
%	\centerline{\resizebox{\linewidth}{!}{\includegraphics{bilder/volatility/firefox/mail.png}}}
%	\label{chart:final-criteria}  
%	\caption{Mail}
%\end{figure}
Wie in Abbildung \ref{chart:firefox-volatility-mail} gezeigt, wurden ausschließlich nach dem Browsing Szenario mit geöffnetem Firefox Browser (RAM Dump 2) alle E-Mail Artefakte gefunden.
Unter den gefundenen Artefakten befindet sich am häufigsten die Absenderadresse ``computerforensikvl@gmail.com``. Dieses Artefakt wurde als einziges E-Mail-Artefakt sechsmal in anderen Prozessen als Firefox gefunden.

Bemerkenswert ist, dass das Passwort des Google-Accounts, mit dem die E-Mails verschickt wurden, viermal als Klartext im RAM gefunden wurden. Das Passwort wurde je zweimal in zwei Firefox Prozessen mit den PIDs 7420 und 8424 gefunden. Tabelle \ref{tab:firefox-mapping-virtaddr-to-byteoffset} zeigt die virtuellen Speicheradressen der Artefakte aus der Yarascan Ausgabe.
\begin{table}[h!]
\resizebox{\linewidth}{!}{
\begin{tabular}{|c|c|c|ll}
\cline{1-3}
\textbf{Virtuelle Speicheradresse} & \textbf{PID} & \textbf{Byte-Offset in extrahierter Speicherseite} &  &  \\ \cline{1-3}
0xb9ce29180c8                      & 7420         & 0x11dd40c8                                         &  &  \\ \cline{1-3}
0x2859f4ffd4e0                     & 7420         & 0x12e234e0                                         &  &  \\ \cline{1-3}
0x24083b41858                      & 8424         & 0x583858                                           &  &  \\ \cline{1-3}
0x240840e5b08                      & 8424         & 0x96bb08                                           &  &  \\ \cline{1-3}
\end{tabular}
}
\caption{Abbildung der virtellen Speicheradressen im Firefox-RAM der gefundenen Strings auf Byte-Offsets der entsprechenden Speicherseiten}
\label{tab:firefox-mapping-virtaddr-to-byteoffset}
\end{table}

Zu diesen Artefakten wurde gemäß Methodik in Kapitel \ref{subsubsection:ergebnisse-firefox-uncommonlocations-analysemitvolatility} der String Kontext -- also die Zeichen vor und nach dem gefundenen Artefakt im Speicherbereich -- ermittelt. Dazu wurde mithilfe des Volatility memmap Plugins die Abbildung der virtuellen Speicheradressen auf den Byte-Offset in der extrahierten Speicherseite des Prozesses ermittelt. 

\begin{figure}[h!]
	\centering
	\subcaptionbox{Byte-Offset 0x11dd40c8}{\includegraphics[width=0.47\textwidth]{bilder/volatility/firefox/password_0xb9ce29180c8_7420.png}}%
	\hfill
	\subcaptionbox{Byte-Offset 0x12e234e0}{\includegraphics[width=0.47\textwidth]{bilder/volatility/firefox/password_0x2859f4ffd4e0_7420.png}}%
	\caption{Passwort-Klartext in Firefox Speicherseiten von PID 7420}
	\label{img:firefox-pw-offset-pid-7420}  
\end{figure}
Wie in Abbildung \ref{img:firefox-pw-offset-pid-7420} gezeigt, sind in der Speicherseite des Prozesses mit PID 7420 in unmittelbarer Umgebung des gefundenen Passworts am Byte-Offset 0xb9ce29180c8 Code-Fragmente der \textit{Gecko-Engine} zu finden. Dieser Teil des Firefox Browsers ist für das Rendering von Webinhalten verantwortlich, einschließlich HTML, CSS, JavaScript und anderen Medienformaten wie Bildern, Audio und Video. \cite{MozillaWiki.05.06.2023}
In der gleichen Datei konnten nach dem gefundenen Passwort am Byte-Offset $0x12e234e0$ die Strings ``Passwd`` sowie ``sessionrestore`` (siehe Common Location \textit{Sessionstore} in Anhang \ref{subsubsection:appendix-firefox-common-locations-writefile-operations}) identifiziert werden. 

\begin{figure}[h!]
	\centering
	\subcaptionbox{Byte-Offset 0x583858}{\includegraphics[width=0.47\textwidth]{bilder/volatility/firefox/password_0x24083b41858_8424.png}}%
	\hfill
	\subcaptionbox{Byte-Offset 0x96bb08}{\includegraphics[width=0.47\textwidth]{bilder/volatility/firefox/password_0x240840e5b08_8424.png}}%
	\caption{Passwort-Klartext in Firefox Speicherseiten von PID 8424}
	\label{img:firefox-pw-offset-pid-8424}  
\end{figure}
Wie in Abbildung \ref{img:firefox-pw-offset-pid-8424} gezeigt, kann in den Byte-Offsets der gefundenen Passwörter in der Speicherseite der PID 8424 kein sinnvoller Kontext ermittelt werden. Im Gegensatz zur Speicherseite der PID 7420 wird das Passwort dort mit 2 Bytes pro Zeichen enkodiert, was eine Unicode-Zeichenenkodierung vermuten lässt.

\paragraph*{Yara-Regel ``DK-Logo``}
Wie in Abbildung \ref{chart:firefox-volatility-image} gezeigt, wurde das im Browsing Szenario geöffnete Donaukurier Logo ausschließlich im zweiten RAM Dump dreimal in Firefox Prozessen gefunden.
\begin{table}[h!]
	\resizebox{\linewidth}{!}{
	\begin{tabular}{r}
		\begin{tikzpicture}
			\begin{axis}[
			xbar stacked,
			width=18cm, 
			height=12cm, 
			ylabel style={align=center}, ylabel=Donaukurier Logo\\(Hexadezimal),
			y=1cm,
			symbolic y coords={RAM-Dump 3, RAM-Dump 2, RAM-Dump 1},
			ytick=data,
			xticklabels={,,},
            xmin = 0,
            xmax = 4,
			nodes near coords, 
			nodes near coords align={horizontal},
			legend style={
				at={(0.5,-0.1)},
				anchor=north
			},
			legend columns=2
			]
				\addplot coordinates {
				(0,RAM-Dump 3) (3,RAM-Dump 2) (0,RAM-Dump 1)
				};
				\addplot coordinates {
				(0,RAM-Dump 3) (0,RAM-Dump 2) (0,RAM-Dump 1)
				};
				\legend{firefox.exe, Andere Prozesse}
			\end{axis}
		\end{tikzpicture}
	\end{tabular}
	}
	\caption{Anzahl gefundener Hexadezimalwerte des Donaukurier-Logos im Firefox RAM}
	\label{chart:firefox-volatility-image}
\end{table}
%\begin{figure}[h!]
%	\centerline{\resizebox{\linewidth}{!}{\includegraphics{bilder/volatility/firefox/image.png}}}
%	\label{chart:final-criteria}  
%	\caption{Image}
%\end{figure}



%\begin{figure}[h!]
%	\centerline{\resizebox{\linewidth}{!}{\includegraphics{bilder/volatility/firefox/summary.png}}}
%	\label{chart:final-criteria}  
%	\caption{Summary}
%\end{figure}

\subsection*{Registry}
\label{subsection:ergebnisse-firefox-registry}
Die Analyse der Registry zählt gemäß Methodik in Kapitel \ref{subsection:methodik-datenanalyse-registry} sowohl zu den Common als auch Uncommon Locations. Weder in den Process Monitor ``SetValue`` Operations noch in den System- und User-Hives konnten PB Artefakte gefunden werden. Eine detaillierte Analyse dieser Common- und Uncommon Locations der Registry ist im Anhang \ref{subsection:appendix-firefox-registry} beschrieben.


%*** TODO: Zusammenfassung Firefox ***


\newpage


% ######################################################################
% ######################################################################
% ######################################################################
% ######################################################################


\section{Tor}

In diesem Abschnitt werden die Ergebnisse der Datenanalyse der Common Locations, Uncommon Locations sowie der Registry für den Tor-Browser präsentiert.

\subsection*{Common Locations}
Zunächst werden die Common Locations analysiert, um potenzielle Hinweise auf Internet-Aktivitäten des Browsing Szenarios zu finden. Bei der Untersuchung der gängigen Speicherorte wurde gemäß der im Kapitel \ref{subsection:methodik-datenanalyse-commonlocations} beschriebenen Methodik zwischen Schreibvorgängen in den Logdateien des Process Monitors und den SQLite-Datenbanken zur Verwaltung von Benutzerdaten unterschieden. Dabei konnten in keiner Datei PB Artefakte gefunden werden. Eine detaillierte Analyse der Process Monitor ``WriteFile`` Operations sowie der SQLite-Datenbänke ist im Anhang \ref{subsection:appendix-tor-common-locations} beschrieben.

\subsection*{Uncommon Locations}
Nachfolgend werden die Analyseergebnisse der Tor Uncommon Locations beschrieben.
Dazu werden die vollständigen Speicherabbilder nach PB Artefakten untersucht, ohne das genaue Browserverhalten zu berücksichtigen. Stattdessen wird sich auf die Vollständigkeit der Funktionen der Forensik-Tools Autopsy und Volatility verlassen.

\subsubsection*{Analyse mit Autopsy}
Autopsy wird bei den Uncommon Locations als konkretes forensisches Werkzeug verwendet, statt nur zur Dateiextraktion, wie es bei den Common Locations der Fall war.

Eine Stichwortsuche nach PB Artefakten in Autopsy in allen fünf Tor Festplatten-Images ergab keine Treffer.

Ebenso wurden in den automatisch von Autopsy kategorisierten Dateien keine PB Artefakte gefunden. 
Im Anhang \ref{subsubsection:appendix-tor-uncommon-locations-autopsy} ist eine detaillierte Analyse der kategorisierten Dateien beschrieben.


\subsubsection*{Analyse mit Volatility}
Bei der Untersuchung der Tor RAM-Dumps mithilfe des Volatility Plugins Yarascan, konnten PB-Artefakte identifiziert werden.
Nachfolgend werden die Ergebnisse der geordnet nach Yara-Regeln beschrieben. Die vollständigen Yara-Regeln sind im Anhang \ref{appendix:yara-regeln} aufgelistet.

\paragraph*{Yara-Regel ``HTML``}
Wie bei Firefox, konnten keine HTML Artefakte im RAM gefunden werden. Deshalb wird diese Kategorie nicht weiter aufgeführt.

\paragraph*{Yara-Regel ``Suchbegriffe``}
Wie in Abbildung \ref{chart:tor-volatility-keywords} gezeigt, wurden nach dem Browsing-Szenarios sowie vor (RAM-Dump 2) als auch nach Erstellen einer ``Neuen Identität`` (RAM Dump 3-1) Suchbegriffe des Browsing-Szenarios gefunden.
Nachdem eine ``Neue Identität`` erstellt wurde, reduzierten sich die gefundenen Artefakte deutlich. 
Die Suchbegriffe wurden hauptsächlich in Firefox Prozessen gefunden. Kein Artefakt war im Tor Prozess zu finden.
Mit 4833 Artefakten wurde am häufigsten der Suchbegriff ``pfaffenhofen`` nach dem Browsing Szenario im zweiten RAM-Dump gefunden. 
Nach dem Schließen des Tor-Browsers wurden keine Suchbegriffe im RAM identifiziert.

\begin{table}[h!]
	\resizebox{\linewidth}{!}{
	\begin{tabular}{l}	
		\begin{tikzpicture}
			\begin{axis}[
			xbar,
			width=12cm, 
			height=3cm, 
			ylabel style={align=center}, ylabel=\textbf{paffenhofen},
			y=0.75cm,
			symbolic y coords={RAM-Dump 3-2, RAM-Dump 3-1, RAM-Dump 2, RAM-Dump 1},
			label style={font=\small},
			tick label style={font=\small},
			ytick=data,
			xticklabels={,,},
            xmin = 0,
            xmax = 5500,
			nodes near coords, 
			nodes near coords align={horizontal},
			nodes near coords style={font=\tiny},
   			nodes near coords={\pgfmathfloatifflags{\pgfplotspointmeta}{0}{}{\pgfmathprintnumber{\pgfplotspointmeta}}},
			bar width=.17cm,
			enlarge y limits={abs=2*\pgfplotbarwidth},
			scaled x ticks=false,
			legend style={
				at={(0.5,-0.1)},
				anchor=north
			},
			legend columns=3,
    		yminorgrids = true,minor tick num=1
			]
				\addplot coordinates {
				(0,RAM-Dump 3-2)  (780,RAM-Dump 3-1) (4833,RAM-Dump 2) (0,RAM-Dump 1)
				};
%				\addplot coordinates {
%				(0,RAM-Dump 3-2)  (0,RAM-Dump 3-1) (0,RAM-Dump 2) (0,RAM-Dump 1)
%				};
				\addplot coordinates {
				(0,RAM-Dump 3-2)  (9,RAM-Dump 3-1) (23,RAM-Dump 2) (0,RAM-Dump 1)
				};
%				\legend{firefox.exe, tor.exe, Andere Prozesse}
			\end{axis}
		\end{tikzpicture}
		\\[-7pt]
		\begin{tikzpicture}
			\begin{axis}[
			xbar,
			width=12cm, 
			height=3cm, 
			ylabel style={align=center}, ylabel=\textbf{nanoradar},
			y=0.75cm,
			symbolic y coords={RAM-Dump 3-2, RAM-Dump 3-1, RAM-Dump 2, RAM-Dump 1},
			label style={font=\small},
			tick label style={font=\small},
			ytick=data,
			xticklabels={,,},
            xmin = 0,
            xmax = 5500,
			nodes near coords, 
			nodes near coords align={horizontal},
			nodes near coords style={font=\tiny},
   			nodes near coords={\pgfmathfloatifflags{\pgfplotspointmeta}{0}{}{\pgfmathprintnumber{\pgfplotspointmeta}}},
			bar width=.17cm,
			enlarge y limits={abs=2*\pgfplotbarwidth},
			scaled x ticks=false,
			legend style={
				at={(0.5,-0.1)},
				anchor=north
			},
			legend columns=3,
    		yminorgrids = true,minor tick num=1
			]
				\addplot coordinates {
				(0,RAM-Dump 3-2)  (9,RAM-Dump 3-1) (439,RAM-Dump 2) (0,RAM-Dump 1)
				};
%				\addplot coordinates {
%				(0,RAM-Dump 3-2)  (0,RAM-Dump 3-1) (0,RAM-Dump 2) (0,RAM-Dump 1)
%				};
				\addplot coordinates {
				(0,RAM-Dump 3-2)  (1,RAM-Dump 3-1) (3,RAM-Dump 2) (0,RAM-Dump 1)
				};
%				\legend{firefox.exe, tor.exe, Andere Prozesse}
			\end{axis}
		\end{tikzpicture}
		\\[-7pt]
		\begin{tikzpicture}
			\begin{axis}[
			xbar,
			width=12cm, 
			height=3cm, 
			ylabel style={align=center}, ylabel=\textbf{mooserliesl},
			y=0.75cm,
			symbolic y coords={RAM-Dump 3-2, RAM-Dump 3-1, RAM-Dump 2, RAM-Dump 1},
			label style={font=\small},
			tick label style={font=\small},
			ytick=data,
			xticklabels={,,},
            xmin = 0,
            xmax = 5500,
			nodes near coords, 
			nodes near coords align={horizontal},
			nodes near coords style={font=\tiny},
   			nodes near coords={\pgfmathfloatifflags{\pgfplotspointmeta}{0}{}{\pgfmathprintnumber{\pgfplotspointmeta}}},
			bar width=.17cm,
			enlarge y limits={abs=2*\pgfplotbarwidth},
			scaled x ticks=false,
			legend style={
				at={(0.5,-0.1)},
				anchor=north
			},
			legend columns=3,
    		yminorgrids = true,minor tick num=1
			]
				\addplot coordinates {
				(0,RAM-Dump 3-2)  (4,RAM-Dump 3-1) (11,RAM-Dump 2) (0,RAM-Dump 1)
				};
%				\addplot coordinates {
%				(0,RAM-Dump 3-2)  (0,RAM-Dump 3-1) (0,RAM-Dump 2) (0,RAM-Dump 1)
%				};
				\addplot coordinates {
				(0,RAM-Dump 3-2)  (0,RAM-Dump 3-1) (0,RAM-Dump 2) (0,RAM-Dump 1)
				};
%				\legend{firefox.exe, tor.exe, Andere Prozesse}
			\end{axis}
		\end{tikzpicture}
		\\[-7pt]
		\begin{tikzpicture}
			\begin{axis}[
			xbar,
			width=12cm, 
			height=3cm, 
			ylabel style={align=center}, ylabel=\textbf{mallofamerica},
			y=0.75cm,
			symbolic y coords={RAM-Dump 3-2, RAM-Dump 3-1, RAM-Dump 2, RAM-Dump 1},
			label style={font=\small},
			tick label style={font=\small},
			ytick=data,
			xticklabels={,,},
            xmin = 0,
            xmax = 5500,
			nodes near coords, 
			nodes near coords align={horizontal},
			nodes near coords style={font=\tiny},
   			nodes near coords={\pgfmathfloatifflags{\pgfplotspointmeta}{0}{}{\pgfmathprintnumber{\pgfplotspointmeta}}},
			bar width=.17cm,
			enlarge y limits={abs=2*\pgfplotbarwidth},
			scaled x ticks=false,
			legend style={
				at={(0.5,-0.1)},
				anchor=north
			},
			legend columns=3,
    		yminorgrids = true,minor tick num=1
			]
				\addplot coordinates {
				(0,RAM-Dump 3-2)  (1056,RAM-Dump 3-1) (1851,RAM-Dump 2) (0,RAM-Dump 1)
				};
%				\addplot coordinates {
%				(0,RAM-Dump 3-2)  (0,RAM-Dump 3-1) (0,RAM-Dump 2) (0,RAM-Dump 1)
%				};
				\addplot coordinates {
				(0,RAM-Dump 3-2)  (11,RAM-Dump 3-1) (3,RAM-Dump 2) (0,RAM-Dump 1)
				};
				\legend{firefox.exe, Andere Prozesse}
			\end{axis}
		\end{tikzpicture}
	\end{tabular}
	}
	\caption{Gefundene Suchbegriffe im Tor RAM}
	\label{chart:tor-volatility-keywords}
\end{table}
%\begin{figure}[h!]
%	\centerline{\resizebox{\linewidth}{!}{\includegraphics{bilder/volatility/tor/keywords.png}}}
%	\label{chart:final-criteria}  
%	\caption{Keywords}
%\end{figure}

\paragraph*{Yara-Regel ``URLs``}
Ähnlich zur Yara-Regel ``Suchbegriffe`` wurden, wie in Abbildung \ref{chart:tor-volatility-urls} gezeigt, ausschließlich nach dem Browsing-Szenario vor (RAM-Dump 2) und nach Zuweisung einer neuen Identität (RAM-Dump 3-1) URL Artefakte gefunden. Dabei wurden im RAM Dump 3-1 deutlich weniger URL-Artefakte als in RAM Dump 2 gefunden. 
Artefakte dieser Yara-Regel wurden nach Firefox-Prozessen hauptsächlich in Tor-Prozessen gefunden. Am wenigsten Artefakte wurden in anderen Prozessen identifiziert.
Auffällig ist, dass die URL ``mallofamerica.com`` 26505 Mal in RAM-Dump 2 gefunden wurde. Im Gegensatz dazu wurde ``mooserliesl.de`` nur 518 Mal gefunden. Nach Schließen des Tor-Browsers wurden keine URL Artefakte mehr im RAM gefunden.
\begin{table}[h!]
	\resizebox{\linewidth}{!}{
	\begin{tabular}{l}	
		\begin{tikzpicture}
			\begin{axis}[
			xbar,
			width=12cm, 
			height=3cm, 
			ylabel style={align=center}, ylabel=\textbf{unitree.com},
			y=0.75cm,
			symbolic y coords={RAM-Dump 3-2, RAM-Dump 3-1, RAM-Dump 2, RAM-Dump 1},
			label style={font=\small},
			tick label style={font=\small},
			ytick=data,
			xticklabels={,,},
            xmin = 0,
            xmax = 30000,
			nodes near coords, 
			nodes near coords align={horizontal},
			nodes near coords style={font=\tiny},
   			nodes near coords={\pgfmathfloatifflags{\pgfplotspointmeta}{0}{}{\pgfmathprintnumber{\pgfplotspointmeta}}},
			bar width=.17cm,
			enlarge y limits={abs=2*\pgfplotbarwidth},
			scaled x ticks=false,
			legend style={
				at={(0.5,-0.1)},
				anchor=north
			},
			legend columns=3,
    		yminorgrids = true,minor tick num=1
			]
				\addplot coordinates {
				(0,RAM-Dump 3-2)  (208,RAM-Dump 3-1) (1549,RAM-Dump 2) (0,RAM-Dump 1)
				};
				\addplot coordinates {
				(0,RAM-Dump 3-2)  (7,RAM-Dump 3-1) (7,RAM-Dump 2) (0,RAM-Dump 1)
				};
				\addplot coordinates {
				(0,RAM-Dump 3-2)  (0,RAM-Dump 3-1) (0,RAM-Dump 2) (0,RAM-Dump 1)
				};
%				\legend{firefox.exe, tor.exe, Andere Prozesse}
			\end{axis}
		\end{tikzpicture}
		\\[-7pt]
		\begin{tikzpicture}
			\begin{axis}[
			xbar,
			width=12cm, 
			height=3cm, 
			ylabel style={align=center}, ylabel=\textbf{mooserliesl.de},
			y=0.75cm,
			symbolic y coords={RAM-Dump 3-2, RAM-Dump 3-1, RAM-Dump 2, RAM-Dump 1},
			label style={font=\small},
			tick label style={font=\small},
			ytick=data,
			xticklabels={,,},
            xmin = 0,
            xmax = 30000,
			nodes near coords, 
			nodes near coords align={horizontal},
			nodes near coords style={font=\tiny},
   			nodes near coords={\pgfmathfloatifflags{\pgfplotspointmeta}{0}{}{\pgfmathprintnumber{\pgfplotspointmeta}}},
			bar width=.17cm,
			enlarge y limits={abs=2*\pgfplotbarwidth},
			scaled x ticks=false,
			legend style={
				at={(0.5,-0.1)},
				anchor=north
			},
			legend columns=3,
    		yminorgrids = true,minor tick num=1
			]
				\addplot coordinates {
				(0,RAM-Dump 3-2)  (29,RAM-Dump 3-1) (508,RAM-Dump 2) (0,RAM-Dump 1)
				};
				\addplot coordinates {
				(0,RAM-Dump 3-2)  (6,RAM-Dump 3-1) (6,RAM-Dump 2) (0,RAM-Dump 1)
				};
				\addplot coordinates {
				(0,RAM-Dump 3-2)  (0,RAM-Dump 3-1) (4,RAM-Dump 2) (0,RAM-Dump 1)
				};
%				\legend{firefox.exe, tor.exe, Andere Prozesse}
			\end{axis}
		\end{tikzpicture}
		\\[-7pt]
		\begin{tikzpicture}
			\begin{axis}[
			xbar,
			width=12cm, 
			height=3cm, 
			ylabel style={align=center}, ylabel=\textbf{mallofamerica.com},
			y=0.75cm,
			symbolic y coords={RAM-Dump 3-2, RAM-Dump 3-1, RAM-Dump 2, RAM-Dump 1},
			label style={font=\small},
			tick label style={font=\small},
			ytick=data,
			xticklabels={,,},
            xmin = 0,
            xmax = 30000,
			nodes near coords, 
			nodes near coords align={horizontal},
			nodes near coords style={font=\tiny},
   			nodes near coords={\pgfmathfloatifflags{\pgfplotspointmeta}{0}{}{\pgfmathprintnumber{\pgfplotspointmeta}}},
			bar width=.17cm,
			enlarge y limits={abs=2*\pgfplotbarwidth},
			scaled x ticks=false,
			legend style={
				at={(0.5,-0.1)},
				anchor=north
			},
			legend columns=3,
    		yminorgrids = true,minor tick num=1
			]
				\addplot coordinates {
				(0,RAM-Dump 3-2)  (3371,RAM-Dump 3-1) (26505,RAM-Dump 2) (0,RAM-Dump 1)
				};
				\addplot coordinates {
				(0,RAM-Dump 3-2)  (90,RAM-Dump 3-1) (136,RAM-Dump 2) (0,RAM-Dump 1)
				};
				\addplot coordinates {
				(0,RAM-Dump 3-2)  (3,RAM-Dump 3-1) (0,RAM-Dump 2) (0,RAM-Dump 1)
				};
%				\legend{firefox.exe, tor.exe, Andere Prozesse}
			\end{axis}
		\end{tikzpicture}
		\\[-7pt]
		\begin{tikzpicture}
			\begin{axis}[
			xbar,
			width=12cm, 
			height=3cm, 
			ylabel style={align=center}, ylabel=\textbf{donaukurier.de},
			y=0.75cm,
			symbolic y coords={RAM-Dump 3-2, RAM-Dump 3-1, RAM-Dump 2, RAM-Dump 1},
			label style={font=\small},
			tick label style={font=\small},
			ytick=data,
			xticklabels={,,},
            xmin = 0,
            xmax = 30000,
			nodes near coords, 
			nodes near coords align={horizontal},
			nodes near coords style={font=\tiny},
   			nodes near coords={\pgfmathfloatifflags{\pgfplotspointmeta}{0}{}{\pgfmathprintnumber{\pgfplotspointmeta}}},
			bar width=.17cm,
			enlarge y limits={abs=2*\pgfplotbarwidth},
			scaled x ticks=false,
			legend style={
				at={(0.5,-0.1)},
				anchor=north
			},
			legend columns=3,
    		yminorgrids = true,minor tick num=1
			]
				\addplot coordinates {
				(0,RAM-Dump 3-2)  (1194,RAM-Dump 3-1) (7451,RAM-Dump 2) (0,RAM-Dump 1)
				};
				\addplot coordinates {
				(0,RAM-Dump 3-2)  (2,RAM-Dump 3-1) (1239,RAM-Dump 2) (0,RAM-Dump 1)
				};
				\addplot coordinates {
				(0,RAM-Dump 3-2)  (0,RAM-Dump 3-1) (9,RAM-Dump 2) (0,RAM-Dump 1)
				};
				\legend{firefox.exe, tor.exe, Andere Prozesse}
			\end{axis}
		\end{tikzpicture}
	\end{tabular}
	}
	\caption{Gefundene URLs im Tor RAM}
	\label{chart:tor-volatility-urls}
\end{table}
%\begin{figure}[h!]
%	\centerline{\resizebox{\linewidth}{!}{\includegraphics{bilder/volatility/tor/url.png}}}
%	\label{chart:final-criteria}  
%	\caption{URL}
%\end{figure}

\paragraph*{Yara-Regel ``E-Mail``}
Nach dem Browsing-Szenario, vor Zuweisung einer ``Neuen Identität`` (RAM-Dump 2), konnten alle E-Mail Artefakte gefunden werden.
\begin{table}[h!]
	\resizebox{\linewidth}{!}{
	\begin{tabular}{r}	
		\begin{tikzpicture}					
			\begin{axis}[
			xbar,
			width=12cm, 
			height=3cm, 
			ylabel style={align=center}, ylabel=\textbf{Betrefftext},
			y=0.75cm,
			symbolic y coords={RAM-Dump 3-2, RAM-Dump 3-1, RAM-Dump 2, RAM-Dump 1},
			label style={font=\small},
			tick label style={font=\small},
			ytick=data,
			xticklabels={,,},
            xmin = 0,
            xmax = 110,
			nodes near coords, 
			nodes near coords align={horizontal},
			nodes near coords style={font=\tiny},
   			nodes near coords={\pgfmathfloatifflags{\pgfplotspointmeta}{0}{}{\pgfmathprintnumber{\pgfplotspointmeta}}},
			bar width=.17cm,
			enlarge y limits={abs=2*\pgfplotbarwidth},
			scaled x ticks=false,
			legend style={
				at={(0.5,-0.1)},
				anchor=north
			},
			legend columns=3,
    		yminorgrids = true,minor tick num=1
			]
				\addplot coordinates {
				(0,RAM-Dump 3-2)  (0,RAM-Dump 3-1) (8,RAM-Dump 2) (0,RAM-Dump 1)
				};
%				\addplot coordinates {
%				(0,RAM-Dump 3-2)  (0,RAM-Dump 3-1) (0,RAM-Dump 2) (0,RAM-Dump 1)
%				};
%				\addplot coordinates {
%				(0,RAM-Dump 3-2)  (0,RAM-Dump 3-1) (0,RAM-Dump 2) (0,RAM-Dump 1)
%				};
%				\legend{firefox.exe, tor.exe, Andere Prozesse}
			\end{axis}
		\end{tikzpicture}
		\\[-7pt]
		\begin{tikzpicture}
			\begin{axis}[
			xbar,
			width=12cm, 
			height=3cm, 
			ylabel style={align=center}, ylabel=\textbf{Mailinhalt},
			y=0.75cm,
			symbolic y coords={RAM-Dump 3-2, RAM-Dump 3-1, RAM-Dump 2, RAM-Dump 1},
			label style={font=\small},
			tick label style={font=\small},
			ytick=data,
			xticklabels={,,},
            xmin = 0,
            xmax = 110,
			nodes near coords, 
			nodes near coords align={horizontal},
			nodes near coords style={font=\tiny},
   			nodes near coords={\pgfmathfloatifflags{\pgfplotspointmeta}{0}{}{\pgfmathprintnumber{\pgfplotspointmeta}}},
			bar width=.17cm,
			enlarge y limits={abs=2*\pgfplotbarwidth},
			scaled x ticks=false,
			legend style={
				at={(0.5,-0.1)},
				anchor=north
			},
			legend columns=3,
    		yminorgrids = true,minor tick num=1
			]
				\addplot coordinates {
				(0,RAM-Dump 3-2)  (0,RAM-Dump 3-1) (12,RAM-Dump 2) (0,RAM-Dump 1)
				};
%				\addplot coordinates {
%				(0,RAM-Dump 3-2)  (0,RAM-Dump 3-1) (0,RAM-Dump 2) (0,RAM-Dump 1)
%				};
%				\addplot coordinates {
%				(0,RAM-Dump 3-2)  (0,RAM-Dump 3-1) (0,RAM-Dump 2) (0,RAM-Dump 1)
%				};
%				\legend{firefox.exe, tor.exe, Andere Prozesse}
			\end{axis}
		\end{tikzpicture}
		\\[-7pt]
		\begin{tikzpicture}
			\begin{axis}[
			xbar,
			width=12cm, 
			height=3cm, 
			ylabel style={align=center}, ylabel=\textbf{Vorlesung23!},
			y=0.75cm,
			symbolic y coords={RAM-Dump 3-2, RAM-Dump 3-1, RAM-Dump 2, RAM-Dump 1},
			label style={font=\small},
			tick label style={font=\small},
			ytick=data,
			xticklabels={,,},
            xmin = 0,
            xmax = 110,
			nodes near coords, 
			nodes near coords align={horizontal},
			nodes near coords style={font=\tiny},
   			nodes near coords={\pgfmathfloatifflags{\pgfplotspointmeta}{0}{}{\pgfmathprintnumber{\pgfplotspointmeta}}},
			bar width=.17cm,
			enlarge y limits={abs=2*\pgfplotbarwidth},
			scaled x ticks=false,
			legend style={
				at={(0.5,-0.1)},
				anchor=north
			},
			legend columns=3,
    		yminorgrids = true,minor tick num=1
			]
				\addplot coordinates {
				(0,RAM-Dump 3-2)  (0,RAM-Dump 3-1) (2,RAM-Dump 2) (0,RAM-Dump 1)
				};
%				\addplot coordinates {
%				(0,RAM-Dump 3-2)  (0,RAM-Dump 3-1) (0,RAM-Dump 2) (0,RAM-Dump 1)
%				};
%				\addplot coordinates {
%				(0,RAM-Dump 3-2)  (0,RAM-Dump 3-1) (0,RAM-Dump 2) (0,RAM-Dump 1)
%				};
%				\legend{firefox.exe, tor.exe, Andere Prozesse}
			\end{axis}
		\end{tikzpicture}
		\\[-8pt]
		\begin{tikzpicture}
			\begin{axis}[
			xbar,
			width=12cm, 
			height=3cm, 
			ylabel style={align=center}, ylabel=\textbf{computerforensikvl}\\\textbf{@gmail.com},
			y=0.75cm,
			symbolic y coords={RAM-Dump 3-2, RAM-Dump 3-1, RAM-Dump 2, RAM-Dump 1},
			label style={font=\small},
			tick label style={font=\small},
			ytick=data,
			xticklabels={,,},
            xmin = 0,
            xmax = 110,
			nodes near coords, 
			nodes near coords align={horizontal},
			nodes near coords style={font=\tiny},
   			nodes near coords={\pgfmathfloatifflags{\pgfplotspointmeta}{0}{}{\pgfmathprintnumber{\pgfplotspointmeta}}},
			bar width=.17cm,
			enlarge y limits={abs=2*\pgfplotbarwidth},
			scaled x ticks=false,
			legend style={
				at={(0.5,-0.1)},
				anchor=north
			},
			legend columns=3,
    		yminorgrids = true,minor tick num=1
			]
				\addplot coordinates {
				(0,RAM-Dump 3-2)  (96,RAM-Dump 3-1) (8,RAM-Dump 2) (0,RAM-Dump 1)
				};
%				\addplot coordinates {
%				(0,RAM-Dump 3-2)  (0,RAM-Dump 3-1) (0,RAM-Dump 2) (0,RAM-Dump 1)
%				};
%				\addplot coordinates {
%				(0,RAM-Dump 3-2)  (0,RAM-Dump 3-1) (0,RAM-Dump 2) (0,RAM-Dump 1)
%				};
%				\legend{firefox.exe, tor.exe, Andere Prozesse}
			\end{axis}
		\end{tikzpicture}
		\\[-7pt]
		\begin{tikzpicture}
			\begin{axis}[
			xbar,
			width=12cm, 
			height=3cm, 
			ylabel style={align=center}, ylabel=\textbf{chs3702}\\\textbf{@gmail.com},
			y=0.75cm,
			symbolic y coords={RAM-Dump 3-2, RAM-Dump 3-1, RAM-Dump 2, RAM-Dump 1},
			label style={font=\small},
			tick label style={font=\small},
			ytick=data,
			xticklabels={,,},
            xmin = 0,
            xmax = 110,
			nodes near coords, 
			nodes near coords align={horizontal},
			nodes near coords style={font=\tiny},
   			nodes near coords={\pgfmathfloatifflags{\pgfplotspointmeta}{0}{}{\pgfmathprintnumber{\pgfplotspointmeta}}},
			bar width=.17cm,
			enlarge y limits={abs=2*\pgfplotbarwidth},
			scaled x ticks=false,
			legend style={
				at={(0.5,-0.1)},
				anchor=north
			},
			legend columns=3,
    		yminorgrids = true,minor tick num=1
			]
				\addplot coordinates {
				(0,RAM-Dump 3-2)  (0,RAM-Dump 3-1) (18,RAM-Dump 2) (0,RAM-Dump 1)
				};
%				\addplot coordinates {
%				(0,RAM-Dump 3-2)  (0,RAM-Dump 3-1) (0,RAM-Dump 2) (0,RAM-Dump 1)
%				};
%				\addplot coordinates {
%				(0,RAM-Dump 3-2)  (0,RAM-Dump 3-1) (0,RAM-Dump 2) (0,RAM-Dump 1)
%				};
%				\legend{firefox.exe, tor.exe, Andere Prozesse}
			\end{axis}
		\end{tikzpicture}
		\\[-7pt]
		\begin{tikzpicture}
			\begin{axis}[
			xbar,
			width=12cm, 
			height=3cm, 
			ylabel style={align=center}, ylabel=\textbf{cas0597}\\\textbf{@gmail.com},
			y=0.75cm,
			symbolic y coords={RAM-Dump 3-2, RAM-Dump 3-1, RAM-Dump 2, RAM-Dump 1},
			label style={font=\small},
			tick label style={font=\small},
			ytick=data,
			xticklabels={,,},
            xmin = 0,
            xmax = 110,
			nodes near coords, 
			nodes near coords align={horizontal},
			nodes near coords style={font=\tiny},
   			nodes near coords={\pgfmathfloatifflags{\pgfplotspointmeta}{0}{}{\pgfmathprintnumber{\pgfplotspointmeta}}},
			bar width=.17cm,
			enlarge y limits={abs=2*\pgfplotbarwidth},
			scaled x ticks=false,
			legend style={
				at={(0.5,-0.1)},
				anchor=north
			},
			legend columns=3,
    		yminorgrids = true,minor tick num=1
			]
				\addplot coordinates {
				(0,RAM-Dump 3-2)  (0,RAM-Dump 3-1) (22,RAM-Dump 2) (0,RAM-Dump 1)
				};
%				\addplot coordinates {
%				(0,RAM-Dump 3-2)  (0,RAM-Dump 3-1) (0,RAM-Dump 2) (0,RAM-Dump 1)
%				};
%				\addplot coordinates {
%				(0,RAM-Dump 3-2)  (0,RAM-Dump 3-1) (0,RAM-Dump 2) (0,RAM-Dump 1)
%				};
				\legend{firefox.exe}
			\end{axis}
		\end{tikzpicture}
	\end{tabular}
	}
	\captionof{figure}{Gefundene E-Mail Artefakte im Tor RAM}
	\label{chart:tor-volatility-mail}
\end{table}
%\begin{figure}[h!]
%	\centerline{\resizebox{\linewidth}{!}{\includegraphics{bilder/volatility/tor/mail.png}}}
%	\label{chart:final-criteria}  
%	\caption{Mail}
%\end{figure}
Nur die Absenderadresse ``computerforensikvl@gmail.com`` wurde noch nach Erstellen der ``Neuen Identität`` (RAM-Dump 3-1) gefunden. Die Absenderadresse ist ebenso das am häufigsten gefundene E-Mail Artefakt.
Wie in Abbildung \ref{chart:tor-volatility-mail} dargestellt, wurden die Artefakte ausschließlich in Firefox Prozess gefunden.
Wie bei der Analyse der Firefox RAM-Dumps in Kapitel \ref{subsubsection:ergebnisse-firefox-uncommonlocations-analysemitvolatility}, wurde das Passwort als Klartext nach dem Browsing-Szenario, vor Zuweisung einer ``Neuen Identität``(RAM-Dump 2) gefunden.
Das Passwort wurde zweimal im Firefox Prozess mit der PID 708 gefunden. Tabelle \ref{tab:tor-mapping-virtaddr-to-byteoffset} zeigt die virtuellen Speicheradressen der Artefakte aus der Yarascan Ausgabe sowie deren Abbildung auf die mithilfe des Volatility Plugins \textit{memmap} identifizierten Byte-Offsets der extrahierten Speicherseiten.
\begin{table}[h!]
	\centering
	\resizebox{\linewidth}{!}{
	\begin{tabular}{|c|c|c|ll}
	\cline{1-3}
	\textbf{Virtuelle Speicheradresse} & \textbf{PID} & \textbf{Byte-Offset in extrahierter Speicherseite} &  &  \\ \cline{1-3}
	0x2b1e2c22318                      & 708         & 0xea0318                                         &  &  \\ \cline{1-3}
	0x2b1e2ecb748                     & 708         & 0x10f7748                                         &  &  \\ \cline{1-3}
	\end{tabular}
	}
	\caption{Abbildung der virtellen Speicheradressen der gefundenen Strings im Tor-RAM auf Byte-Offsets der entsprechenden Speicherseiten}
	\label{tab:tor-mapping-virtaddr-to-byteoffset}
\end{table}


\begin{figure}[h!]
	\centering
	\subcaptionbox{Byte-Offset 0xea0318}{\includegraphics[width=0.47\textwidth]{bilder/volatility/tor/password_0xea0318.png}}%
	\hfill
	\subcaptionbox{Byte-Offset 0x10f7748}{\includegraphics[width=0.47\textwidth]{bilder/volatility/tor/password_0x10f7748.png}}%
	\caption{Passwort-Klartext in Firefox Speicherseiten von PID 708}
	\label{img:firefox-pw-offset-pid-708}  
\end{figure}
Bei Untersuchung des String-Kontexts in Abbildung \ref{img:firefox-pw-offset-pid-708}, wurden für das Passwort am Byte-Offset 0xea0318 keine auffälligen Artefakte entdeckt.
Im Bereich des gefundenen Passworts am Byte-Offset 0x10f7748 befindet sich der String ``CSP\_ignoringSrcForStrictDynamic``, dessen Bedeutung nicht näher bestimmt werden konnte.
Weiterhin wurde die Zeichenkette ``invalidation/lcs/client`` in der Nähe des Passworts gefunden. Auf diesen String wird in einem Firefox Bug-Ticket verwiesen, welches bereits 2017 geschlossen wurde. Der Bug betraf ein Speicher-Leck. \cite{Bugzilla.05.06.2023}
	
\paragraph*{Yara-Regel ``DK-Logo``}
Wie in Abbildung \ref{chart:tor-volatility-image} dargestellt, wurde der Hexadezimal-Wert des Donaukurier-Logos ein einziges Mal nach dem Browsing Szenario, vor Erstellen der ``Neuen Identität`` (RAM-Dump 2) in einem Firefox Prozess gefunden.
\begin{figure}[h!]
	\resizebox{\linewidth}{!}{
		\begin{tikzpicture}
			\begin{axis}[
			xbar,
			width=18cm, 
			height=12cm, 
			ylabel style={align=center}, ylabel=\textbf{Donaukurier Logo}\\\textbf{(Hexadezimal)},
			y=0.65cm,
			symbolic y coords={RAM-Dump 3-2, RAM-Dump 3-1, RAM-Dump 2, RAM-Dump 1},
			label style={font=\small},
			tick label style={font=\small},
			ytick=data,
			xticklabels={,,},
            xmin = 0,
            xmax = 2.5,
			nodes near coords, 
			nodes near coords align={horizontal},
   			nodes near coords={\pgfmathfloatifflags{\pgfplotspointmeta}{0}{}{\pgfmathprintnumber{\pgfplotspointmeta}}},	
			legend style={
				at={(1,0)},
				anchor=south east
			},
			legend columns=2,
    		yminorgrids = true,minor tick num=1
			]
				\addplot coordinates {
				(0,RAM-Dump 3-2)  (0,RAM-Dump 3-1) (2,RAM-Dump 2) (0,RAM-Dump 1)
				};
				\legend{firefox.exe}
			\end{axis}
		\end{tikzpicture}
	}
	\caption{Gefundener Hexadezimalwert des Donaukurier-Logos im Tor RAM}
	\label{chart:tor-volatility-image}
\end{figure}
%\begin{figure}[h!]
%	\centerline{\resizebox{\linewidth}{!}{\includegraphics{bilder/volatility/tor/image.png}}}
%	\label{chart:final-criteria}  
%	\caption{Image}
%\end{figure}


%\begin{figure}[h!]
%	\centerline{\resizebox{\linewidth}{!}{\includegraphics{bilder/volatility/tor/summary.png}}}
%	\label{chart:final-criteria}  
%	\caption{Summary}
%\end{figure}
%Literatur:
%o Autopsy: \cite{Muir.2019}
%	•	Configuration files, downloaded files, and browserrelated data are recoverable from the file system.
%	•	Significant data-leakage from the browsing session occurred: HTTP header information, titles of web pages and an instance of a URL were found in registry files, system files, and unallocated space.
%o RAM-Analyse nach \cite{Muir.2019}:
%	•	Live-Analyse identifiziert auch nach dem Schließen und Deinstallieren des Browsers und Abmelden des Benutzers Spuren von Tor-Prozessen, einschließlich des absoluten Pfads zur Browser-Executable, des Benutzernamens und des Geräts, von dem es ausgeführt wurde.
%	•	The data-leakage contained the German word for ’search’ in reference to a Google search. This hints at the locale of the Tor server used to exit the network (exit relay).
%
%o RAM-Analyse nach \cite{Hariharan.2022}:
%	o	process was found to be firefox.exe
%	o	pslist and pstree: parent process was shown 
%	o	Belkasoft Ram Capturer: retrieve information about facebook
%	o	Cmdline: file path of the browser “E:/TorBrowser/Browser/firefox.exe” + name of process tor.exe and firefox.exe
%	o	Dlllist: DLL files of the executable files were not captured
%	o	Netscan: tor.exe + obfs4proxy.exe -> showed “LISTENING” connections to nonstandardized ports as output.
%	Yarascan: was able to retrieve all the browsing sessions
%o RAM-Analyse nach \cite{Sajan.2021} mit Volatility
%	•	process list extracted from the memory
%	•	registry hives been extracted from the memory dump
%	•	threads were extracted: “D:/VolatilityWorkbench/volatility.exe”–plugins=”D:/VolatilityWorkbench/profiles” pslistfilename =”C:/Users/username/Desktop/tor.raw” –profile=Win10x64 17763 –kdbg=0xf807606ac5e0
%	•	Handles: resources used by the process 5672
%	•	Dlls: These dlls can be found from prefetch file --> Can be found in “prefetch” file -> Analyzed with “winprefetchview”
%	•	Places.sqlite: SQLite viewer has been used to recover bookmarks and frequently visited sites even after uninstalling the application
%	•	Visited Websites: Using keyword search in Dump’s Hex
%
%o Registry:
%	> Shellactivites (siehe Firefox) \cite{Muir.2019}: instance of a URL were found in registry file
%	> \cite{Nelson.2020} The userassist key is located in the NTUSER.dat hive of the
%		 -> Registry and indicates the execution path of the program, as well as the number of times the program was executed 

\subsection*{Registry}
Wie in der Methodik in Kapitel \ref{subsection:methodik-datenanalyse-registry} beschrieben, teilt sich die Analyse der Registry sowohl in Common als auch Uncommon Locations. Weder in den Process Monitor ``SetValue`` Operations noch über die Stringsuche in den System- und User-Hives konnten PB Artefakte gefunden werden. Eine detaillierte Analyse der Registry ist im Anhang \ref{subsection:appendix-firefox-registry} beschrieben. 

\section{Chrome}

*** TODO: Christoph ***

%
%\subsection*{Uncommon Locations}
%
%o Autopsy Keyword-Suche: 
%	> Chrome and Edge produced five artefacts as reported by both tools. (FTK, Autopsy) \cite{Gabet.2018}
%		--> Artefakte werden nicht genannt!
%	> only two temporary files (Figure 7) were recovered with Minitool Power Data Recovery but it was a dead end; Location: appdata/…/Chrome/…/ Preferences/RF1533fa.TMP \cite{Fayyad.2021}
%	> pagefile.sys file showed no traces at all \cite{Said.2011}
	

\section{Brave}

*** TODO: Christoph ***
	
	\chapter{Vergleich der Browser}

- Zusammenfassung: Vergleich Tor v. Firefox und Brave v. Chrome
- Firefox v. Chrome ("Standardbrowser")
- Tor v. Brave ("Sichere Browser")
- Zum Schluss: "Eine große Tabelle" mit den wichtigsten Kategorien?
	
	\chapter{Diskussion}

> Artefakte im DNS Cache: \cite{Satvat.2014}
	•	DNS-Caching ist eine Bedrohung für private Browsing
	•	Diese Schwachstelle entsteht, weil das Betriebssystem DNS-Anfragen des Browsers im Cache speichert, unabhängig davon, ob der Browser im privaten Modus ist oder nicht
	•	Mehrere Jahre nach der Meldung dieser Schwachstelle besteht sie immer noch in allen Browsern fort
	•	Es wurden einige Erweiterungen von Drittanbietern entwickelt, um dieses Problem zu beheben, aber keine davon wurde von den Browserherstellern übernommen.
	

> Viele RAM-Artefakte
	- Firefox \cite{Muir.2019}
		•	Darcie et al. (2014) fanden Beweise für das Web-Browsing in Form von JPEG- und HTML-Dateien in Live-Forensik, aber eine statische Forensik war erfolglos.
		•	Eine vorherige Live-Forensik-Analyse des Firefox-Browsers zeigte, dass Artefakte aus einer privaten Browsing-Sitzung aus dem Speicher wiederhergestellt werden konnten. (Findlay and Leimich, 2014). 
		

> IE hinterlässt viele Spuren im Gegensatz zu Ergebnissen: \cite{Md.2018}
	o	hidden folders are usually stored at C/Users/User/AppData
	o	evidence searches are conducted extensively in the C:\ partition
	o	bookmarks remain and can be viewed
	o	downloads remain in the downloads folder until the user manually deletes them
	o	CacheView trace entire URL and browsing histories including the temporary files
	CacheView enables to find the image’s URL and from specific website
	
> Urteil über die Privatheit von Tor nach \cite{Muir.2019}
	The design aim of preventing Tor from writing to disk (Perry et al., 2018) is not achieved in this version.
		•	Configuration files, downloaded files, and browserrelated data are recoverable from the file system.
		•	Significant data-leakage from the browsing session occurred: HTTP header information, titles of web pages and an instance of a URL were found in registry files, system files, and unallocated space.
		•	The data-leakage contained the German word for ’search’ in reference to a Google search. This hints at the locale of the Tor server used to exit the network (exit relay).
	The Tor Project’s design aim of enabling secure deletion of the browser (Sandvik, 2013) is not achieved in this version.
		•	References to: the installation directory, Firefox SQLite files, bridging IPs/ports, default bookmarks, Tor-related DLLs and Tor product information were all recovered after the browser was deleted.
		•	In a scenario where the operating system paged memory, an instance

Weiterführende Arbeiten:
> Cross-mode interference \cite{Hedberg.2013}:
	o	the Chrome://memory page displays all the opened tabs in the browser regardless if they are in the usual or private mode -> Nicht mehr aktuell -> Stattdessen: Chrome Task-manager (Ctrl + Esc), Funktioniert auch bei Firefox
> Unser Scope: Process Monitor nach Prozessnamen gefiltert
	- Weiterführend: Nach Pathnamen filtern: "Common Locations"

> Für wen wird Browser entwickelt
> Warum und für wen wird Private Browsing analysiert?
> Ist das Auffinden privater Browsing Artefakte Schuld von Browser Entwicklern? (Oder Schuld des Betriebssystem, wie in (TODO!) erwähnt)

> bei Process Monitor nur nach Browser-Prozessen gefiltert

> Warum in Literatur nur Windows untersucht?

Tor:
	"Unsere Mission:
	Menschenrechte und Freiheiten durch die Entwicklung und Verbreitung von Open Source Anonymitäts- und Privatsphäre-Technologien zu fördern, ihre ungehinderte Verfügbarkeit zu unterstützen und ihr Verständnis in Wissenschaft und der Allgemeinheit zu vergrößern."
	% https://www.torproject.org/de/download/


	
	\chapter{Fazit und Ausblick}\label{chap:Fazit-Ausblick}
\thispagestyle{plain.scrheadings}
\ohead{\headmark}

Im Rahmen dieser Arbeit wurden vier verschiedene Browser im privaten Modus auf hinterlassene Browsing-Artefakte untersucht. Ein transparenter Ansatz wurde verfolgt, bei dem ein Browsing-Szenario definiert und erwartete Artefakte festgelegt wurden. Die Aktivitäten der Browser wurden vor, während und nach dem Szenario aufgezeichnet sowie Festplatten- und Arbeitsspeicherabbilder wurden erstellt. Die Artefakte wurden an an gängigen sowie ungewöhnlichen Speicherorten, wie dem Arbeitsspeicher analysiert. Ein quantitativer Vergleich der Browser wurde durchgeführt, wobei der Tor-Browser die geringste Menge an Artefakten hinterließ. Qualitative Kriterien wie die Nutzererfahrung wurden nicht berücksichtigt.

Obwohl keine Artefakte in den Festplattenabbildern gefunden wurden, wurden im Arbeitsspeicher Spuren entdeckt. Der private Modus verhindert zwar das Speichern von Browser-Cache und -Verlauf, aber bei einer forensischen Analyse des laufenden Systems können dennoch eindeutig zuordenbare Artefakte gefunden werden.

Die identifizierten Schwachstellen stellen Browser-Entwickler vor einem Dilemma.
Von diesen Ergebnissen profitieren einerseits Browser-Nutzer, da Browser-Entwickler auf Basis der Ergebnisse die aufgedeckten Schwächen beseitigen können.
Allerdings besteht auch das Risiko des Missbrauchs des privaten Browsens für illegale Aktivitäten.
Das Schließen von Sicherheitslücken erschwert es somit Forensikern, kriminelle Aktivitäten aufzudecken und nachzuweisen. 
Andererseits profitieren Forensiker selbst von solchen Analysen, um bei identifizierten Schwachstellen gezielt nach Artefakten zu suchen. 

Zukünftige Arbeiten könnten den Umfang dieser Arbeit um folgende Punkte erweitern, welche aufgrund von Zeitbeschränkungen und begrenztem Umfang nicht möglich war. Statt nach Prozessnamen zu filtern, könnten Process Monitor Logfiles nach den bekannten Speicherorten durchsucht werden, um festzustellen, ob andere Prozesse dort Schreibaktivitäten ausführen. Eine detailliertere Untersuchung weiterer Betriebssystem-Speicherorte wie des DNS-Caches könnte vorgenommen werden. Es wäre wichtig, eine Analyse von Browsern für Mac oder Linux durchzuführen, da bisherige Literatur nur Windows betrachtet. Außerdem könnte der private Modus direkt mit dem normalen Browsing-Modus verglichen werden, um Unterschiede festzustellen. Schließlich wäre es interessant, weitere gängige Browser wie Microsoft Edge oder Safari einer forensischen Analyse zu unterziehen.

Diese Arbeit hat den persönlichen Wissensstand der Autoren erweitert, indem sie sich intensiv mit forensischer Analyse, Tools, Browsern, virtuellen Maschinen und dem Verhalten des Windows-Betriebssystems auseinandergesetzt haben. Die Arbeit mit Speicherabbildern verdeutlichte den umfangreichen und zeitintensiven Aufwand solcher Untersuchungen. Insgesamt war es eine spannende und anspruchsvolle Erfahrung.
%
%Diese Arbeit hat nicht nur zu neuen Erkenntnissen geführt, sondern auch den persönlichen Wissensstand der Autoren erweitert. Durch die intensive Auseinandersetzung mit forensischer Analyse, spezifischen Tools und relevanten Themen wie Browsern, virtuellen Maschinen und dem Verhalten des Windows-Betriebssystems konnten die Autoren ihr Fachwissen deutlich ausbauen. Die Arbeit mit Speicherabbildern und die komplexe Analyseprozesse haben gezeigt, welchen umfangreichen und zeitlichen Aufwand eine solche Untersuchung erfordert. Insgesamt war es eine faszinierende und anspruchsvolle Erfahrung.
%
%
%Neben neuen Erkenntnissen, wie dem Vorgehen einer forensischen Analyse und speziell dafür geeignete Tools, konnte bereits vorhandenes Wissen zu den Themen Browser, die Verwendung von virtuellen Maschinen oder dem Verhalten sowie wichtigen Dateien des Windows Betriebssystems erweitert werden. Speziell die Arbeit mit Speicherabbildern, wie Arbeitsspeicherabbildern oder ganzen Snapshots von virtuellen Maschinen, war bzgl. des Vorgehens und der Analyse neu und interessant, wie die Extraktion möglichst vieler Informationen wie aus einer Art Speicher-Blackbox abläuft. Dabei ist sehr deutlich geworden, welch großen und zeitlichen Aufwand sowohl hinter der Ergebung dieser Abbilder sowie deren Auswertung steckt. Zusammenfassend war es jedoch eine sehr spannende und herausfordernde Arbeit.

	
	\begin{appendices}
	%	\addtocontents{lof}{\protect\contentsline{chapter}{%
			%			Appendices \vspace{10pt}
			%	}{}
%		\addtocontents{toc}{\protect\setcounter{tocdepth}{1}}
%		\makeatletter
%		\addtocontents{toc}{%
%			\begingroup
%			\let\protect\l@chapter\protect\l@section
%			\let\protect\l@section\protect\l@subsection
%		}
%		\makeatother
		
%		\huge{\textbf{All File Operations Firefox}}
%%		\enlargethispage{6cm}
%		\begin{figure}[h!]
%			\centerline{\resizebox{!}{0.9\textheight}{\includegraphics{bilder/firefox-all-write-operations.png}}}
%			%	\label{...}
%			\caption{All File Operations Firefox: Logfile 1 vs. Logfile 2}
%		\end{figure}
%		
%		\huge{\textbf{All File Operations Tor}}
%%		\enlargethispage{6cm}
%		\begin{figure}[h!]
%			\centerline{\resizebox{!}{0.9\textheight}{\includegraphics{bilder/tor-all-write-operations.png}}}
%			%	\label{...}
%			\caption{All File Operations Firefox: Logfile 1 vs. Logfile 2 vs. Logfile 3}
%		\end{figure}

\subsection*{Firefox Common Locations}

\subsubsection*{Process Monitor WriteFile Operations}

Gemäß Versuchsdurchführung in Abbildung X (TODO!) wurden für Firefox mit dem Process Monitor Tool zwei Logfiles erstellt. Diese Dateien enthalten alle aufgezeichneten Prozessaktivitäten während und nach dem Browsing Szenario.
Zunächst werden die beiden Logfiles gemäß Methodik in Kapitel X (TODO!) in Excel aufbereitet. 
Im Anhang X (TODO!) ist dazu eine Tabelle mit allen in den gefilterten Logfiles identifizierten Dateien aufgeführt.
Dabei wurde für jede Datei vermerkt
ob und wie sie wiederherstellbar war, mit welchem Tool die Datei analysiert wurde und ob PB Artefakte enthalten sind.

Abbildung X (TODO!) zeigt diese Tabelle in reduzierter Darstellung.
Dazu wurden ausschließlich wiederherstellbare Dateien aufgeführt. 
Die Dateien wurden in die fünf Kategorien "Cache", "datareporting", "Sessionstore-Backup" und "Sonstige Dateien" eingeordnet.
Für jede Datei wurde vermerkt, ob in der entsprechenden Logfile PB Artefakte geschrieben wurden.
Dies trifft für keine der identifizierten Dateien zu.
\begin{figure}[h!]
	\resizebox{\linewidth}{!}{\includegraphics{bilder/firefox-tabelle-logfile1vlogfile2-reduced.png}}
%	\label{...}
	\caption{Tabelle mit wiederherstellbaren Dateien: Logfile 1 vs. Logfile 2}
\end{figure}

Bei detaillierter Untersuchung der Dateien, können zwei Pfade identifiziert werden, in die Firefox während des Versuchs Dateien schreibt. Nur die Dateien in der Cache Kategorie sind im Local Pfad gespeichert.
\begin{itemize}
\item[\textbf{Local}] \texttt{C:$\backslash$Users$\backslash$<User>$\backslash$AppData$\backslash$Local$\backslash$Mozilla$\backslash$Firefox$\backslash$Profiles$\backslash$<Profile>.default-release$\backslash$}
\item[\textbf{Roaming}] \texttt{C:$\backslash$Users$\backslash$<User>$\backslash$AppData$\backslash$Roaming$\backslash$Mozilla$\backslash$Firefox$\backslash$Profiles$\backslash$<Profile>.default-release$\backslash$}
\end{itemize}
In Tabelle X (TODO!) sind die Dateien je nach Speicherort "Local" (Hellblau) oder "Roaming" (Dunkelblau) entsprechend eingefärbt. 

\paragraph*{Cache}
Firefox verwendet den Cache, um Webseiten und deren Ressourcen temporär lokal zu speichern. Dadurch können wiederholte Anfragen an den Server vermieden und die Ladezeiten verringert werden. Die Inhalte dieser Dateien sind binär.
Die Dateien im Format \texttt{$\backslash$cache2$\backslash$entries$\backslash$<ID>} werden dem Cache zugeordnet und im Local Pfad gespeichert.
% https://www.techguy.org/threads/what-exactly-is-in-firefoxs-cache2-folder.1221567/
Wie in Kapitel X beschrieben, können diese Dateien mit dem Tool MZCacheView eingelesen werden.
Wie in Abbildung X gezeigt, konnten im Cache-Ordner im zweiten Snapshot drei JSON Dateien identifiziert werden. Dabei handelt es sich um Zertifikatsdateien, die von der "One Certificate Revocation List" stammen, ein Mechanismus von Firefox zur Überprüfung von Zertifikaten. In keinem der Zertifikate konnten mit HxD private Browsing Artefakte oder besuchte Seiten gefunden werden.
Weiterhin befindet sich im Cache das HTML-Dokument der Firefox Datenschutzseite, welche sich beim ersten Start des Browsers automatisch öffnete. % TODO: Siehe Kapitel X
Weitere Cache Dateien konnten in keinem Snapshot gefunden werden.
\begin{figure}[h!]
	\resizebox{\linewidth}{!}{\includegraphics{bilder/firefox-cache.png}}
%	\label{...}
	\caption{Tabelle mit wiederherstellbaren Dateien: Logfile 1 vs. Logfile 2}
\end{figure}
Die Indexdatei \texttt{$\backslash$cache2$\backslash$index} dient als Datenbank im Cache. Sie ermöglicht dem Firefox-Browser, schnell auf die zwischengespeicherten Ressourcen zuzugreifen und diese effizient zu verwalten. Sowohl mit HxD als auch dem Tool FirefoxCache2 konnten keine PB Artefakte identifiziert werden.
*** TODO: Erst in Logfile 2 geschrieben ***

Schließlich enthält die Datei \texttt{$\backslash$jumpListCache$\backslash$ZKJGVJPzPe7w4w0KwEY0jw==.ico} ein $64x64$ Pixel großes Mozilla Logo. Dieses Logo ist keinem Schritt aus dem Browsing Szenario zuzuordnen


\paragraph*{Datareporting}
Dateien im Ordner \texttt{$\backslash$datareporting$\backslash$glean$\backslash$db} sind Teil des Glean-Systems, das für die Sammlung von Telemetriedaten und deren Übermittlung an Mozilla verwendet wird. 
% https://github.com/mozilla/glean
Die Datei \texttt{data.safe.bin} enthält verschlüsselte und anonyme Informationen über die Nutzung des Browsers. In HxD konnten keine keine PB Artefakte gefunden werden
*** TODO: Vergleich Logfile 1 vs 2 ***

Dateien im Foremat \texttt{$\backslash$datareporting$\backslash$glean$\backslash$db$\backslash$<Profilname>.new-profile.jsonlz4} speichern Informationen über das Firefox-Profil, das von Glean verwendet wird. Wie in Kapitel X beschrieben, lassen sich Dateien, im proprietären \textit{jsonlz4}-Format mit dem Tool dejsonlz4 dekomprimieren. Die entstandene JSON Datei wird mit dem Notepad++ JSON Plugin untersucht. Dabei konnten keine PB Artefakte gefunden werden.
*** TODO: Nur in Logfile 2 geschrieben ***

\paragraph*{Sessionstore}
Die Datei \texttt{$\backslash$sessionstore-backups$\backslash$recovery.jsonlz4} enthält eine Sicherungskopie der vorherigen Sitzung. Sie wird erstellt, wenn der Firefox-Browser nach einem Absturz oder einem unerwarteten Beenden neu gestartet wird." % https://support.mozilla.org/de/questions/1221836
Jefferson Scher entwickelte ein Online-Tool zur Analyse von \textit{Sessionstore-Backup} Dateien.
% https://www.jeffersonscher.com/ffu/scrounger.html)
In der Sitzungswiederherstellung konnten wie in Abbildung X gezeigt lediglich die automatisch geöffnete Seite über Firefox Datenschutzhinweise identifiziert werden.
\begin{figure}[h!]
	\resizebox{\linewidth}{!}{\includegraphics{bilder/firefox-sessionstore.png}}
%	\label{...}
	\caption{Tabelle mit wiederherstellbaren Dateien: Logfile 1 vs. Logfile 2}
\end{figure}
*** TODO: Logfile 1 vs Logfile 2 ***

\paragraph*{Sonstige Dateien}
In der Datei \texttt{prefs-1.js} werden benutzerspezifische Einstellungen und Konfigurationen für den Firefox-Browser gespeichert. Die Datei enthält Präferenzen des Benutzers in Form von JavaScript-Objekten. Es konnten mit HxD keine PB Artefakte gefunden werden.
% https://kb.mozillazine.org/Prefs.js_file
*** TODO: Vergleich Logfile 2 ***

Schließlich speichert die Datei \texttt{xulstore.json} benutzerspezifische Anpassungen und Konfigurationen für den Firefox-Browser. In der Datei konnten mit Notepad++ keine PB Artefakte gefunden werden.
% https://support.mozilla.org/de/kb/firefox-support-troubleshooting-guide
*** TODO: Vergleich Logfile 2 ***	
\subsubsection*{SQLite Datenbänke}
Wie in Kapitel X (Methodik, TODO!) erwähnt, werden SQLite Datenbanken als Datenstrukturen für Nutzerdaten genauer untersucht. Mithilfe der Process Monitor Logfiles wurden die in Tabelle X dargestellten SQLite-Datenbanken für Firefox identifiziert:

\begin{table}[]
\resizebox{\linewidth}{!}{
\begin{tabular}{|l|l|lll}
\cline{1-2}
\textbf{Datenbank}                        & \textbf{Gespeicherte Daten}                                                                                              &  &  &  \\ \cline{1-2}
\textit{places.sqlite}                    & Informationen über Lesezeichen und Verlauf. Zu jeder besuchten Webseite: URL, Seitentitel, Zeitstempel des Besuchs etc.  &  &  &  \\ \cline{1-2}
\textit{cookies.sqlite}                   & Von besuchten Webseiten verwendete Cookies.                                                                              &  &  &  \\ \cline{1-2}
\textit{storage.sqlite}                   & Diverse Webdaten, z. B. Indexed-Datenbanken, Offline-Cache-Daten und andere lokale Speicherinformationen.                &  &  &  \\ \cline{1-2}
\textit{favicons.sqlite}                  & Enhtält Favicons (kleine Symbole in der Adressleiste) um besuchte Webseiten visuell zu identifizieren.                   &  &  &  \\ \cline{1-2}
\textit{webappsstore.sqlite}              & Speichert Informationen über installierte Webanwendungen im Firefox-Browser, z.B. Berechtigungen und Einstellungen.      &  &  &  \\ \cline{1-2}
\textit{1657114595AmcateirvtiSty.sqlite}  & Datenspeicher für Activity Stream, eine personalisierte Übersicht über Browser-Aktivitäten beim Öffnen eines neuen Tabs. &  &  &  \\ \cline{1-2}
\textit{3870112724rsegmnoittet-es.sqlite} & Datenspeicher für Remote Settings, eine zentrale Verwaltung von benutzerspezifischen Browsereinstellungen.               &  &  &  \\ \cline{1-2}
\end{tabular}
}
\end{table}

Jede dieser Datenbanken wurde in allen vier Snapshots miteinander verglichen. Die Dateiextraktion und Dateianalyse erfolgte analog zur Methodik in Kapitel X (TODO!).
Die Ergebnisse wurden in Tabelle X (TODO!) dargestellt.

\begin{figure}[h!]
	\centerline{\resizebox{\linewidth}{!}{\includegraphics{bilder/firefox-sqlite-table.png}}}
	\label{chart:final-criteria}  
	\caption{Comparison of found PB artifacts between RAM Dumps}
\end{figure}
Nach Browser-Installation (Snapshot 1) existierte noch keine der SQLite-Dateien.

Nach dem Browsing Szenario (Snapshot 2) wurde festgestellt, dass alle SQLite-Datenbanken 
initialisiert wurden, außer \texttt{webappsstore.sqlite}. Dabei wurden in \texttt{places.sqlite} die automatisch im normalen Modus geöffnete Datenschutzhinweise Seite eingetragen. 
In restlichen Datenbanken wurden leer initialisiert, nur die Spaltennamen wurden eingetragen.
Der Inhalt aller erstellten Datenbanken blieb nach Durchführung von PRAGMA WAL Checkpoints unverändert.

Nach Schließen des Browsers (Snapshot 3) wurden in \texttt{places.sqlite} die Indizes bei eingetragenen Seiten aktualisiert. Die SQLite-Datenbank \texttt{1657114595AmcateirvtiSty.sqlite} erhielt ein binäres Datenobjekt als Eintrag. Bei der Untersuchung mit HxD konnten keine Artefakte gefunden werden. Weiterhin wurde \texttt{webappsstore.sqlite} leer initialisiert. Die restlichen Daten blieben im Vergleich mit Snapshot 2 unverändert. Ebenfalls veränderte sich nicht der Inhalt nach Durchführung von PRAGMA WAL Checkpoints.

Nach herunterfahren der VM (Snapshot 4) gab es keine Änderungen in den SQLite Datenbanken, auch nach Durchführung der PRAGMA WAL Checkpoints.
	
Somit wurden in den SQLite Datenbanken von Firefox keine zurückverfolgbaren PB Artefakte im privaten Modus hinterlassen.


Mithilfe des Process Monitors wurde festgestellt, dass sowohl während des Browsing Szenarios (Logfile 1) als auch danach (Logfile 2) Inhalte in Dateien geschrieben wurden. Wie zusammenfassend in Abbildung X (TODO!) dargestellt, wurde mit Ausnahme der Datareporting Dateien gab es in Logfile 1 stets mehr oder genauso viele Schreiboperationen wie in Logfile 2.
Keine Schreiboperation hinterließ jedoch Private Browsing Artefakte.
\begin{figure}[h!]
	\centerline{\resizebox{\linewidth}{!}{\includegraphics{bilder/bar-chart-logfile1vs2-test.png}}}
	\label{chart:final-criteria}  
	\caption{Comparison of found PB artifacts between RAM Dumps}
\end{figure}


%Literatur:
%	o no traces were found in “common locations” \cite{Montasari.2015}
%		>  “places.sqlite”, “webappsstore. sqlite”, “sessionstore.bak”, “search.json” and “nssckbi.dll”
%	o	Safebrowsing: Alle Dateien in /safebrowsing-updating/ nicht relevant. Dort nur .vlpset und .sbstore Dateien. Speichern 256-Bit Hash von URLs, die auf SafeSearch Blacklist stehen 
%	o	Cache-Dateien: drei Caches: startupCache, jumpListCache (beide enthalten Binärdateien ohne Browsing Artefakte) und cache2 (können mit MozillaCacheView untersucht werden, enthalten keine Browsing Artefakte)
%	o	SQLite Datenbanken: Sqlite Dateien erst ohne WAL Dateien untersuchen, Danach mit sqlite3 Konsole: WAL in Datenbank schreiben mit: PRAGMA wal\_checkpoint; places.sqlite besonders relevant, da dort Browser in public Modus Browsing URLs verwaltet (Am besten hier vergleich mit Public Browsing machen)	
%		> \cite{Fayyad.2021} for Mozilla Firefox, 7 database files were recovered: cookies.sqlite-shm, places.sqlite-shm, prefs.js etc.
%		> \cite{Muir.2019} The two SQLite databases used by Firefox to track cookies and history (cookies.sqlite und places.sqlite) were both recoverable from the file system after deletion	
%		Ergebnisse stehen im Gegensatz zu \cite{Hedberg.2013} :
%			o	Chrome und Firefox: Einträge in places.sqlite + history.sqlite DB gefunden während PB! (Noch aktuell??)
%		Sonderfall: SQlite DB-Crash \cite{Hedberg.2013}
%			> WAL Files/Journal Files bei Crash gefunden -> Kann genutzt werden um zu beweisen, dass privater Browser genutzt wurde
%			> Daher: WAL Rollback mit sqlite3	
%	o	Jsonlz4 \& balkz4: Enthalten komprimierte Firefox-Sessions, jsonlz4 Dateien können mit Tool "entkomprimiert" werden: https://www.jeffersonscher.com/ffu/scrounger.html

\subsection*{Firefox Uncommon Locations}

\subsubsection*{Analyse mit Autopsy}

Gemäß Methodik in Kapitel X wurden die Dateien der Kategorien "Web Bookmarks", "Web Cookies", "Web History" sowie "Web Categories" analysiert.
Beim Vergleich der Festplattenabbilder wurde festgestellt, dass ein Snapshot stets die kategoriesierten Dateien des vorherigen Snapshots enthielt. Es sind innerhalb einer Kategorie nur neue Dateien dazugekommen. Somit enthält Snapshot 4 in jeder Kategorie alle Dateien der vorherigen Snapshots.

*** TODO: Edge vorinstalliert erwähnen ***

\paragraph*{Web Bookmarks}

Bereits vor Durchführung des Browsing Szenarios enthielt Firefox im ersten Snapshot die Bing Startseite als gespeichertes Leesezeichen. In den restlichen Snapshots 2 -- 4 blieb diese Kategorie unverändert.

\begin{figure}[h!]
	\centerline{\resizebox{\linewidth}{!}{\includegraphics{bilder/cfv_firefox_autopsy_web_bookmarks.png}}}
	\label{chart:final-criteria}  
	\caption{Autopsy Web Bookmarks}
\end{figure}

\paragraph*{Web Cookies}
Auch diese Kategorie enthält bereits vor Beginn des Browsing Szenarios zehn Cookie-Einträge in der Datei \texttt{WebCacheV01.dat}. Dabei handelt es sich um eine Datenbank des Microsoft Edge Browsers zur Speicherung von Nutzerdaten. Diese Datei verhält sich ähnlich wie die in diesem Versuch relevanten SQLite-Dateien. Die Datei enhält. Bei den Einträgen handelt es sich um Cookies für Bing und die Outlook Webseite, obwohl diese Seiten nie in Microsoft Edge geöffnet wurden. In den Snapshots 2 -- 4 kamen keine weiteren Einträge in dieser Kategorie hinzu.
\begin{figure}[h!]
	\centerline{\resizebox{\linewidth}{!}{\includegraphics{bilder/cfv_firefox_autopsy_web_cookies.png}}}
	\label{chart:final-criteria}  
	\caption{Autopsy Web Cookies}
\end{figure}

\paragraph*{Web History}
Diese Kategorie listet alle Dateien mit gespeichertem Suchverlauf auf. Vor Beginn des Browsing Szenarios (Snapshot 1) enthält die Kategorie ebenfalls zwei Einträge zur Outlook Webseite in der Datei \texttt{WebCacheV01.dat}. Nach Durchführung des Browsing Szenarios (Snapshot 2) wurde ein Eintrag in der \texttt{places.sqlite} Datenbank hinzugefügt. Dabei handelt es sich um die automatisch im normalen Browsingmodus geöffnete Firefox-Standardseite über Datenschutzhinweise. Dies deckt sich mit den Beobachtungen der Common Locations in Kapitel X. Darüber hinaus enthält dieser Snapshot für die Datei \texttt{WebCacheV01.dat} den Eintrag \texttt{file:///Z:/Logfile\_1}. Dabei handelt es sich um das Process Monitor Logfile, das gemäß Methodik in Kapitel X (TODO!) über den gemeinsamen VM-Ordner zum Analyse-Rechner transportiert wurde. Ergänzt wird das in Snapshot 3 durch den Eintrag \texttt{file:///Z:/Logfile\_2}, dem zweiten Process Monitor Logfile. In Snapshot 4 werden in dieser Katgeorie keine neuen Dateien erfasst.
\begin{figure}[h!]
	\centerline{\resizebox{\linewidth}{!}{\includegraphics{bilder/cfv_firefox_autopsy_web_history.png}}}
	\label{chart:final-criteria}  
	\caption{Autopsy Web History}
\end{figure}

\paragraph*{Web Categories}
Diese Kategorie klassifiziert im Speicherabbild gefundene Browsing Artefakte nach Inhalt.
Vor Beginn des Browsing Szenarios (Snapshot 1) werden hier bereits zwei Einträge aufgelistet. Der Eintrag \texttt{bing.com} wird als "Suchmaschine" klassifiziert und \texttt{live.com} als "Web-Email".
Wie oben erwähnt, wurden beide Seiten nie aufgerufen. Es gab keine zusätzlichen Einträge in dieser Kategorie in den Snapshots 2 bis 4.
\begin{figure}[h!]
	\centerline{\resizebox{\linewidth}{!}{\includegraphics{bilder/cfv_firefox_autopsy_web_categories.png}}}
	\label{chart:final-criteria}  
	\caption{Autopsy Web Categories}
\end{figure}
		

Somit wurden in allen Kategorien ausschließlich Browsing Artefakte des Edge Browsers in der Datei \texttt{WebCacheV01.dat} gefunden, sowie ein Eintrag in der Firefox SQLite Datenbank \texttt{places.sqlite}. Die eingetragene Firefox-Standardseite deckt sich mit den Ergebnissen der Common Locations in Tabelle X. Die aufgelisteten Einträge in der Datei \texttt{WebCacheV01.dat} sind nicht auf Schritte des Browsing Szenarios zurückzuführen. Die Einträge sind bereits im ersten Snapshot enthalten, obwohl vor Beginn des Browsing Szenarios keine Browseraktivitäten durchgeführt wurden. Weiterhin enthölt diese Datei Einträge über die Process Monitor Logfiles, welche über einen gemeinsamen VM-Ordner zum Rechner transportiert wurde, auf dem die virtuelle Maschine läuft.
In keiner der Kategorien konnten private Browsing Artefakte identifiziert werden.

%Literatur:
%	o	Autopsy Keywortsuche: 
%		>	In alles Snapshots ergebnislos (keine Keyword-Hits
%		-->	In Literatur: Autoren fanden Ergebnisse in pagefile.sys 
%			> Autopsy: websites and some of the keywords found in hidden file called “pagefile.sys” \cite{Mahlous.2020}
%			o \cite{Montasari.2015} traces were found in: 
%				> However, on investigating the “pagefile.sys”, some entries were discovered
%				> Using the “data carving” technique, profile picture was recovered
%			o \cite{Said.2011} 
%				> Examining pagefile.sys showed some positive hits 			
%		--> Evtl. hier zeigen, was gefunden werden kann, wenn RAM reduziert
%		--> Aber auf Problem hinweisen, dass gefundener String in pagefile nicht direkt Browser zugeordnet werden kann
%		> \cite{Gabet.2018}	Firefox only produced three recoverable artefacts as reported by both tools (FTK, Autopsy) --> Artefakte werden nicht genannt!
%		> \cite{Muir.2019} Autopsy Keyword Suche nach Suchbegriffen: unallocated space
%		> Autopsy Carving Module (\$Carved): \cite{Muir.2019}
%			•	When searching for the string ’clot’ from the browsing protocol, six .dll, .edb and .reg files were discovered in unallocated space.
%			•	Further searching of unallocated space uncovered references to the Tor installation directory and the obfs4 bridging IP addresses
%			•	browsing data found in NTUSER.DAT was also replicated in unallocated space.
%	o	Autopsy PlugIns:
%		>	*** TODO: Hier Liste mit PlugIns ***

\subsection*{Firefox Registry}

\subsubsection*{Process Monitor SetValue Operations}

Als Teil der Common Locations werden für Firefox alle Registry "SetValue" Schreiboperationen der beiden Process Monitor Logfiles untersucht.

In beiden Logfiles wurden zwei Kategorien von Registry Keys geschrieben: "PreXULSkeletonUISettings" und "Business Activity Monitoring". In Abbildung X ist der Anteil der Schreiboperationen je Kategorie für beide Logfiles gezeigt.
\begin{figure}[h!]
	\centerline{\resizebox{\linewidth}{!}{\includegraphics{bilder/firefox-registry-stacked-bar-chart.png}}}
	\label{chart:final-criteria}  
	\caption{Comparison of found PB artifacts between RAM Dumps}
\end{figure}

\paragraph*{PreXULSkeletonUISettings}
%https://itigic.com/skeleton-ui-new-firefox-interface-to-start-up-much-faster/#google_vignette
Der "PreXULSkeletonUISettings" Registry Key enthält Einstellungen für die Benutzeroberfläche (UI) des Firefox-Browsers, insbesondere für das sogenannte "Skeleton UI, eine vereinfachte Benutzeroberfläche, die während des Ladens des Browsers angezeigt wird, bevor die vollständige Benutzeroberfläche geladen ist. 
PreXULSkeletonUISettings Registry Keys haben das Format \texttt{HKCU$\backslash$SOFTWARE$\backslash$Mozilla$\backslash$Firefox$\backslash$PreXULSkeletonUISettings$\backslash$<Absoluter Firefox Installationspfad>$\backslash$firefox.exe|<Skeleton UI Setting>}.
Somit enthält der Key den absoluten Installationspfad von Firefox gefolgt von einer Skeleton UI Einstellung. Nachfolgend sind alle möglichen UI Einstellungen aufgelistet, gefolgt vom Datentyp des Keys.
\begin{itemize}
	\item ScreenX (DWORD)
	\item ScreenY (DWORD)
	\item Width (DWORD)
	\item Height (DWORD)
	\item Maximized (DWORD)
	\item Flags (DWORD)
	\item CssToDevPixelScaling (REG\_BINARY)
	\item UrlbarCSSSpan (REG\_BINARY)
	\item SearchbarCSSSpan (REG\_BINARY)
	\item SpringsCSSSpan (REG\_BINARY)
\end{itemize}
Somit enthalten die Keys nur Daten zur Formatierung und Struktur der grafischen Oberfläche. Es wurden keine PB Artefakte geschrieben

\paragraph*{Business Activity Monitoring}
"Business Activity Monitoring", kurz BAM ist eine weitgehend undokumentierte Windows Funktion, die im Hintergrund ausgeführte Programme steuert.
Der Registry Key hat das Format \texttt{HKLM$\backslash$System$\backslash$CurrentControlSet$\backslash$Services$\backslash$bam$\backslash$State$\backslash$UserSettings$\backslash$<SID>$\backslash$Device$\backslash$HarddiskVolume2$\backslash$<Absoluter Firefox Installationspfad>$\backslash$firefox.exe} und den Datentyp REG\_BINARY.
Jeder Schlüssel wird durch die Sicherheits-ID (SID) des Benutzers identifiziert.
Ein BAM Registry Key schreibt für alle ausgeführten Programme --- hier Firefox --- den Zeitstempel der letzten Ausführung.
PB Artefakte sind dabei nicht enthalten.
% https://learn.microsoft.com/de-de/biztalk/core/business-activity-monitoring-bam
% https://notes.qazeer.io/dfir/windows/_artefacts_overview 

\subsubsection*{Stringsuche in Registry Hives}

Gemäß Methodik in Kapitel X wird die Firefox Registry als Uncommon Location behandelt, indem über alle auf der Festplatte vorhandenen Registry Datenbanken, den Registry-Hives, eine Stringsuche durchgeführt wird, ohne die Struktur der Hives zu beachten. 
Dazu wurden sowohl die System-Hives als auch die User-Hives aus Tabelle X (TODO!) aus jedem Snapshot extrahiert und mithilfe des Registry Explorers nach PB Artefakten durchsucht.
Dabei wurde in keinem Snapshot in keinem Hive ein PB Artefakt gefunden.

%Literatur:
%	>	Auf Autor verweisen: angeblich in Shellactivities Ergebnisse. --> Nicht mehr vorhanden in aktueller Version (Verweis auf E-Mail)
%	>	Process Monitor/Regshot zeigen keine relevanten Key-Änderungen
%	> \cite{Muir.2019}: Autopsy Keyword Suche nach Suchbegriffen: Ergebnisse in \%SystemRoot\%Minidump NTUSER.DAT, ntuser.dat.LOG1 (a log of changes to NTUSER.DAT)
%	> Zentral: shellactivites Key:	NTUSER.DAT --> “shellactivities” key \cite{Muir.2019}
%	> \cite{Rochmadi.2017} Detection of registry changes helps to determine what the appropriate plugin is used to search for digital evidence using volatility memory forensic:
%	- RegQueryValue:	HKCU/Software/Microsoft/Windows/CurrentVersion/InternetSettings/Connections/DefaultConnectionSettings
%	- RegCloseValue: 	HKCU/Software/Microsoft/Windows/CurrentVersion/InternetSettings/Connections
%	- IRP\_MJ\_READ: C:/pagefile.sys


%%%%%%%%%%%%%%%%%%%%%%%%%%%%%%%%%%%%%%%%%%%%%%%%%%%%%%%%%%%%
%%%%%%%%%%%%%%%%%%%%%%%%%%%%%%%%%%%%%%%%%%%%%%%%%%%%%%%%%%%%
%%%%%%%%%%%%%%%%%%%%%%%%%%%%%%%%%%%%%%%%%%%%%%%%%%%%%%%%%%%%


\subsection*{Tor Common Locations}

\subsubsection*{Process Monitor WriteFile Operations}

Bei der Versuchsdurchführung für den Tor-Browser gemäß Kapitel X wurden drei Process Monitor Logfiles erstellt.
Diese Dateien enthalten alle aufgezeichneten Prozessaktivitäten während des Browsing Szenarios, nach dem Erzeugen einer "Neuen Identität" sowie nach Schließen des Browsers.
Eine gemäß Methodik in Kapitel X verarbeitete Tabelle mit allen in den Logfiles identifizierten Dateien ist im Anhang X aufgeführt.
Für jede Datei wurde vermerkt ob und wie sie wiederherstellbar war, mit welchem Tool die Datei analysiert wurde und ob PB Artefakte enthalten sind.

In Tabelle X (TODO!) sind die ausschließlich wiederherstellbaren Dateien aufgeführt.
Die Dateien wurden in die vier Kategorien "Cache", "datareporting", und "Sonstige Dateien" eingeordnet.
In keiner der identifizierten Dateien konnten PB Artefakte gefunden werden.

\begin{figure}[h!]
	\resizebox{\linewidth}{!}{\includegraphics{bilder/tor-tabelle-logfile1vlogfile2vlogfile3-reduced.png}}
%	\label{...}
	\caption{Tabelle mit wiederherstellbaren Dateien: Logfile 1 vs. Logfile 2}
\end{figure}

Ähnlich wie bei der Analyse der Schreiboperationen von Firefox in Kapitel X (TODO!), konnten für den Tor-Browser zwei Pfade identifiziert werden, wo sich Dateien befinden, in die geschrieben wurde.
\begin{itemize}
\item[\textbf{Caches}] \texttt{C:$\backslash$Users$\backslash$Forensik$\backslash$Desktop$\backslash$Tor Browser$\backslash$Browser$\backslash$TorBrowser$\backslash$Data$\backslash$Browser$\backslash$Caches$\backslash$profile.default$\backslash$}
\item[\textbf{Profile.default}] \texttt{C:$\backslash$Users$\backslash$Forensik$\backslash$Desktop$\backslash$Tor Browser$\backslash$Browser$\backslash$TorBrowser$\backslash$Data$\backslash$Browser$\backslash$profile.default$\backslash$}
\end{itemize}
In Tabelle X (TODO!) sind die Dateien je nach Speicherort "Caches" (Hellblau) oder "Profile.default" (Dunkelblau)  eingefärbt. 

Bei der Auswertung der Process Monitor Logfiles wurde festgestellt, dass alle Schreibopertationen von "firefox.exe" Prozessen durchgeführt wurde, nicht "tor.exe" 

Obwohl keine der Dateien PB Artefakte enthält, werden zum vollständigen Browserverständnis im Sinne der White-Box-Forensik die wichtigsten Dateien im Zusammenhang des Tor-Browsers genauer untersucht.


\paragraph*{Cache}
Der Tor-Browser schreibt eine einzige Cache-Datei \texttt{$\backslash$Caches$\backslash$profile.default$\backslash$startupCache$\backslash$startupCache.8.little} im Caches-Pfad. Alle anderen geschriebenen Dateien befinden sich im "Profile.default"-Pfad.
Die Datei "startupCache.8.little" ist eine interne Datei, welche erstellt wird, um den Startvorgang des Browsers zu beschleunigen. Sie enthält Informationen über bereits geladene Browser-Komponenten wie JavaScript-Code, CSS-Dateien, Bilder und andere Ressourcen.  %https://wiki.mozilla.org/StartupCache
Bei Untersuchung mit HxD konnten keine PB Artefakte gefunden werden.

\paragraph*{Datareporting}
Im "Datareporting"-Ordner wird vom Tor-Browser die Datei \texttt{$\backslash$datareporting$\backslash$data.safe.bin} geschrieben. Sie enthält verschlüsselte und anonyme "Glean" Informationen über die Nutzung des Browsers. % https://github.com/mozilla/glean
In HxD konnten keine keine PB Artefakte gefunden werden.
Weiterhin wird die Datei \texttt{$\backslash$datareporting$\backslash$state.json} geschrieben
Sie enthält Informationen über den Zustand und die Konfiguration des Tor-Browsers, beispielsweise installierte Add-Ons, oder Browser-Einstellungen. Sie wird verwendet, um dem Browser bei Bedarf den Zustand und die Einstellungen wiederherzustellen."
% https://github.com/mozilla/firefox-data-store-docs/blob/master/README.md
Eine Analyse mit Notepad++ und dem JSON-Plugin brachte keine PB-Artefakte.

\paragraph*{Sonstige Dateien}
Die im ersten Logfile geschriebene Datei \texttt{AlternateServices.txt}
enthält onion URLs des HTTP Alternative Services ist ein Mechanismus. Dieser ermöglicht es Servern, Clients mitzuteilen, dass der Dienst, auf den sie zugreifen, an einem anderen Netzwerkstandort oder über ein anderes Protokoll verfügbar ist. Die Datei speichert diese Zuordnung.
% https://support.mozilla.org/en-US/questions/1310302

Weiterhin wird währen des Browsing Szenarios die Datei \texttt{$\backslash$extensions$\backslash$staged$\backslash${73a6fe31-595d-460b-a920-fcc0f8843232}.xpi} geschrieben. Dabei handelt es sich um das von Tor verwendete "NoScript"-AddOn zur selektiven Ausführung von JavaScript Webseiteninhalten.
Nach Extraktion dieser Datei, kann diese per Drag-and-Drop in ein geöffnetes Firefox-Fenster gezogen werden und es ist möglich, die Erweiterung zu installieren.

Die geschriebene Datei \texttt{onion-aliases.json} enthält "SecureDrop" Adressen, beispielsweise für die Süddeutsche Zeitung.  %https://www.sueddeutsche.de/projekte/kontakt/#securedrop)
SecureDrop ist ein Open-Source-Softwaretool, das von Journalisten und Nachrichtenorganisationen verwendet wird, um anonyme Informationen von Whistleblowern entgegenzunehmen. Es ermöglicht den sicheren Austausch von Informationen, ohne die Identität der Quelle preiszugeben. Whistleblower können über .onion-URLs auf die SecureDrop-Websites zugreifen und vertrauliche Dokumente oder Nachrichten sicher und anonym übermitteln.

%	> % \security_state\data.safe.bin
%		Zweck: The file containing the updated security data % (https://bbs.archlinux.org/viewtopic.php?pid=1952286#p1952286)

Schließlich wurde in die Datei \texttt{SiteSecurityServiceState.txt} geschrieben.
Diese Datei speichert Daten wie Zertifikate, Verschlüsselungseinstellungen und andere Sicherheitsmerkmale, die von den besuchten Websites verwendet werden.
Es ist anzumerken, dass diese Datei früher private Browsing Artefakte enthielt % https://gitlab.torproject.org/tpo/applications/tor-browser/-/issues/18589
. In der akutellen Tor-Browser-Version konnten keine private Browsing Artefakte gefunden werden.
	
\subsubsection*{SQLite Datenbänke} 
Anhand der Process Monitor Logfiles ist erkennbar, dass Tor die gleichen SQLite Datenbanken wie Firefox aus Kapitel X (TODO!) verwaltet und beschreibt.
			
Wie bei der Analyse der SQLite-Datenbanken bei Firefox wird die Entwicklung von Dateiinhalt in allen fünf Festplatten-Images der Snapshots 1, 2, 3-1, 3-2 und 4 betrachtet. Die Ergebnisse sind in Tabelle X (TODO!) dargestellt.
\begin{figure}[h!]
	\centerline{\resizebox{\linewidth}{!}{\includegraphics{bilder/tor-sqlite-table.png}}}
	\label{chart:final-criteria}  
	\caption{Comparison of found PB artifacts between RAM Dumps}
\end{figure}
Nach Browser-Installation wurde noch keine SQLite-Datei angelegt (Snapshot 1).

Während des Browsing Szenarios wurden alle Datenbänkte initialisiert.
In places.sqlite wurden automatisch .onion URLs geschreiben, die zu Tor Standardseiten führen. Beispielsweise Seiten wie "The Tor Blog" oder "Tor Browser Manual".
Die gleichen Einträge wurden in der favicons.sqlite Datenbank geschrieben, mit dem Präfix "Fake-favicon-uri". Ein tatsächliches Icon wie bei Firefox in Kapitel X wurde nicht in die Datenbank geschrieben. 
Weiterhin erhielt die "remote settings" Datenbank den gleichen Eintrag wie es bereits bei Firefox der Fall war. Der Eintrag enthält keine PB Artefakte.
Die restliche SQLite-Dateien wurden ohne Inhalt angegelt, nur die Spaltennamen wurden definiert.
Nach Durchführung der WAL Checkpoints bleiben Dateien unverändert.

Nachdem dem Tor-Browser eine "Neue Identität" zugewiesen wurde (Snapshot 3-1), wurden in places.sqlite die Indizes bei den eingetragenen Seiten aktualisiert. Die restlichen Dateien blieben unverändert. Das Schreiben der WAL-Dateien in die Hauptdatenbanken veränderte den Inhalt nicht.

Nach Schließen des Browsers (Snapshot 3) wurden in places.sqlite sowie favicons.sqlite erneut Indizes bei eingetragenen Seiten aktualisiert. Die restliche Dateien blieben unverändert, ebenso ergaben die WAL Checkpoints keine Veränderungen.

Nach Herunterfahren der VM (Snapshot 4) blieben alle Datenbanken unverändert. Auch nach Durchführung der WAL Checkpoints gab es keine neuen Inhalte.

Im Balkendiagramm X (TODO!) ist zu erkennen, dass die meisten Schreiboperationen im ersten Logfile stattfinden. Dort werden Dateien jeder Kategorie beschrieben. Das Schließen des Tor-Browsers führt zu mehr oder genauso vielen Schreiboperationen wie das Zuweisen einer "Neuen Identität". Keine der geschriebenen Dateien enthielt private Browsing Artefakte.
*** TODO: Was noch? ***
	\begin{figure}[h!]
		\centerline{\resizebox{\linewidth}{!}{\includegraphics{bilder/tor-bar-chart-logfile1vs2cs3.png}}}
		\label{chart:final-criteria}  
		\caption{Comparison of found PB artifacts between RAM Dumps}
	\end{figure}

%Literatur:
%	o no traces were found in “common locations” \cite{Montasari.2015}
%		>  “places.sqlite”, “webappsstore. sqlite”, “sessionstore.bak”, “search.json” and “nssckbi.dll”
%	o	Safebrowsing: Alle Dateien in /safebrowsing-updating/ nicht relevant. Dort nur .vlpset und .sbstore Dateien. Speichern 256-Bit Hash von URLs, die auf SafeSearch Blacklist stehen 
%	o	Cache-Dateien: drei Caches: startupCache, jumpListCache (beide enthalten Binärdateien ohne Browsing Artefakte) und cache2 (können mit MozillaCacheView untersucht werden, enthalten keine Browsing Artefakte)
%	o	SQLite Datenbanken: Sqlite Dateien erst ohne WAL Dateien untersuchen, Danach mit sqlite3 Konsole: WAL in Datenbank schreiben mit: PRAGMA wal\_checkpoint; places.sqlite besonders relevant, da dort Browser in public Modus Browsing URLs verwaltet (Am besten hier vergleich mit Public Browsing machen)	
%		> \cite{Fayyad.2021} for Mozilla Firefox, 7 database files were recovered: cookies.sqlite-shm, places.sqlite-shm, prefs.js etc.
%		> \cite{Muir.2019} The two SQLite databases used by Firefox to track cookies and history (cookies.sqlite und places.sqlite) were both recoverable from the file system after deletion	
%		Ergebnisse stehen im Gegensatz zu \cite{Hedberg.2013} :
%			o	Chrome und Firefox: Einträge in places.sqlite + history.sqlite DB gefunden während PB! (Noch aktuell??)
%		Sonderfall: SQlite DB-Crash \cite{Hedberg.2013}
%			> WAL Files/Journal Files bei Crash gefunden -> Kann genutzt werden um zu beweisen, dass privater Browser genutzt wurde
%			> Daher: WAL Rollback mit sqlite3	
%	o	Jsonlz4 \& balkz4: Enthalten komprimierte Firefox-Sessions, jsonlz4 Dateien können mit Tool "entkomprimiert" werden: https://www.jeffersonscher.com/ffu/scrounger.html


\subsection*{Tor Uncommon Locations}

\subsubsection*{Analyse mit Autopsy}

Wie bei Firefox in Kapitel X wurden keine der kategorisierten Dateien gelöscht. Somit befanden sich im letzten Festplatten-Image des Snapshots 4 alle kategorisierten Dateien der vorherigen Festplatten-Images

\paragraph*{Web Bookmarks}
Wie in Abbildung X (TODO!) gezeigt, wurden nur in der Datei \texttt{Bing.url} ein Leesezeichen zur Bing-Startseite gefunden. Diese Datei wurde im Festplatten-Image des ersten Snapshots geschrieben.
\begin{figure}[h!]
	\centerline{\resizebox{\linewidth}{!}{\includegraphics{bilder/cfv_tor_autopsy_web_bookmarks.png}}}
	\label{chart:final-criteria}  
	\caption{Autopsy Web Bookmarks}
\end{figure}

\paragraph*{Web Cookies}
Im Festplatte-Image des ersten VM-Snapshots wurden neun Cookies-Einträge in die Datenbank von des vorinstallierten Edge Browsers geschrieben. Dabei handelt es sich um Cookies für die Bing- und Outlook-Startseite.
\begin{figure}[h!]
	\centerline{\resizebox{\linewidth}{!}{\includegraphics{bilder/cfv_tor_autopsy_web_cookies.png}}}
	\label{chart:final-criteria}  
	\caption{Autopsy Web Cookies}
\end{figure}

\paragraph*{Web History}
Zwei Einträge mit Browsingverläufen wurden im Festplatten-Image des ersten VM-Snapshots in der Datei \texttt{WebCacheV01.dat} geschrieben. Dabei handelt es sich zweimal um die Outlook-Startseite, obwohl diese nie bei der Versuchsdurchführung geöffnet wurde. Wie bei Firefox in Kapitel X (TODO!) wurden in der Datei ebenfalls die zum Analyserechner über den gemeinsamen Ordner transportierten Process Monitor Logfiles gespeichert.
\begin{figure}[h!]
	\centerline{\resizebox{\linewidth}{!}{\includegraphics{bilder/cfv_tor_autopsy_web_history.png}}}
	\label{chart:final-criteria}  
	\caption{Autopsy Web History}
\end{figure}

\paragraph*{Web Categories}
Autopsy klassifizierte im Festplatten-Image des ersten VM-Snapshots den Eintrag \texttt{bing.com} als "Suchmaschine"  und \texttt{live.com} als "Web-Email".
Es gab keine zusätzlichen Einträge in dieser Kategorie in den Festplatten-Images der restlichen Snapshots.
\begin{figure}[h!]
	\centerline{\resizebox{\linewidth}{!}{\includegraphics{bilder/cfv_tor_autopsy_web_categories.png}}}
	\label{chart:final-criteria}  
	\caption{Autopsy Web Categories}
\end{figure}
		
Somit wurden in den von Autopsy kategorisierten Dateien keine PB Artefakte entdeckt. Weiterhin gab es verglichen mit der Analyse der Common Locations keine neuen Erkenntnisse. Autopsy registrierte nicht die in der places.sqlite Datenbank geschriebenen .onion-URLs der Tor-Standardseiten.

%Literatur:
%	o	Autopsy Keywortsuche: 
%		>	In alles Snapshots ergebnislos (keine Keyword-Hits
%		-->	In Literatur: Autoren fanden Ergebnisse in pagefile.sys 
%			> Autopsy: websites and some of the keywords found in hidden file called “pagefile.sys” \cite{Mahlous.2020}
%			o \cite{Montasari.2015} traces were found in: 
%				> However, on investigating the “pagefile.sys”, some entries were discovered
%				> Using the “data carving” technique, profile picture was recovered
%			o \cite{Said.2011} 
%				> Examining pagefile.sys showed some positive hits 			
%		--> Evtl. hier zeigen, was gefunden werden kann, wenn RAM reduziert
%		--> Aber auf Problem hinweisen, dass gefundener String in pagefile nicht direkt Browser zugeordnet werden kann
%		> \cite{Gabet.2018}	Firefox only produced three recoverable artefacts as reported by both tools (FTK, Autopsy) --> Artefakte werden nicht genannt!
%		> \cite{Muir.2019} Autopsy Keyword Suche nach Suchbegriffen: unallocated space
%		> Autopsy Carving Module (\$Carved): \cite{Muir.2019}
%			•	When searching for the string ’clot’ from the browsing protocol, six .dll, .edb and .reg files were discovered in unallocated space.
%			•	Further searching of unallocated space uncovered references to the Tor installation directory and the obfs4 bridging IP addresses
%			•	browsing data found in NTUSER.DAT was also replicated in unallocated space.
%	o	Autopsy PlugIns:
%		>	*** TODO: Hier Liste mit PlugIns ***

\subsection*{Tor Registry}

\subsubsection*{Process Monitor SetValue Operations}

Bei Betrachtung als Common Locations werden gemäß Methodik in Kapitel X alle "SetValue" Schreiboperationen in den Process Monitor Logfiles für die Prozesse "tor.exe" und "firefox.exe" untersucht. 

Dabei wurden die gleichen beiden Registry Keys identifiziert, wie es bereits bei Untersuchung der Firefox Registry in Kapitel X (TODO!) der Fall war: \texttt{PreXULSkeletonUISettings} und \texttt{Business Activity Monitoring}. In keinem Registry-Key befinden sich PB Artefakte.
Wie in Abbildung X dargestellt, wurden beide Registry Keys annähernd gleich oft geschrieben. Bei Vergleich der drei Process Monitor Logfiles 1, 2 und 3 nimmt Anzahl der Registry "SetValue"-Operationen bei Logfile 2 und 3 kontinuierlich ab.
\begin{figure}[h!]
	\centerline{\resizebox{\linewidth}{!}{\includegraphics{bilder/tor-registry-stacked-bar-chart.png}}}
	\label{chart:final-criteria}  
	\caption{Comparison of found PB artifacts between RAM Dumps}
\end{figure}

\subsubsection*{Stringsuche in Registry Hives}
Bei Betrachtung der Registry als Uncommon Locations, wurden die in Tabelle X im Kapitel der Methodik X aufgelisteten Registry-Hives mithilfe des Registry Explorers untersucht. 
Weder in den System-Hives noch in den User-Hives konnte in keinem Festplatten-Image PB Artefakte identifiziert werden. 


%Literatur:
%	>	Auf Autor verweisen: angeblich in Shellactivities Ergebnisse. --> Nicht mehr vorhanden in aktueller Version (Verweis auf E-Mail)
%	>	Process Monitor/Regshot zeigen keine relevanten Key-Änderungen
%	> \cite{Muir.2019}: Autopsy Keyword Suche nach Suchbegriffen: Ergebnisse in \%SystemRoot\%Minidump NTUSER.DAT, ntuser.dat.LOG1 (a log of changes to NTUSER.DAT)
%	> Zentral: shellactivites Key:	NTUSER.DAT --> “shellactivities” key \cite{Muir.2019}
%	> \cite{Rochmadi.2017} Detection of registry changes helps to determine what the appropriate plugin is used to search for digital evidence using volatility memory forensic:
%	- RegQueryValue:	HKCU/Software/Microsoft/Windows/CurrentVersion/InternetSettings/Connections/DefaultConnectionSettings
%	- RegCloseValue: 	HKCU/Software/Microsoft/Windows/CurrentVersion/InternetSettings/Connections
%	- IRP\_MJ\_READ: C:/pagefile.sys

	
%		\addtocontents{toc}{\endgroup}
\end{appendices}
	
	% Literatur anzeigen
	\addcontentsline{toc}{chapter}{Literaturverzeichnis}
	\printbibliography
\end{document}