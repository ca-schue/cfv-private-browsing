\chapter{Stand der Technik}

\blindtext

Die Fahrzeugphysik wird umfangreich von \citeauthor{Mitschke.2014} in \citetitle{Mitschke.2014} zusammengefasst \cite{Mitschke.2014}. 
Vergleiche auch \cite{Romberg.2011} und Abbildung \ref{fig:logoTH}. 


\begin{figure}[h!]
  \centering
  \textbf{a}
  \includegraphics[width=0.4\linewidth]{thi_logo}
  \hfil
  \textbf{b}
  \includegraphics[width=0.4\linewidth]{efs_logo_black}
  \caption[Logos der Hochschule]{Überblick über die Logos der Technischen Hochschule Ingolstadt: \newline \textbf{a} Main Logo, \textbf{b} Fakultät EIT.}
  \label{fig:logoTH}
\end{figure}

\blindtext

Die Kraft $F$ auf einen Körper ist beschrieben durch die Masse $m$ und der wirksamen Beschleunigung $a$, sodass gilt
\begin{equation}
  F = m \, a.
\end{equation}
Dabei ist die Beschleunigung die zeitliche Veränderung der Geschwindigkeit $v$
\begin{equation}
  \begin{aligned}
    F & = m a, \\
      & = m \frac{\text{d} v}{\text{d} t}.\\
  \end{aligned}
  \label{eq:newton01}
\end{equation}
%
Mit Gleichung \eqref{eq:newton01} und $$ v = \dot{x} = 5{,}2 \, \frac{\text{m}}{\text{s}}, $$ wobei $x$ die Bewegung darstellt.

Soll das Maximum dargestellt werden, gilt
$$ F = F_\text{max} = m a_\text{max} = m \frac{\text{d}^2 x_\text{max}}{\text{d} t^2}.$$



















