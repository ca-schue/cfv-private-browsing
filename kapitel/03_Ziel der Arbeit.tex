\chapter{Ziel der Arbeit}
Die vorliegende Seminararbeit hat das Ziel, die Auswirkungen privater Browsingmodi auf potenziell hinterlassene Dateien einer Internetsitzung, die \textit{Browsing-Artefakte}, auf dem lokalen Rechner zu untersuchen. Konkret werden die Browser Firefox, Tor-Browser, Chrome und Brave analysiert, um festzustellen, welche dieser Browser die geringsten Spuren nach einer privaten Browsing-Sitzung hinterlassen.

Zentral für diese Arbeit ist eine \textit{transparente Versuchsdurchführung}. Dies umfasst sowohl die Kontaminierung als auch die Analyse der Browsing-Artefakte durch die gleichen Akteure. Dadurch ist bereits vor der Analyse bekannt, nach welchen spezifischen Browsing-Artefakten gesucht wird. Dies entspricht keinem realistischen forensischen Analyseszenario von Strafverfolgungsbehörden. Dort ist in der Regel nicht bekannt, welche Webseiten besucht wurden. Stattdessen wird meist nach verdächtigen Browsing-Artefakten gesucht.
Die transparente Versuchsdurchführung zielt darauf ab, das Verhalten des privaten Browsing-Modus umfassend zu analysieren und dabei alle potenziellen Artefakte zu identifizieren. Dies verbessert die Effizienz zukünftiger Untersuchungen und verhindert, dass wichtige Inhalte übersehen werden. \cite{Horsman.2019}

Das oberste Ziel dieser Arbeit besteht darin, gefundene Browsing-Artefakte eindeutig dem entsprechenden Browsing-Szenario oder Browser-Prozess zuzuordnen. 
Dies ist nötig da digitale Beweise bei Gerichtsverfahren eine Beweisauthentifizierung erfordern, wordurch der Beweis eindeutig einer Straftat zugeordnet werden muss.

Diese Arbeit grenzt sich von bestimmten Themengebieten ab, die nicht im Fokus der Untersuchung liegen. Diese Arbeit beschränkt sich auf den lokalen Angreifer, wie er in Kapitel X (TODO!) definiert wird und betrachtet nicht den Webangreifer. Eine Zuordnung gefundener Artefakte zu bestimmten Zeitstempeln wird nicht berücksichtigt. 
Weritehin werden keine Indikatoren untersucht, die anzeigen, ob und wann ein Browser gestartet, geschlossen oder im privaten Modus verwendet wurde.
Schließlich werden nicht die Auswirkungen von Browser-Erweiterungen auf die privaten Browsingmodi untersucht.

