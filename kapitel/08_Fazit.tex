\chapter{Fazit}

Da in den vorangegangenen Kapiteln ausführlich Begriffe definiert, die Methodik erläutert, die Ergebnisse präsentiert sowie vergleichen und weitergehende Punkte aufgeführt wurden, folgt abschließend für diese Arbeit ein kurzes persönliches Fazit. \\
Zunächst war es sehr informativ, über den allgemeinen Prozess bei einer forensischen Untersuchung mehr zu erfahren. Darunter zählt eine klar definierte Methodik, das Wissen über die verwendeten Tools sowie Schlussendlich die Durchführung der Analyse. Dabei war es stets wichtig, alles stets wie vorgeschrieben durchzuführen, um keine Ergebnisse in irgendeiner Art zu verfälschen. Auch die Verwendung von virtuellen Maschinen sowie Möglichkeiten, Speicherabbilder zu erstellen, war eine neue Erkenntnis. Ebenso wurden erstmals verschiedene Forensik-Tools wie Autopsy und Volatility verwendet sowie Möglichkeiten innerhalb dieser Tools benutzt, um Speicherabbilder mit unterschiedlichen Verfahren zu analysieren. Vor Allem beim Einlesen der Snapshots in Autopsy ist deutlich geworden, welcher zeitliche Aufwand hinter solch einer Analyse steckt, da selbst bei unserem Beispiel dieser Prozess ca. 3 Stunden andauerte, wenn man alle Plugins in Autopsy verwenden möchte. Außerdem konnte neues Wissen im Bezug auf die Funktionsweise von Browsern erlangt werden, hier speziell über das Abspeichern von Informationen in bestimmten Dateien in speziellen Dateipfaden und deren Bedeutungen. Auch verschiedene Analyse-Tools wie ein Hex-Editor oder Database-Viewer wurden dafür benötigt und es konnte ein tieferes Verständnis für verschiedene Datentypen sowie Kodierungen gewonnen werden. 

Abschließend war es eine sehr spannende und herausfordernde Arbeit, bei der man sehr viel über Browser, deren Funktionsweise und über Funktionen und Prozesse in Windows gelernt hat. 
