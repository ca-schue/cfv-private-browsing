\chapter{Fazit und Ausblick}\label{chap:Fazit-Ausblick}
\thispagestyle{plain.scrheadings}
\ohead{\headmark}

Im Rahmen dieser Arbeit wurden vier verschiedene Browser im privaten Modus auf hinterlassene Browsing-Artefakte untersucht. Ein transparenter Ansatz wurde verfolgt, bei dem ein Browsing-Szenario definiert und erwartete Artefakte festgelegt wurden. Die Aktivitäten der Browser wurden vor, während und nach dem Szenario aufgezeichnet sowie Festplatten- und Arbeitsspeicherabbilder wurden erstellt. Die Artefakte wurden an an gängigen sowie ungewöhnlichen Speicherorten, wie dem Arbeitsspeicher analysiert. Ein quantitativer Vergleich der Browser wurde durchgeführt, wobei der Tor-Browser die geringste Menge an Artefakten hinterließ. Qualitative Kriterien wie die Nutzererfahrung wurden nicht berücksichtigt.

Obwohl keine Artefakte in den Festplattenabbildern gefunden wurden, wurden im Arbeitsspeicher Spuren entdeckt. Der private Modus verhindert zwar das Speichern von Browser-Cache und -Verlauf, aber bei einer forensischen Analyse des laufenden Systems können dennoch eindeutig zuordenbare Artefakte gefunden werden.

Die identifizierten Schwachstellen stellen Browser-Entwickler vor einem Dilemma.
Von diesen Ergebnissen profitieren einerseits Browser-Nutzer, da Browser-Entwickler auf Basis der Ergebnisse die aufgedeckten Schwächen beseitigen können.
Allerdings besteht auch das Risiko des Missbrauchs des privaten Browsens für illegale Aktivitäten.
Das Schließen von Sicherheitslücken erschwert es somit Forensikern, kriminelle Aktivitäten aufzudecken und nachzuweisen. 
Andererseits profitieren Forensiker selbst von solchen Analysen, um bei identifizierten Schwachstellen gezielt nach Artefakten zu suchen. 

Zukünftige Arbeiten könnten den Umfang dieser Arbeit um folgende Punkte erweitern, welche aufgrund von Zeitbeschränkungen und begrenztem Umfang nicht möglich war. Statt nach Prozessnamen zu filtern, könnten Process Monitor Logfiles nach den bekannten Speicherorten durchsucht werden, um festzustellen, ob andere Prozesse dort Schreibaktivitäten ausführen. Eine detailliertere Untersuchung weiterer Betriebssystem-Speicherorte wie des DNS-Caches könnte vorgenommen werden. Es wäre wichtig, eine Analyse von Browsern für Mac oder Linux durchzuführen, da bisherige Literatur nur Windows betrachtet. Außerdem könnte der private Modus direkt mit dem normalen Browsing-Modus verglichen werden, um Unterschiede festzustellen. Schließlich wäre es interessant, weitere gängige Browser wie Microsoft Edge oder Safari einer forensischen Analyse zu unterziehen.

Diese Arbeit hat den persönlichen Wissensstand der Autoren erweitert, indem sie sich intensiv mit forensischer Analyse, Tools, Browsern, virtuellen Maschinen und dem Verhalten des Windows-Betriebssystems auseinandergesetzt haben. Die Arbeit mit Speicherabbildern verdeutlichte den umfangreichen und zeitintensiven Aufwand solcher Untersuchungen. Insgesamt war es eine spannende und anspruchsvolle Erfahrung.
%
%Diese Arbeit hat nicht nur zu neuen Erkenntnissen geführt, sondern auch den persönlichen Wissensstand der Autoren erweitert. Durch die intensive Auseinandersetzung mit forensischer Analyse, spezifischen Tools und relevanten Themen wie Browsern, virtuellen Maschinen und dem Verhalten des Windows-Betriebssystems konnten die Autoren ihr Fachwissen deutlich ausbauen. Die Arbeit mit Speicherabbildern und die komplexe Analyseprozesse haben gezeigt, welchen umfangreichen und zeitlichen Aufwand eine solche Untersuchung erfordert. Insgesamt war es eine faszinierende und anspruchsvolle Erfahrung.
%
%
%Neben neuen Erkenntnissen, wie dem Vorgehen einer forensischen Analyse und speziell dafür geeignete Tools, konnte bereits vorhandenes Wissen zu den Themen Browser, die Verwendung von virtuellen Maschinen oder dem Verhalten sowie wichtigen Dateien des Windows Betriebssystems erweitert werden. Speziell die Arbeit mit Speicherabbildern, wie Arbeitsspeicherabbildern oder ganzen Snapshots von virtuellen Maschinen, war bzgl. des Vorgehens und der Analyse neu und interessant, wie die Extraktion möglichst vieler Informationen wie aus einer Art Speicher-Blackbox abläuft. Dabei ist sehr deutlich geworden, welch großen und zeitlichen Aufwand sowohl hinter der Ergebung dieser Abbilder sowie deren Auswertung steckt. Zusammenfassend war es jedoch eine sehr spannende und herausfordernde Arbeit.
