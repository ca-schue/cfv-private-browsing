\chapter{Diskussion}

\begin{comment}

> Artefakte im DNS Cache: \cite{Satvat.2014}
	•	DNS-Caching ist eine Bedrohung für private Browsing
	•	Diese Schwachstelle entsteht, weil das Betriebssystem DNS-Anfragen des Browsers im Cache speichert, unabhängig davon, ob der Browser im privaten Modus ist oder nicht
	•	Mehrere Jahre nach der Meldung dieser Schwachstelle besteht sie immer noch in allen Browsern fort
	•	Es wurden einige Erweiterungen von Drittanbietern entwickelt, um dieses Problem zu beheben, aber keine davon wurde von den Browserherstellern übernommen.
	


> Viele RAM-Artefakte
	- Firefox \cite{Muir.2019}
		•	Darcie et al. (2014) fanden Beweise für das Web-Browsing in Form von JPEG- und HTML-Dateien in Live-Forensik, aber eine statische Forensik war erfolglos.
		•	Eine vorherige Live-Forensik-Analyse des Firefox-Browsers zeigte, dass Artefakte aus einer privaten Browsing-Sitzung aus dem Speicher wiederhergestellt werden konnten. (Findlay and Leimich, 2014). 
		

> IE hinterlässt viele Spuren im Gegensatz zu Ergebnissen: \cite{Md.2018}
	o	hidden folders are usually stored at C/Users/User/AppData
	o	evidence searches are conducted extensively in the C:\ partition
	o	bookmarks remain and can be viewed
	o	downloads remain in the downloads folder until the user manually deletes them
	o	CacheView trace entire URL and browsing histories including the temporary files
	CacheView enables to find the image’s URL and from specific website
	
> Urteil über die Privatheit von Tor nach \cite{Muir.2019}
	The design aim of preventing Tor from writing to disk (Perry et al., 2018) is not achieved in this version.
		•	Configuration files, downloaded files, and browserrelated data are recoverable from the file system.
		•	Significant data-leakage from the browsing session occurred: HTTP header information, titles of web pages and an instance of a URL were found in registry files, system files, and unallocated space.
		•	The data-leakage contained the German word for ’search’ in reference to a Google search. This hints at the locale of the Tor server used to exit the network (exit relay).
	The Tor Project’s design aim of enabling secure deletion of the browser (Sandvik, 2013) is not achieved in this version.
		•	References to: the installation directory, Firefox SQLite files, bridging IPs/ports, default bookmarks, Tor-related DLLs and Tor product information were all recovered after the browser was deleted.
		•	In a scenario where the operating system paged memory, an instance

Weiterführende Arbeiten:
> Cross-mode interference \cite{Hedberg.2013}:
	o	the Chrome://memory page displays all the opened tabs in the browser regardless if they are in the usual or private mode -> Nicht mehr aktuell -> Stattdessen: Chrome Task-manager (Ctrl + Esc), Funktioniert auch bei Firefox
> Unser Scope: Process Monitor nach Prozessnamen gefiltert
	- Weiterführend: Nach Pathnamen filtern: "Common Locations"

> Für wen wird Browser entwickelt
> Warum und für wen wird Private Browsing analysiert?
> Ist das Auffinden privater Browsing Artefakte Schuld von Browser Entwicklern? (Oder Schuld des Betriebssystem, wie in (TODO!) erwähnt)
\end{comment}

Zusammenfassend kann die Aussage getroffen werden, dass alle der vier betrachteten Browser ein gutes Ergebnis lieferten, da keine Browsing-Artefakte im nichtvolatilen, also persistenten Speicher identifiziert werden konnten. Jene konnten nur im RAM gefunden werden und teilweise im DNS-Cache, wobei diese auch durch das Leeren des Caches erfolgreich beseitigt worden konnten. Artefakte im DNS-Cache sind dabei eine bekannte Problematik. DNS-Anfragen des Browsers werden dabei von Betriebssystem darin gespeichert, unabhängig davon, ob der Browser im privaten Modus ist oder nicht \cite{Satvat.2014}. Trotz der Bekanntheit dieser Schwachstelle besteht dieser immer noch in allen Browsern fort. Es wurden gezielt dafür auch schon Erweiterungen von Drittanbietern entwickelt, um dieses Problem des Speicherns von DNS-Anfragen zu cachen, jedoch wurde keine davon von den Browserherstellern übernommen \cite{Satvat.2014}. 

Unserer Meinung nach ist der beste Browser, um anonym und sicher zu surfen, der Tor-Browser. Wie bereits im vorherigen Kapitel gezeigt, hinterlässt er am wenigsten Artefakte im RAM und somit am wenigsten Artefakte überhaupt, wie bereits öfter angesprochen wurde. Zusätzlich zum Schutz vor einem local attacker bietet Tor den Vorteil, dass man dadurch auch noch vor web attackern geschützt ist, da der Datenverkehr über mehrere Router, den sogenannten Nodes, umleitet und somit die Herkunft der Anfrage verschleiert. Somit ist Tor als Gesamtpaket gesehen ein sicherer und guter Browser, um sicher und anonym im Internet zu browsen. \\
Trotz der gerade angesprochenen Vorteile muss man jedoch auch die Nachteile von Tor sehen. Erstens ist das Browsen an sich meistens langsamer, da der Datenverkehr eben über mehrere Knoten geleitet wird. Zusätzlich wird dabei der Internetverkehr auch noch verschlüsselt. Beides ist positiv für die Privatsphäre, schränkt aber die User Experience ein [cite webpage]. Zusätzlich dazu kann es dazu kommen, dass Websiten den Datenverkehr von Tor-Benutzern als potenzielles Sicherheitsrisiko ansehen und somit zusätzliche Captchas und Sicherheitsmaßnahmen einfügen, was wieder die User Experience einschränkt. Dazu kommt auch noch, dass man von der Infrastruktur des Tor-Netzwerkes angewiesen ist und es zu Verbindungsproblemen führen kann, wenn das Netzwerk gestört ist oder bestimmte nodes nicht verfügbar sind. 

Schließlich ist noch die Frage zu klären, für welche Personen diese Analyse grundsätzlich durchgeführt wird bzw. wer von solch einer forensischen Analyse einen Nutzen zieht. Grundsätzlich wird eine solche forensische Analyse durchgeführt, um Schwächen in Browsern und den privaten Modi aufzudecken. Von diesen Ergebnissen profitieren zunächst einmal die Browser-Entwickler, welche auf Basis der Ergebnisse das Ziel haben (sollten), die aufgedeckten Schwächen zu beseitigen, um ein privateres und sichereres Browsen zu ermöglichen. Dieses private Browsen kann dann entweder genutzt werden, um von „normalen“ Benutzern für legale Aktivitäten verwendet zu werden, oder von Kriminellen für illegale Tätigkeiten. Für diese wäre es dann sinnvoll, wenn der Browser möglichst wenige Artefakte verursacht, um möglichst unerkannt diesen Tätigkeiten nachzugehen. Dies würde es jedoch Forensikern wieder schwieriger machen, jene Tätigkeiten aufzudecken und nachzuweisen. Auch profitieren von solchen Analysen die Forensiker, welche auf Basis der Ergebnisse gezielt nach Artefakten suchen können und es somit leichter gemacht wird, kriminelle Tätigkeiten aufzudecken. \\
Wie daraus ersichtlich wird, ist es nicht einfach zu sagen, wem eine solche forensische Analyse zunächst mehr nutzt und ob es gut oder schlecht ist, eine solche zu betreiben, da man nicht vorhersagen kann, ob die Ergebnisse zuerst gutartig zum besseren Aufdecken von illegalen Aktivitäten oder bösartig zum besseren Verschleiern von diesen verwendet wird. 

Da jetzt alle Ergebnisse ausführlich ausgeführt wurden und noch weitergehende Fragestellungen angesprochen wurden, ist es nun noch wichtig anzusprechen, was man anschließend an diese Arbeit noch analysieren könnte, um die Browser-Forensik im Allgemeinen noch genauerer zu betrachten bzw. um noch 
Zum einen kann der scope, welcher für diese Arbeit getroffen wurde, erweitert werden. Hier wurde beim Process Monitor nur nach Prozessnamen, also nach Browser gefiltert. Eine weitere Möglichkeit wäre es, nach Speicherorten zu filtern, wie beispielsweise den browser-typischen common locations, und dann zu analysieren, welche Prozesse eine Schreibaktivität darin durchführen. Auch könnten noch weitere Aktivitäten des Betriebssystems näher betrachtet werden wie der DNS-Cache, ob es noch weitere Ort gibt, an welchem Windows hier noch Dateien anlegt oder bearbeitet, welche nicht von Browser bearbeitet werden. Auch eine Analyse der Browser für Mac oder Linux wäre wichtige Aufgabe, da sich die gesamte Literatur nur hauptsächlich mit Windows als Betriebssystem beschäftigt. Zusätzlich könnte man mehr auf den web attacker eingehen, was in dieser Arbeit komplett ausgeschlossen wurde. Ebenfalls könnten man einen direkten Vergleich mit den Aktivitäten der public-Modi ziehen, was aus zeitlichen Gründen und wegen des beschränkten Umfangs hier nicht mehr durchführbar war und den Rahmen diese Arbeit gesprengt hätte. Zuletzt wäre es auch noch möglich, andere Browser wie Edge oder Safari dieser forensischen Analyse zu unterziehen.


