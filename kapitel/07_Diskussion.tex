\chapter{Zusammenfassung und Diskussion}\label{chap:Zusammenfassung-Diskussion}

\begin{comment}

> Artefakte im DNS Cache: \cite{Satvat.2014}
	•	DNS-Caching ist eine Bedrohung für private Browsing
	•	Diese Schwachstelle entsteht, weil das Betriebssystem DNS-Anfragen des Browsers im Cache speichert, unabhängig davon, ob der Browser im privaten Modus ist oder nicht
	•	Mehrere Jahre nach der Meldung dieser Schwachstelle besteht sie immer noch in allen Browsern fort
	•	Es wurden einige Erweiterungen von Drittanbietern entwickelt, um dieses Problem zu beheben, aber keine davon wurde von den Browserherstellern übernommen.
	


> Viele RAM-Artefakte
	- Firefox \cite{Muir.2019}
		•	Darcie et al. (2014) fanden Beweise für das Web-Browsing in Form von JPEG- und HTML-Dateien in Live-Forensik, aber eine statische Forensik war erfolglos.
		•	Eine vorherige Live-Forensik-Analyse des Firefox-Browsers zeigte, dass Artefakte aus einer privaten Browsing-Sitzung aus dem Speicher wiederhergestellt werden konnten. (Findlay and Leimich, 2014). 
		

> IE hinterlässt viele Spuren im Gegensatz zu Ergebnissen: \cite{Md.2018}
	o	hidden folders are usually stored at C/Users/User/AppData
	o	evidence searches are conducted extensively in the C:\ partition
	o	bookmarks remain and can be viewed
	o	downloads remain in the downloads folder until the user manually deletes them
	o	CacheView trace entire URL and browsing histories including the temporary files
	CacheView enables to find the image’s URL and from specific website
	
> Urteil über die Privatheit von Tor nach \cite{Muir.2019}
	The design aim of preventing Tor from writing to disk (Perry et al., 2018) is not achieved in this version.
		•	Configuration files, downloaded files, and browserrelated data are recoverable from the file system.
		•	Significant data-leakage from the browsing session occurred: HTTP header information, titles of web pages and an instance of a URL were found in registry files, system files, and unallocated space.
		•	The data-leakage contained the German word for ’search’ in reference to a Google search. This hints at the locale of the Tor server used to exit the network (exit relay).
	The Tor Project’s design aim of enabling secure deletion of the browser (Sandvik, 2013) is not achieved in this version.
		•	References to: the installation directory, Firefox SQLite files, bridging IPs/ports, default bookmarks, Tor-related DLLs and Tor product information were all recovered after the browser was deleted.
		•	In a scenario where the operating system paged memory, an instance

Weiterführende Arbeiten:
> Cross-mode interference \cite{Hedberg.2013}:
	o	the Chrome://memory page displays all the opened tabs in the browser regardless if they are in the usual or private mode -> Nicht mehr aktuell -> Stattdessen: Chrome Task-manager (Ctrl + Esc), Funktioniert auch bei Firefox
> Unser Scope: Process Monitor nach Prozessnamen gefiltert
	- Weiterführend: Nach Pathnamen filtern: "Common Locations"

> Für wen wird Browser entwickelt
> Warum und für wen wird Private Browsing analysiert?
> Ist das Auffinden privater Browsing-Artefakte Schuld von Browser Entwicklern? (Oder Schuld des Betriebssystem, wie in (TODO!) erwähnt)
\end{comment}

Zusammenfassend liefert jeder der vier betrachteten Browser ein - für den Nutzer des privaten Modus - gutes Ergebnis, da keine Private-Browsing-Artefakte in den Festplatten-Abbildern identifiziert werden konnten. 
Ausschließlich im Arbeitsspeicher konnten PB-Artefakte gefunden werden.

Solange der Browser noch mit allen besuchten Seiten geöffnet ist, sind in einem realistischen Szenario keine forensischen Werkzeuge nötig, um den Besuch dieser Seiten nachzuweisen. Somit ist für Forensiker insbesondere der RAM-Dump nach dem Schließen des Browsers von besonderer Relevanz. 
Wie in Kapitel \ref{chapter:vergleich-der-browser} gezeigt, wurden dort kaum PB-Artefakte gefunden.
Teilweise konnten die zu diesem Zeitpunkt gefundenen Artefakte beseitigt werden.
Beispielsweise wurden bei Firefox aufgerufene URLs im DNS-Cache gefunden. DNS-Anfragen werden vom Betriebssystem gespeichert, unabhängig davon, ob der private Modus aktiviert ist oder nicht \cite{Satvat.2014}. Diese Einträge können per Kommandozeilenbefehl geleert werden. Es existieren bereits Browser-Erweiterungen, um das Speichern von DNS-Anfragen zu verhindern \cite{Satvat.2014}.

Wie in \autoref{chapter:vergleich-der-browser} übersichtlich dargelegt wurde, hinterlässt der Tor-Browser am wenigsten Artefakte auf dem lokalen Computer. 
Zusätzlich zum Schutz vor einem Local Attacker bietet Tor den Vorteil, dass der Datenverkehr durch das Tor-Netzwerks vor dem Web Attacker geschützt ist. 

Hier muss jedoch beachtet werden, dass Tor in dieser Arbeit nur quantitativ gesehen der beste Browser ist.
Bei der Ermittlung des besten Browsers müssen stets individuelle qualitative Präferenzen berücksichtigt werden. 
Beispielsweise hinterlässt der Tor-Browser zwar weniger E-Mail-Artefakte, speichert jedoch im Gegensatz zu Google Chrome und Brave das Passwort als Klartext im RAM. Umgekehrt hinterlässt der Tor-Browser keine HTML-Artefakte.
Ist für einen Nutzer das Auffinden des Passworts schwerwiegender als hinterlassene HTML-Artefakte, wären Chrome und Brave \glqq{}besser\grqq{} beziehungsweise \glqq{}sicherer\grqq{}.

Weiterhin unterscheidet sich die Nutzererfahrung der Browser deutlich.
Zwar hinterlässt der Tor-Browser am wenigsten PB-Artefakte, ist jedoch aufgrund des Tor-Netzwerks oft langsam und erfordert häufig das Lösen von Captchas, was sich negativ auf die Nutzererfahrung auswirkt.
%Weiterhin ist der Nutzer des Tor-Browsers auf die Infrastruktur des Tor-Netzwerkes angewiesen. Bei einer Netzstörung - beispielsweise beim Ausfall bestimmter Tor-Knoten - kommt es zu Verbindungsproblemen. 
Im Gegensatz dazu legt beispielsweise der Bave-Browser besonderen Wert auf die Nutzerfreundlichkeit, beispielsweise durch einen integrierten Ad-Blocker. Weiterhin kann entschieden werden, ob der private Modus mit oder ohne Verbindung zum Tor-Netzwerk aktiviert werden soll. \cite{Brave.}

%Somit ist Tor als Gesamtpaket gesehen ein sicherer und guter Browser, um sicher und anonym im Internet zu browsen. \\
%
%\thispagestyle{plain.scrheadings}
%\ohead{}
%
%Grundsätzlich wird eine solche forensische Analyse durchgeführt, um Schwächen in Browsern und den privaten Modi aufzudecken. 
%Beim Umgang mit identifizierten Schwachstellen stehen jedoch Browser-Entwickler vor einem Dilemma.
%Von diesen Ergebnissen profitieren Browser-Nutzern, wie freien Journalisten in autokratischen Staaten mit Angst vor Repressionen, da Browser-Entwickler auf Basis der Ergebnisse die aufgedeckten Schwächen beseitigen können.
%Wie zu Beginn dieser Arbeit erwähnt, kann das private Browsen jedoch auch von kriminellen Nutzern für illegale Tätigkeiten missbraucht werden. 
%Somit erschwert das Schließen der Sicherheitslücken Forensikern, kriminelle Tätigkeiten aufzudecken und nachzuweisen.
%Andererseits profitieren Forensiker von solchen Analysen, um bei identifizierten Schwachstellen gezielt nach Artefakten suchen können. 



