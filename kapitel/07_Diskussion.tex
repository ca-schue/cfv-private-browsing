\chapter{Diskussion}

> Artefakte im DNS Cache: \cite{Satvat.2014}
	•	DNS-Caching ist eine Bedrohung für private Browsing
	•	Diese Schwachstelle entsteht, weil das Betriebssystem DNS-Anfragen des Browsers im Cache speichert, unabhängig davon, ob der Browser im privaten Modus ist oder nicht
	•	Mehrere Jahre nach der Meldung dieser Schwachstelle besteht sie immer noch in allen Browsern fort
	•	Es wurden einige Erweiterungen von Drittanbietern entwickelt, um dieses Problem zu beheben, aber keine davon wurde von den Browserherstellern übernommen.
	


> Viele RAM-Artefakte
	- Firefox \cite{Muir.2019}
		•	Darcie et al. (2014) fanden Beweise für das Web-Browsing in Form von JPEG- und HTML-Dateien in Live-Forensik, aber eine statische Forensik war erfolglos.
		•	Eine vorherige Live-Forensik-Analyse des Firefox-Browsers zeigte, dass Artefakte aus einer privaten Browsing-Sitzung aus dem Speicher wiederhergestellt werden konnten. (Findlay and Leimich, 2014). 
		

> IE hinterlässt viele Spuren im Gegensatz zu Ergebnissen: \cite{Md.2018}
	o	hidden folders are usually stored at C/Users/User/AppData
	o	evidence searches are conducted extensively in the C:\ partition
	o	bookmarks remain and can be viewed
	o	downloads remain in the downloads folder until the user manually deletes them
	o	CacheView trace entire URL and browsing histories including the temporary files
	CacheView enables to find the image’s URL and from specific website
	
> Urteil über die Privatheit von Tor nach \cite{Muir.2019}
	The design aim of preventing Tor from writing to disk (Perry et al., 2018) is not achieved in this version.
		•	Configuration files, downloaded files, and browserrelated data are recoverable from the file system.
		•	Significant data-leakage from the browsing session occurred: HTTP header information, titles of web pages and an instance of a URL were found in registry files, system files, and unallocated space.
		•	The data-leakage contained the German word for ’search’ in reference to a Google search. This hints at the locale of the Tor server used to exit the network (exit relay).
	The Tor Project’s design aim of enabling secure deletion of the browser (Sandvik, 2013) is not achieved in this version.
		•	References to: the installation directory, Firefox SQLite files, bridging IPs/ports, default bookmarks, Tor-related DLLs and Tor product information were all recovered after the browser was deleted.
		•	In a scenario where the operating system paged memory, an instance

Weiterführende Arbeiten:
> Cross-mode interference \cite{Hedberg.2013}:
	o	the Chrome://memory page displays all the opened tabs in the browser regardless if they are in the usual or private mode -> Nicht mehr aktuell -> Stattdessen: Chrome Task-manager (Ctrl + Esc), Funktioniert auch bei Firefox
> Unser Scope: Process Monitor nach Prozessnamen gefiltert
	- Weiterführend: Nach Pathnamen filtern: "Common Locations"

> Für wen wird Browser entwickelt
> Warum und für wen wird Private Browsing analysiert?
> Ist das Auffinden privater Browsing Artefakte Schuld von Browser Entwicklern? (Oder Schuld des Betriebssystem, wie in (TODO!) erwähnt)



