\chapter{Methodik}

Warum Methodik? 
	> \cite{Izzati.2022}
		•	Die Verfahren für die digitale Forensik für Browser-Forensik müssen angemessen befolgt werden, um dem Ermittler bei der Durchführung der Untersuchung zu helfen. Die Verfahren unterscheiden sich je nachdem, wie die Untersuchung durchgeführt werden soll.
	> \cite{Horsman.2019}
		•	Das Fehlen von Klarheit hat einen signifikanten Einfluss auf forensische Untersuchungen von Strafverfolgungsbehörden und deren Ansätze
		•	Eine Kette von Beweisführung muss dokumentiert werden, um die Integrität und Zuverlässigkeit der Daten sicherzustellen.
		•	Ein formaler forensischer Bericht wird dann vor Gericht präsentiert.
	



Phasen nach \cite{Izzati.2022}:
	•	Es gibt verschiedene Modelle für digitale Forensik, die sich in ihren Phasen unterscheiden können.
	•	Fünf Phasen sind besonders wichtig: Identifikation und Sammlung, Bewahrung, Erwerb, Analyse und Prüfung sowie Dokumentation.
	•	In der Identifikations- und Sammelphase werden alle potenziellen Beweismittel identifiziert, gekennzeichnet und gesammelt, um sie in der nächsten Phase zu verwenden.
	•	Beweismittel können z.B. Log-Dateien, temporäre Dateien, Netzwerkverbindungen, Browserverlauf und Cache sein.
	> Phasen:
		Preparation Phase
		o	Versuchsplanung + Konfiguration der HW/SW + Durchführen des Experiments Acquisition Stage
		o	Abbildung von der Festplatte (Static Forensics) und des RAMs (Live Forensics) Analysephase
		o	Bilder der Speicherabbilder mit einem forensischen Tool untersuchen	Validierungsphase
		o	gefundenen Artefakte verglichen und dokumentiert



\section{Preparation Stage}
\Blindtext[1][3]

\subsection{Vorbereitung der VMs}

\subsubsection*{Browserauswahl}

> Browserstudie \cite{Izzati.2022}
	- Die Herstellerangaben unterschiedlicher Browser bzgl. Privatheit untersucht
	- Firefox 58.02: No Browsing History stored, No Cookies stored, No login Info stored, Tracking Protection Enabled: Disconnect, Download Files not Hidden
	- Chrome 63.0.3239: No Browsing History stored, No Cookies stored, No login Info stored, Tracking Protection Enabled: No, Download Files not Hidden

> design aim of Tor: \cite{Muir.2019}
	- preventing from writing to disk (Perry et al., 2018) 
	- enabling secure deletion of the browser (Sandvik, 2013) (hier nicht relevant)


\subsection{Vorbereitung des Analyserechners}

\subsubsection*{Volatility}
Plugins-Liste: \cite{Dayalamurthy.2013}

\subsection{Browsing Szenario}


\section{Acquisition Stage}
- Warum Process Monitor während Browsing?
	o Während Browsing Szenario Filechanges untersuchen: DaemonFS set to monitor all activity within local hard drive\cite{Ohana.2013}
	
- Registry: \cite{Rochmadi.2017}
	•	Das Windows-Registrierungsverzeichnis enthält viele Informationen zur Nutzung des Computers, Benutzerkonfigurationen, Anwendungen und Hardwaregeräte
	•	Informationen im Registrierungsverzeichnis werden nach Ausführungsreihenfolge, Suchschlüsselwörtern, zuletzt aufgerufenen Ordnern, Anwendungsprotokollen und anderen Kategorien sortiert.
	
	

\section{Analysis und Validation Stage}
\Blindtext[1][4]
