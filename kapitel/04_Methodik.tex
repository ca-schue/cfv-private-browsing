\chapter{Methodik}

Warum Methodik? 
	> \cite{Aggarwal.2010}
		Aufgrund der Komplexität moderner Browser ist eine systematische Methode erforderlich, um zu testen, ob der private Browsing-Modus ausreichend gegen die Bedrohungsmodelle aus Abschnitt 2 verteidigt.		
	> \cite{Izzati.2022}
		•	Die Verfahren für die digitale Forensik für Browser-Forensik müssen angemessen befolgt werden, um dem Ermittler bei der Durchführung der Untersuchung zu helfen. Die Verfahren unterscheiden sich je nachdem, wie die Untersuchung durchgeführt werden soll.
	> \cite{Horsman.2019}
		•	Das Fehlen von Klarheit hat einen signifikanten Einfluss auf forensische Untersuchungen von Strafverfolgungsbehörden und deren Ansätze
		•	Eine Kette von Beweisführung muss dokumentiert werden, um die Integrität und Zuverlässigkeit der Daten sicherzustellen.
		•	Ein formaler forensischer Bericht wird dann vor Gericht präsentiert.
	
	
Bekanntes Computer Forensik Vorgehensmodell: \cite{Yusoff.2011}: Generic Model Computer Forensics Investigations (GCFIM) -> Daran orientieren sich alle in der Literatur


Phasen nach \cite{Montasari.2015}
	•	Die forensische Analyse erfolgt in zwei Phasen.
	1.	Zunächst wird die Analyse an sowohl "üblichen" als auch "ungewöhnlichen" Speicherorten auf der Festplatte durchgeführt.
	2.	In der zweiten Phase wird der physische Arbeitsspeicher (RAM) untersucht.


Phasen nach \cite{Izzati.2022}:
	•	Es gibt verschiedene Modelle für digitale Forensik, die sich in ihren Phasen unterscheiden können.
	•	Fünf Phasen sind besonders wichtig: Identifikation und Sammlung, Bewahrung, Erwerb, Analyse und Prüfung sowie Dokumentation.
	•	In der Identifikations- und Sammelphase werden alle potenziellen Beweismittel identifiziert, gekennzeichnet und gesammelt, um sie in der nächsten Phase zu verwenden.
	•	Beweismittel können z.B. Log-Dateien, temporäre Dateien, Netzwerkverbindungen, Browserverlauf und Cache sein.
	> Phasen:
		Preparation Phase
		o	Versuchsplanung + Konfiguration der HW/SW + Durchführen des Experiments Acquisition Stage
		o	Abbildung von der Festplatte (Static Forensics) und des RAMs (Live Forensics) Analysephase
		o	Bilder der Speicherabbilder mit einem forensischen Tool untersuchen	Validierungsphase
		o	gefundenen Artefakte verglichen und dokumentiert



\section{Preparation Stage}
\Blindtext[1][3]

\subsection{Vorbereitung der VMs}

\subsubsection*{Konfiguration}
RAM:
	- Kaum Angaben in der Literatur:
		> \cite{Rochmadi.2017}: 2 GB 
		> \cite{Ohana.2013}: 4 GB
	- Hier: 6 GB 
	-> Ausblick: Kritik an Literatur, dass RAM-Größe kaum thematisiert wird, obwohl sie Auswirkungen auf Ergebnisse hat -> Siehe Kapitel X (TODO!)

Netzwerkeinstellungen: 
	- TODO

Windows 10 Installation:
	- TODO

\subsubsection*{Tools auf VM}
	- Process Monitor
	- Regshot

\subsubsection*{Browserauswahl}

> Browserstudie \cite{Izzati.2022}
	- Die Herstellerangaben unterschiedlicher Browser bzgl. Privatheit untersucht
	- Firefox 58.02: No Browsing History stored, No Cookies stored, No login Info stored, Tracking Protection Enabled: Disconnect, Download Files not Hidden
	- Chrome 63.0.3239: No Browsing History stored, No Cookies stored, No login Info stored, Tracking Protection Enabled: No, Download Files not Hidden

> design aim of Tor: \cite{Muir.2019}
	- preventing from writing to disk (Perry et al., 2018) 
	- enabling secure deletion of the browser (Sandvik, 2013) (hier nicht relevant)


\subsection{Vorbereitung des Analyserechners}

\subsubsection*{Volatility}
Plugins-Liste: (Ähnlich zu \cite{Hariharan} und \cite{Dayalamurthy.2013})
	- TODO
	
\subsubsection*{Autopsy}
Evtl. hier Sleuthkit vs Autopsy thematisieren
	- TODO

\subsubsection*{Sonstige Tools}
WinHex SQLite Viewer etc.
	- TODO

-> Evtl. am Schluss Tabelle mit allen Tool, Versionen und Plug-Ins


\subsection{Browsing Szenario}
- Wichtig für White-Box-Ansatz: Browsing Szenario ist bekannt
- URL X … 	(TODO!)



\section{Acquisition Stage}

> Browsing Szenario durchfüren
> Zeitpunkte von -> Orientieren an: \cite{Muir.2019}
	- RAM-Dumps 
	- VM-Snapshots (nur letzter Snapshot ist Post-Mortem Forensik)
	- Process Monitor Logfiles
	- Registry Snapshots
	

- Warum Process Monitor während Browsing?
	o Während Browsing Szenario Filechanges untersuchen: DaemonFS set to monitor all activity within local hard drive\cite{Ohana.2013}
	
- Registry: \cite{Rochmadi.2017}
	•	Das Windows-Registrierungsverzeichnis enthält viele Informationen zur Nutzung des Computers, Benutzerkonfigurationen, Anwendungen und Hardwaregeräte
	•	Informationen im Registrierungsverzeichnis werden nach Ausführungsreihenfolge, Suchschlüsselwörtern, zuletzt aufgerufenen Ordnern, Anwendungsprotokollen und anderen Kategorien sortiert.
	
	

\section{Analysis und Validation Stage}

> Analysis Stage (= Kapitel „Results“)
	- Analyse der akquirierten Artefakte der vorherigen Phase: VM-Snapshots, RAM-Dumps, Process Monitor Logfiles und Registry Snapshots mit ggf. zusätzlichen Tools
	- Oberster Leitsatz dabei: gefundenes Artefakt muss eindeutig Browser zugeordnet werden können: Deshalb einfache Stringsuche in RAM mit WinHex ungenügend -> Hier evtl. negatives Beispiel zu Stringsuche einflechten

> Validation Stage (= Kapitel „Vergleich der Browser“)


