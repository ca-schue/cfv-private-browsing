\chapter{Methodik}

> Validation Stage (= Kapitel „Vergleich der Browser“)

Warum Methodik? 
	> \cite{Aggarwal.2010}
		Aufgrund der Komplexität moderner Browser ist eine systematische Methode erforderlich, um zu testen, ob der private Browsing-Modus ausreichend gegen die Bedrohungsmodelle aus Abschnitt 2 verteidigt.		
	> \cite{Izzati.2022}
		•	Die Verfahren für die digitale Forensik für Browser-Forensik müssen angemessen befolgt werden, um dem Ermittler bei der Durchführung der Untersuchung zu helfen. Die Verfahren unterscheiden sich je nachdem, wie die Untersuchung durchgeführt werden soll.
	> \cite{Horsman.2019}
		•	Das Fehlen von Klarheit hat einen signifikanten Einfluss auf forensische Untersuchungen von Strafverfolgungsbehörden und deren Ansätze
		•	Eine Kette von Beweisführung muss dokumentiert werden, um die Integrität und Zuverlässigkeit der Daten sicherzustellen.
		•	Ein formaler forensischer Bericht wird dann vor Gericht präsentiert.
	
	
Bekanntes Computer Forensik Vorgehensmodell: \cite{Yusoff.2011}: Generic Model Computer Forensics Investigations (GCFIM) -> Daran orientieren sich alle in der Literatur

BSI: "abschnittsbasierter Verlauf einer forensischen Untersuchung" % https://www.bsi.bund.de/SharedDocs/Downloads/DE/BSI/Cyber-Sicherheit/Themen/Leitfaden_IT-Forensik.pdf?__blob=publicationFile&v=2
- Vorbereitung;
- Datensammlung;
- Untersuchung: Extraktion von Bilddateien aus dem Image
- Datenanalyse: Verbindungen zwischen mehreren Daten herstellen
- Dokumentation

Phasen nach \cite{Montasari.2015}
	•	Die forensische Analyse erfolgt in zwei Phasen.
	1.	Zunächst wird die Analyse an sowohl "üblichen" als auch "ungewöhnlichen" Speicherorten auf der Festplatte durchgeführt.
	2.	In der zweiten Phase wird der physische Arbeitsspeicher (RAM) untersucht.

Phasen nach \cite{Izzati.2022}:
	•	Es gibt verschiedene Modelle für digitale Forensik, die sich in ihren Phasen unterscheiden können.
	•	Fünf Phasen sind besonders wichtig: Identifikation und Sammlung, Bewahrung, Erwerb, Analyse und Prüfung sowie Dokumentation.
	•	In der Identifikations- und Sammelphase werden alle potenziellen Beweismittel identifiziert, gekennzeichnet und gesammelt, um sie in der nächsten Phase zu verwenden.
	•	Beweismittel können z.B. Log-Dateien, temporäre Dateien, Netzwerkverbindungen, Browserverlauf und Cache sein.
	> Phasen:
		Preparation Phase
		o	Versuchsplanung + Konfiguration der HW/SW + Durchführen des Experiments Acquisition Stage
		o	Abbildung von der Festplatte (Static Forensics) und des RAMs (Live Forensics) Analysephase
		o	Bilder der Speicherabbilder mit einem forensischen Tool untersuchen	Validierungsphase
		o	gefundenen Artefakte verglichen und dokumentiert

\section{Vorbereitung}

- \cite{Izzati.2022} Versuchsplanung + Konfiguration der HW/SW + Durchführen des Experiments Acquisition Stage
- BSI: alle Maßnahmen, die nach vermuteten Eintreten eines Vorfalls aber vor der eigentlichen Datensammlung erfolgen. 
	-> z.B. Identifikation und Enumeration potentieller Datenquellen

\subsection{Konfiguration der Versuchsumgebung}

\subsection{Browserauswahl}

> Browserstudie \cite{Izzati.2022}
- Die Herstellerangaben unterschiedlicher Browser bzgl. Privatheit untersucht
- Firefox 58.02: No Browsing History stored, No Cookies stored, No login Info stored, Tracking Protection Enabled: Disconnect, Download Files not Hidden
- Chrome 63.0.3239: No Browsing History stored, No Cookies stored, No login Info stored, Tracking Protection Enabled: No, Download Files not Hidden

> design aim of Tor: \cite{Muir.2019}
- preventing from writing to disk (Perry et al., 2018) 
- enabling secure deletion of the browser (Sandvik, 2013) (hier nicht relevant)

\subsection{Browsing Szenario}
- Wichtig für White-Box-Ansatz: Browsing Szenario ist bekannt
- URL X … 	(TODO!)

\subsubsection*{VM Konfiguration}

Virtualisierung:
	> Oracle VirtualBox VM, Version 7.0.8 r156879 (Qt5.15.2)
	TODO: Warum Virtualisierung?
	> Pro Browser eine VM

VM Betriebssystem: In Literatur (wie viele?) ausschließlich Windows untersucht -> Da Ziel dieser Arbeit: keine neuen wissenschaftlichen Ergebnisse => auch Windows 10
	Dazu: Windows 10 Pro, Build: 19045.2006 (nicht aktiviert)
	> Auf einer VM installiert, danach OVA erstellt. Von dieser ausgehend, für jeden Browser VM erstellt

VM Massenspeicher: 30 GB (VDI-Format), kein SSD-Laufwerk

VM RAM: Kaum Angaben in der Literatur:
	> \cite{Rochmadi.2017}: 2 GB 
	> \cite{Ohana.2013}: 4 GB
	- Hier: 6 GB (maximal möglich, mit verfügbaren Speicherresourcen, um später RAM-Dumps zu speichern)
	-> Ausblick: Kritik an Literatur, dass RAM-Größe kaum thematisiert wird, obwohl sie Auswirkungen auf Ergebnisse hat -> Siehe Pagefile-Thematik in Analysis Stage

VM Netzwerkeinstellungen: An Literatur orientiert (Quelle?)
	- Netzwerkadapter mit Netzwerkbrücke = direkt mit dem physischen Netzwerk deines Hostsystems verbinden -> virtuelle Maschine eine eigene IP-Adresse im physischen Netzwerk
	- Netzwerkadapter erst nach Browserinstallation aktiviert

Programme auf VM:
	- Untersuchter Browser. Installationsverzeichnisse:
		> Firefox: % C:\Program Files\Mozilla Firefox\firefox.exe
		> Tor: % "C:\Program Files\Tor Browser\Browser\firefox.exe"
		> Chrome:
		> Brave:
			
	- Process Monitor
		"Process Monitor ist ein Dienstprogramm für das Windows-Betriebssystem, das von Microsoft entwickelt wurde. Es ermöglicht die Überwachung und Aufzeichnung aller Aktivitäten und Ereignisse, die auf einem Windows-System im Zusammenhang mit Prozessen, Dateisystemen, Registrierungseinträgen, Netzwerkverbindungen und vielem mehr stattfinden. Mit Process Monitor kannst du detaillierte Informationen über den Betrieb deines Systems erhalten, um Probleme zu diagnostizieren, Softwarekonflikte zu lösen oder verdächtige Aktivitäten zu untersuchen. Das Tool bietet eine umfassende Echtzeit-Überwachungsfunktion, die es dir ermöglicht, alle Ereignisse und Aktionen zu erfassen, die während der Ausführung von Prozessen auf deinem System auftreten."
		% https://learn.microsoft.com/de-de/sysinternals/downloads/procmon
		Version: v3.93
		Hier: Verwendet, um Aktivitäten des Browsers aufzuzeichnen (vgl. Wireshark Capture) -> Siehe Sammlungsphase
	
	- Process Explorer
		"Process Explorer erweitert die Funktionen des Windows Task Managers und ermöglicht es Benutzern, einen umfassenden Überblick über alle aktiven Prozesse und deren Eigenschaften zu erhalten." 
		Hier wichtig: Ermöglicht zudem alle ausgeführten Services einer PID zuzuordnen
		% https://learn.microsoft.com/de-de/sysinternals/downloads/process-explorer
		Version: v17.04

\subsubsection*{Verwendete Tools zur Analyse}

Volatility: % https://www.volatilityfoundation.org/
	"Das Volatility Memory Forensics Framework ist ein Open-Source-Tool, das für die forensische Analyse von Arbeitsspeicherabbildern verwendet wird. Es ist speziell darauf ausgerichtet, Informationen und Artefakte aus dem physischen oder virtuellen Arbeitsspeicher eines Computers zu extrahieren.
	
	Frei auf GitHub verfügbar: % https://github.com/volatilityfoundation/volatility3	
	Version: 2.4.1 (aktuellster Release)
	Basierend auf python

	Die Volatility Foundation wurde ins Leben gerufen, um die Interessen der Volatility-Community zu vertreten und sicherzustellen, dass das Framework frei verfügbar 	bleibt und kontinuierlich weiterentwickelt wird. Das Projekt wird von einer Gemeinschaft von Freiwilligen vorangetrieben, die zur Code-Entwicklung, Dokumentation und Fehlerbehebung beitragen.

	Das Tool wurde entwickelt, um Forensikern und Sicherheitsexperten bei der Untersuchung von Sicherheitsvorfällen, Malware-Angriffen und anderen verdächtigen Aktivitäten zu unterstützen. Durch die Analyse des Arbeitsspeichers können wichtige Informationen wie Prozesse, Netzwerkverbindungen, Dateisystemstrukturen, Registrierungseinträge und sogar verschlüsselte Daten wiederhergestellt werden."

	Volatility3:
	"Volatility 3 tatsächlich eine vollständige Neuschreibung des Volatility Memory Forensics Frameworks zu sein, die im Jahr 2020 veröffentlicht wurde. Diese Neuschreibung sollte viele technische und Leistungsprobleme der ursprünglichen Version beheben, die seit der ersten Veröffentlichung im Jahr 2007 aufgetreten sind.

	Volatility 3 wurde mit dem Ziel entwickelt, eine robustere, flexiblere und skalierbarere Plattform für die Analyse von Arbeitsspeicherabbildern bereitzustellen. Es wurden Verbesserungen hinsichtlich der Architektur, der Leistung und der Plugin-Entwicklung vorgenommen. Die Neuschreibung ermöglichte auch eine Anpassung der Lizenzierung durch die Einführung der Volatility Software License (VSL)."

	Großer Vorteil von Volatility3: kein "Profile" mehr notwendig. = Konfigurationseinstellung, die angibt, um welches Betriebssystem und welche Version es sich im Arbeitsspeicherabbild handelt. Ein Profil definiert die Speicherstruktur und die Verhaltensweisen des Betriebssystems, die Volatility bei der Analyse des Arbeitsspeichers berücksichtigen muss.
		=> Grund: kann neue Symboltabellen für die meisten Windows-Speicherabbilder generieren, basierend auf dem Speicherabbild selbst
		% https://volatility3.readthedocs.io/en/latest/vol2to3.html

	Plugins-Liste: (Ähnlich zu \cite{Hariharan} und \cite{Dayalamurthy.2013})
	  > pslist: Das Plugin "pslist" wird verwendet, um eine Liste der aktiven Prozesse im Arbeitsspeicherabbild zu erstellen. Es extrahiert Informationen wie Prozessnamen, PID (Prozess-ID), Elternprozess, Startzeitpunkt, Priorität und andere Attribute für jeden laufenden Prozess.
			% vol.py -f ram_dump.img windows.pslist > pslist.txt
	
	  > yarascan: Das Plugin "yarascan" ermöglicht die Durchführung einer YARA-basierten Malware-Erkennung im Arbeitsspeicherabbild. YARA ist eine flexible Regel-Engine, die verwendet werden kann, um nach bestimmten Mustern, Signaturen oder Verhaltensmerkmalen von Malware zu suchen. Das Plugin wendet YARA-Regeln auf den Arbeitsspeicher an und gibt potenzielle Treffer aus.
			% vol.py -f ram_dump.img windows.vadyarascan --yara-file yara_rules.yara > yarascan.txt
			Wichtig: yara\_rules.yara = In domainspecific Yara-Skriptsprache geschriebene Regeln, nach welchen RAM durchsucht wird.
			Hier: Einfache String-Pattern, die jeden Schritt des Browsing-Szenarios abdecken:
			*** TODO Yara-Rules einfügen ***
			*** TODO: Erklärung Yara-Ausgaben, mit Screenshot ***
			Wichtig hier: Gibt u.a. virtuelle Adresse des gefundenen Strings im Speicherbereich des Prozesses an
		
	  > memmap: Das Plugin "memmap" dient dazu, eine detaillierte Karte des physischen Speichers des Systems zu erstellen. Es gibt Informationen über die physischen Speicherbereiche, deren Startadressen, Größen, Schutzattribute und andere relevante Details. Dieses Plugin kann bei der Analyse von Speicherlayout und -nutzung hilfreich sein.
			1) % vol.py -f ram_dump.img windows.memmap --pid <PID> > memmap.txt
				gibt Abbildung der Adressen des virtuellen Speichers auf die physikalische Adresse des Speicherbereichs des Prozesses mit der PID <PID>.
				Weiterhin und hier relevant: gibt Abbildung von virtuellen Adresse auf einen Datei-Offset an. Dieser bezieht sich auf die Datei, die erzeugt wird, wenn der Befehl mit dem --dump Flag ausgeführt wird:
			2) % vol.py -f ram_dump.img -o \dump_dir\ windows.memmap --pid <PID> --dump
				führt dies zur physischen Extraktion des Speicherbereichs des Prozesses mit der PID <PID> in eine separate Datei.		
	 	
	  > filescan: Das filescan-Plugin von Volatility durchsucht den Arbeitsspeicher nach Speicherbereichen, die Informationen über Dateien enthalten. Die Ausgabe des Plugins enthält eine Liste von Dateinamen, die im Speicher gefunden wurden. Können sein: Ausführbare Dateien, Bibliotheksdateien oder Konfigurationsdateien.
			% vol.py -f ram_dump.img windows.filescan > filescan.txt
		
	Zusammenhang der Plugins: Siehe Analysephase

	
Autopsy:
	"Autopsy ist ein Open-Source-Digital-Forensik-Tool, das auf der Sleuthkit-Bibliothek basiert. Es wurde entwickelt, um forensische Untersuchungen von digitalen Beweismitteln auf Computern und mobilen Geräten durchzuführen.

	Sleuthkit ist eine Sammlung von Open-Source-Tools für die forensische Analyse von Dateisystemen. Es bietet Funktionen zum Lesen, Analysieren und Wiederherstellen von Daten aus verschiedenen Dateisystemen, darunter NTFS, FAT, exFAT, HFS+, Ext2/3/4 und mehr. Sleuthkit ermöglicht die Extraktion von Metadaten, das Durchsuchen von Dateien und die Analyse von Dateisystemstrukturen, um wichtige Informationen für forensische Untersuchungen zu gewinnen.
	
	Autopsy baut auf der Funktionalität von Sleuthkit auf und bietet eine benutzerfreundliche grafische Benutzeroberfläche für die forensische Analyse. Es erweitert die Funktionalität von Sleuthkit, indem es zusätzliche Tools, Plug-Ins und Automatisierungsfunktionen bereitstellt, um den forensischen Untersuchungsprozess zu unterstützen.
	
	Mit Autopsy können Forensiker die Daten auf einem Datenträger untersuchen, Dateien und Metadaten analysieren, gelöschte Dateien wiederherstellen, Keyword-Suchen durchführen, Berichte generieren und vieles mehr."

	"Sowohl Autopsy als auch Sleuthkit sind weit verbreitete und respektierte Tools in der digitalen Forensik-Community. Sie werden von Forensikern, Strafverfolgungsbehörden, Sicherheitsexperten und anderen Fachleuten eingesetzt, um digitale Beweise zu sammeln, zu analysieren und Berichte zu erstellen, die in rechtlichen Verfahren verwendet werden können."
	
	Hier: Verwendet um Festplattenimages der VMs einzulesen. 

Darstellung von Daten:
	- HxD
	- Notepad++ (inkl. PlugIns, bsow. JSON)

Tools zur Dateiverwaltung:
	- dejsonlz4: % https://github.com/avih/dejsonlz4
		= Hilfsprogramm des GitHub Nutzers "avih", das verwendet wird, um JSON-Dateien zu dekomprimieren, die im LZ4-komprimierten Format vorliegen.
		> jsonlz4 = proprietäres Firefox-Format Benutzerdaten zu speichern. Das Format basiert auf JSON (JavaScript Object Notation), einem verbreiteten Datenformat zur Repräsentation strukturierter Daten. Die Daten werden jedoch zusätzlich mit dem LZ4-Komprimierungsalgorithmus komprimiert, um Speicherplatz zu sparen und den Zugriff zu beschleunigen.
		> Um auf die Informationen in einer "jsonlz4"-Datei zuzugreifen, muss sie zuerst dekomprimiert werden. 
		> "dejsonlz4": Dekomprimierung von lz4. Danach: normale JSON-Datei, die gelesen und verarbeitet werden kann.
	- MZCacheView:
		= Tool zum Parsen von Firefox Cache-Dateien. Kann auf diese Dateien zugreifen und verschiedene Informationen anzeigen, wie z.B. URL, Dateigröße, MIME-Typ und Datum des Downloads.
		> Erwartet Firefox Cache-Dateien: % C:\Benutzer<Benutzername>\AppData\Local\Mozilla\Firefox\Profile<Profilname>\cache2
		> "Das Cache2-Verzeichnis ist der Speicherort, an dem Firefox zwischengespeicherte Dateien speichert" % https://github.com/libyal/dtformats/blob/main/documentation/Firefox%20cache%20file%20format.asciidoc
	- FirefoxCache2: % https://github.com/JamesHabben/FirefoxCache2 
		> Skript "firefox-cache2-index-parser.py"
		> Erweitert MZCacheView, indem die sogenannte "index"-Datei im Cache2 Ordner analysiert wird.
			"Die Indexdatei fungiert als Referenz für den Cache und ermöglicht es Firefox, schnell auf die zwischengespeicherten Dateien zuzugreifen"

\section{Acquisition Stage}

> Browsing Szenario durchfüren

> Zeitpunkte von -> Orientieren an: \cite{Muir.2019}
	- RAM-Dumps 
	- VM-Snapshots (nur letzter Snapshot ist Post-Mortem Forensik)
	- Process Monitor Logfiles
	- Registry Snapshots

VM Snapshot durchführen:
	...
	Dazu folgende Schritte notwendig:
		> VM Festplatte liegt als VirtualBox Disk Image (.vdi  Datei) vor -> Wird nicht von Autopsy unterstützt
		> Deshalb: Umwandlung in .img Format
			"Das Image-Format (.img) ist ein generisches Dateiformat, das verwendet wird, um ein binäres Abbild eines gesamten Datenträgers zu speichern. Es erfasst den Inhalt des Datenträgers, einschließlich des MBR (Master Boot Record), der Partitionstabellen, der Dateisysteme und aller Dateien und Ordner."
		> Dazu "vboxmanage" verwendet = Kommandozeilen-Tool, das bei Installation von VirtualBox enthalten ist.	
			% vboxmanage clonehd <VDI_File>.vdi <IMG_File>.img --format raw
		  	Wobei: .vdi Datei das zu konvertierende VirtualBox Disk Image und .img die neue Image-Datei
		> Hinweis

RAM Dump durchführen:

Process Monitor Logfile erstellen:



- Warum Process Monitor während Browsing?
	o Während Browsing Szenario Filechanges untersuchen: DaemonFS set to monitor all activity within local hard drive\cite{Ohana.2013}
	
- Registry: \cite{Rochmadi.2017}
	•	Das Windows-Registrierungsverzeichnis enthält viele Informationen zur Nutzung des Computers, Benutzerkonfigurationen, Anwendungen und Hardwaregeräte
	•	Informationen im Registrierungsverzeichnis werden nach Ausführungsreihenfolge, Suchschlüsselwörtern, zuletzt aufgerufenen Ordnern, Anwendungsprotokollen und anderen Kategorien sortiert.
	
	

\section{Analysis Stage}

> Analysis Stage (= Kapitel „Results“)
- Analyse der akquirierten Artefakte der vorherigen Phase: VM-Snapshots, RAM-Dumps, Process Monitor Logfiles und Registry Snapshots mit ggf. zusätzlichen Tools
- Oberster Leitsatz dabei: gefundenes Artefakt muss eindeutig Browser zugeordnet werden können: Deshalb einfache Stringsuche in RAM mit WinHex ungenügend -> Hier evtl. negatives Beispiel zu Stringsuche einflechten

\subsection{Common Locations}

Whitebox-Analyse: (gezieltes Suchen nach Dateien) \cite{Bonetti.2014}
	Definition: "White-Box" Computer Forensik bezieht sich auf eine forensische Untersuchungsmethode, bei der der forensische Analyst über umfassende Kenntnisse und Zugriff auf das untersuchte System verfügt. Im Kontext der Computerforensik bezieht sich "White-Box" darauf, dass der Analyst über volle Transparenz und Zugriff auf alle Informationen, Ressourcen und Artefakte des Systems verfügt.
	
	Die "White-Box" Forensik kann verschiedene Techniken und Tools umfassen (z.B. Process Monitor, Regshot, Registry Explorer, Dekomprimierungstools), um Daten wiederherzustellen, gelöschte Informationen wiederherzustellen, Metadaten zu analysieren, Netzwerkaktivitäten zu überwachen und weitere forensische Analysen durchzuführen. Der Fokus liegt darauf, das System vollständig zu verstehen und alle relevanten Beweise zu sammeln.
	
	Hier: In Orten gesucht, die 1. Process Monitor ermittelt hat und 2. in der Literatur vorgeschlagen wurden.

Definition: Common Location
	(= i.d.R. Installationsverzeichnisse der Browser) = „Bekannte Speicherorte“, z.B. bei Firefox   	
	- TODO: Quelle
		> Welche Dateien in Common Locations mit Process Monitor identifiziert
		> Wie haben sich Dateien verändert in verschiedenen Snapshots?
		> Was in Dateien gefunden?	
	-	Ziel: Befinden sich unter den Dateien, die ein Browser direkt auf die Festplatte schreibt private Browsing Artefakte?
	-	Dateien sind Browserspezifisch, befinden sich in bekannten Pfaden. Beispiele: Datenbank-Dateien, Caches, temporäre Dateien.
	-	String-Suche wäre nicht ausreichend, da Artefakte teilweise komprimiert (siehe .jsonlz4)	
	- Beispiele:
		> Cache folder, Web history \cite{Montasari.2015}

\subsubsection*{Schreiboperationen mit Process Monitor verfolgen}
- Process Monitor: WriteFile Operationen von Browser
- Vorgehen: (Siehe Aktivitätsdiagramm)
	o	Basis = Process Monitor Logfile 1 und 2
	o	Processmonitor Filter: Browser-Prozess, Dateioperationen, nur WriteFile
	o	Export als CSV
	o	Datenaufbereitung in Excel
	o	Irrelevante Spalten löschen: Time of Day (zeitl. Kontext nicht wichtig), Process Name (Da in Process Monitor bereits nach Namen gefiltert wurde -> Alle Prozesse haben gleichen Namen), Operation (Da in Process Monitor bereits nach Operation gefiltert wurde -> Alle Prozesse haben gleiche Operation „WriteFile“), Result, Detail
	o	Gleiche Operationen (Duplikate) löschen
	o	Neue Spalte mit Dateiendung
	-> Weiteres gruppieren, sortieren und analysieren ist browserspezifisch
	o	Wenn Daten aufbereitet wurden: 
	1.	Autopsy: Prüfen, ob Dateien noch in Snapshot Image vorhanden
	2.	Wenn ja, Dateien mit Autopsy extrahieren 
	3.	Wenn nein, prüfen, ob Datei in RAM gecacht
	-> Hier beschreiben, wie mit Volatility filelist etc. Dateien aus RAM wiederhergestellt werden können
	4.	Prüfen ob Browsing Artefakte in Dateien enhalten sind: Stringsuche nach Aktionen des Browsing-Protokolls
	
	TODO:
	Allgemein: Nur Dateien untersucht, die gemäß Methodik (Kapitel X) entweder im Snapshot vorhanden sind oder sich über Autopsy Carving PlugIn bzw. RAM wiederherstellen lassen.
	> Wenn Temp-Dateien nicht mehr vorhanden, wird die nicht-Temp Datei aufgeführt
	
	\begin{figure}[h!]
		\centering
		\small
		\centerline{\resizebox{\linewidth}{!}{\input{bilder/process_monitor_to_excel-Latex.pdf_tex}}}
		\caption{TODO: Process Monitor Write Operation to Excel Spreadsheet}
		\label{fig:jes}
	\end{figure}

\subsubsection*{SQLite-Datenbänke}
- Gesondert betrachtet: Zeitlicher Vergleich von SQLite Datenbänken
	> Begründung: In Literatur ermittelt, dass SQLite DB von zentraler Bedeutung bei Browser History -> Hier wird i.d.R. Suchverlauf gespeichert
	> Zählt zu den wichtigsten "Common Locations"
	-> Vorgehen: Siehe Aktivitätsdiagramm
		TODO: WAL Checkpoint
		
\begin{figure}[h!]
	\centering
	\small
	\centerline{\resizebox{\linewidth}{!}{\input{bilder/sqlite_file_diffs-Latex.pdf_tex}}}
	\caption{TODO: Process Monitor Write Operation to Excel Spreadsheet}
	\label{fig:jes}
\end{figure}

\subsection{Registry}
- Registry:
	> Process Monitor: SetValue Operationen von Browser 
		-> Values der Keys untersucht (je nach Datentyp) -> Sonderfall: REG\_BIN
		- Kategorien der Keys auflisten
		Diagramm: z.B. Kreisdiagramm mit Anteil der Kategorien an gesamten Schreiboperationen
	> Stringsuche in Registry Hives mit Registry Explorer (Siehe Liste)
		- Suchbegriffe auflisten
		- Hives (Speicherorte) auflisten
	> "shellactivities-ähnliche" Keys untersucht
		- Arbeit von "shellactivities-ähnliche" Keys erklären

\subsection{Uncommon Locations}

Blackbox-Analyse: \cite{Bonetti.2014} (Stringsuchen im gesamten Image mithilfe von Tool) 
Definition: Auch "triage-style keyword search" \cite{Horsman.2019} genannt, = Durchsuchung des Beweismaterials ohne 
Vorwissen über Browserverhalten (d.h. welche Dateien geschrieben wurden) sowie ohne Vorverarbeitung der Dateien (z.B. Entpacken von Dateien).
Stattdessen: Untersuchen der Images nur mittels vordefinierter Funktionen von Forensik-Tools
"Triage", da dies schnelles erstes Mittel von Forensikern, um nach Acquisition Phase Ergebnisse zu erhalten

Hier entscheidend "Uncommon Locations":
	= „Unbekannte Speicherorte“, nur durch tiefgehende forensische Analyse entdeckt
	
	- TODO: Quelle
	o	Registry
	o	Pagefile.sys
	o	Unallocated Disk Space
	->	Suche nach „obfs4“ deckt Bridging IP-Adressen auf
	o	Windows-Prefetching
	o	Timestamps
	o	\$MFT
	o	\$Unalloc
	o	\$LogFile
	o	Favicons
	o	etilqs
	o	Manifest.json
	o	slack space
	
	- Beispiele in der Literatur:
	> “\$MFT”, “\$LogFile”, “Favicons”, “etilqs”, “Manifest.json”, “pagefile.sys.”, “unallocated space” and “slack space” \cite{Montasari.2015}	
	
	-	Ziel: Untypische Orte, wo private Browsing Artefakte gefunden werden können. Im
	-	Unterschied zu Common Locations: Weitergreifendes Konzept, umfasst Dateien, die nicht von Browsern in bekannten Browser-Ordnern gespeichert werden, sondern auch Speicherorte wie RAM, Registry oder Caches des Rechners, wie 
	-	In Literatur ermittelt: für private Browsing drei uncommon Locations relevant:
	o	Stichwortsuchen in kompletten Speicherabbildern: Festplatte (Common Location Browser-Pfade ausgenommen) + RAM 
	-> Wichtig: String-Treffer muss Browser zugeordnet werden können
	-> Negativbeispiele:
	o \cite{Rochmadi.2017}: in WinHex: URLs, Passwörter gefunden -> Wie wird URL Browser zugeordnet? Reicht gefundener String in RAM-Hex als Beweis aus?
	o \cite{Md.2018} WinHex: email account can be retrieved, retrieves all URL histories including the directories visited by a user
	o \cite{Montasari.2015}	Firefox: URLs und Keywords als Strings in WinHex gesucht und gefunden 
	o \cite{Montasari.2015}	Chrome: URLs und Keywords als Strings in WinHex gesucht und gefunden
	
o	In Literatur oft verwendet: Stichwortsuchen:
	>	Autopsy Keyword-Suche außerhalb der Common Locations, in allen Partitionen
		•	Definition der gesuchten Strings
		•	Weiterführend:  In Literatur nichts über verwendete Plugins gefunden. Hier:
			o	Automatische Kategorisierung von Dateien
			o	Timeliner-Plugin (Wenn verwendbar?)
	>	RAM: Yarascan Treffer -> String Kontext
		•	Definierte Yararules
			- TODO!
		•	HTML-Fragmente: \cite{Said.2011} We were also able to find blocks of HTML code that constructs Web sites we visited.
		•	Image Carving: 
	> Carved from Memdump \cite{Ohana.2013}
	> Bildsuche mit: Griffeye’s DI Analyze Pro with LACE plug-in \cite{Horsman.2019}

- Windows: Prozess-Struktur im RAM: 
	(--> TODO: Wo gefunden?)
	The EPROCESS data structure contains information about process instances, such as image name and ProcessID, the resources allocated in terms of memory allocations (how much and where), types (private, mapped, shareable, etc.), memory protections (combinations of read, write, execute, and reserved), modules loaded, and pointers to ETHREADs and the process environment block.
	
	Both EPROCESS and ETHREAD are considered opaque objects by Microsoft [28], inhibiting analysis; fortunately, third-party work has been done to understand these struc tures [29], [30]. Microsoft does provide symbol files1, which help communicate the layout of data structures [31]. Indeed, Volatility uses these symbols for its own processing.
	
	Included in EPROCESS, the ETHREAD object is an opaque structure which contains useful information about the stack. We calculated the size of a stack from the difference between its limit and base, both of which are attached to the ETHREAD.
	
	Another member of the EPROCESS structure, the VAD tree, maps out the virtually allocated memory for a process [32]. VAD nodes refer to loaded modules (in the allocations in which they were referenced) and also have unique permission flags per node.
	
	The PEB (process environment block) contains data about the number of heaps, which modules have been loaded into memory, and the command-line string that invoked the process [33]. The module list may not match the VAD tree’s list
	exactly, the difference of these two sets indicating images of interest


In Literatur der Web Browser Forensik vorwiegend verwendet: 
- Autopsy Stichwort Suche nach PB Artefakten + Indizieren der Dateien durch Autopsy-PlugIns
- Stichwortsuche in RAM mit Volatility Yarascan PlugIn. Vertiefende Untersuchung für jeden Yara-Rule-Treffer *** Hier werden Artefakte gefunden *** -> Flussdiagramm

\subsubsection*{Analyse mit Autopsy}
Bei White-Box Analyse: Autopsy nur zur Dateiextraktion genutzt, hier: als konkretes forensisches Werkzeug
Wichtig dabei:
- Stichwortsuche 
	-> Screenshot von Suchfunktion
	-> Suchbegriffe auflisten 
- Nutzen der Plug-Ins
- evtl. "pagefile.sys"-Problematik ansprechen


\subsubsection*{Analyse mit Volatility}
Bei White-Box Analyse: RAM nur zur Dateiextraktion genutzt, hier: als konkretes forensisches Werkzeug

Wichtig: Auf Ziel der Arbeit verweisen: gefundenes PB Artefakt muss zwingend Browser zugeordnet werden können -> d.h. gefundener String des Browsing Protocols in Hex des RAM-Dumps reicht nicht als Beweis für gefundenes PB Artefakt aus.
Stattdessen: gefundenes PB Artefakt im RAM muss Browser zugeordnet werden können -> Passendes Werkzeug = Volatility PlugIn "Yarascan" 
-> TODO: Definition Yarascan
-> TODO: Auflistung der Yara-Rules
-> Vorgehen: Siehe Baumdiagramm
\begin{figure}[h!]
	\centering
	\small
	\centerline{\resizebox{\linewidth}{!}{\input{bilder/yarascan_plugin_tree-Latex.pdf_tex}}}
	\caption{TODO: Process Monitor Write Operation to Excel Spreadsheet}
	\label{fig:jes}
\end{figure}





