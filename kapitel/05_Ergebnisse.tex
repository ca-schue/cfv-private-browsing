\chapter{Ergebnisse}

*** TODO ***

\section{Firefox}

Im nachfolgenden Abschnitt werden die Ergebnisse der Datenanalyse für den Webbrowser Firefox detailliert beschrieben. Die Analyse ist in drei Hauptkategorien unterteilt: Common Locations, Uncommon Locations und Registry.

\subsection*{Common Locations}
Zunächst werden die standardmäßigen Speicherorte für Browserartefakte nach potentiellen privaten Browsing Artefakten untersucht. Diese Common Locations beziehen sich ausschließlich auf Dateien, die auf die Festplatte geschrieben werden. In diesem Versuch wird gemäß Methodik in Kapitel X (TODO!) zwischen Schreiboperationen aus den Process Monitor Logfiles und SQLite Datenbänken zur Verwaltung von Nutzerdaten unterschieden. Weder in den Schreiboperationen der Process Monitor Logfiles noch in den SQLite-Datenbänken konnten PB Artefakte gefunden werden. 

Eine detaillierte Analyse der untersuchten Datein im Anhang X beschrieben.

\subsection*{Uncommon Locations}
Nachfolgend werden die Analyseergebnisse der Firefox Uncommon Locations beschrieben.
Wie in Kapitel X erläutert, wird im Gegensatz zu Common Locations die Suchrichtung umgekehrt und es werden alle gesammelten Daten nach einem spezifischen PB Artefakt durchsucht.
Somit benötigt ein Forensiker kein Wissen über das Browserverhalten. Stattdessen wird sich auf die Vollständigkeit der Funktionen von Forensik-Tools verlassen. Im Rahmen dieses Versuchs werden die Tools Autopsy und Volatility verwendet.

\subsubsection*{Analyse mit Autopsy}
Bei den Common Locations in Kapitel X wird Autopsy nur zur Dateiextraktion genutzt. Im Falle der Uncommon Locations dient Autopsy als forensisches Werkzeug zur Datenanalyse.

Eine Autopsy Stichwortsuche gemäß Methodik in Kapitel X (TODO!) lieferte in allen Snapshots keine Treffer. Es wurde zusätzlich das \texttt{\$Carved} Verzeichnis durchsucht, in dem Autopsy alle wiederhergestellten Dateien speichert.

Ebenso wurden in den von Autopsy automatisch kategorisierten Dateien keine PB Artefakte gefunden. Eine detaillierte Analyse der Kategorien "Web Bookmarks", "Web Cookies", "Web History" sowie "Web Categories" ist im Anhang X beschrieben.

\subsubsection*{Analyse mit Volatility}
Nachdem die Firefox Festplattenabbilder als Uncommon Location mit Autopsy untersucht wurden, werden nachfolgend die Analyseergebnisse des RAMs als Uncommon Location beschrieben. 
Dazu wurde eine Stringsuche im gesamten RAM nach PB Artefakten durchgeführt.
Wie in Kapitel X ausführlich beschrieben muss ein gefundener String eindeutig einem Browser zugeordnet werden können. 

Deshalb wurde dazu das Volatility PlugIn "Yarascan" verwendet, ein Werkzeug um nach bestimmten Mustern im RAM zu suchen. Dazu wurden die in Tabelle X aufgeführten Yara-Regeln verwendet.
Wie in Kapitel Methodik (TODO!) beschrieben, wird davon ausgehend das PlugIn "pslist" verwendet, um den Prozessnamen anhand PID zu identifizieren.
Die Ergebnisse dieser Stringsuche sind nachfolgend nach Kategorie geordnet.

\paragraph*{Yararule HTML}
In keinem der Firefox RAM Dumps wurden HTML Fragemente der besuchten Seiten gefunden. Somit wird diese Yara-Regel nicht weiter betrachtet.

\paragraph*{Yararule Keyword}
Wie in Abbildung X (TODO!) gezeigt, wurden alle Suchbegriffe "pfaffenhofen", "nanoradar", "mooserliesl" sowie "mallofamerica" identifiziert im zwiten RAM Dump, nach dem Browsing Szenario mit geöffnetem Browser, identifiziert. Die Artefakte befinden sich ausschließlich im zweiten RAM Dump. Die Suchbegriffe wurden größtenteils in den Speicherbereichen von Firefox-Prozessen gefunden. Nur in zwölf Fällen wurden Suchbegriffe in anderen Prozessen identifiziert. Am häufigsten wurde der Suchbegriff "pfaffenhofen" mit 1301 gefundenen Artefakten im zweiten Firefox RAM Dump gefunden. Dies ist vermutlich auf den Google Maps Kartenbereich zurückzuführen, einen visuellen Ausschnitt der Karte, welcher bei der Google-Suche erscheint und Informationen über die geografische Lage, Straßen, Sehenswürdigkeiten und andere relevante Orte in der gesuchten Stadt zeigt. In den RAM Dumps 1 und 3 konnten Artefakte zu den Suchbegriffen identifiziert werden.
\begin{figure}[h!]
	\centerline{\resizebox{\linewidth}{!}{\includegraphics{bilder/volatility/firefox/keywords.png}}}
	\label{chart:final-criteria}  
	\caption{Keywords}
\end{figure}

\paragraph*{Yararule URL}
\begin{figure}[h!]
	\centerline{\resizebox{\linewidth}{!}{\includegraphics{bilder/volatility/firefox/url.png}}}
	\label{chart:final-criteria}  
	\caption{URL}
\end{figure}
Es konnten in den Arbeitsspeicherabbildern alle besuchten URLs unitree.com, mooserliesl.de, mallofamerica.com sowie donaukurier.de identifiziert werden.
Dabei wurden die meisten Artefakte nach dem Browsing Szenario mit geöffnetem Browser (RAM Dump 2) gefunden. Alle besuchten URLs wurden in diesem Dump sowohl in Firefox als auch anderen Prozessen gefunden, wobei die meisten Artefakte in Firefox Prozessen zu finden sind. Dabei wurde "mooserliesl.de" mit insgesamt 390 Artefakten am wenigsten gefunden, "donaukurier.de" mit über 3600 Artefakten am häufigsten.

Bemerkenswert ist, dass URL Artefakte gefunden wurden, nachdem der Browser geschlossen wurde (RAM Dump 3). Dabei wurde kein URL Artefakte in einem Firefox Prozess gefunden.
Anhand der PID $2252$ wurde festgestellt, dass sich alle URL Artefakte des dritten RAM Dumps in einem "svchost.exe" Prozess mit der gleichen PID befinden. Unter dem "Service Host" Prozess laufen gruppierte Windows-Dienste, um Ressourcen zu sparen und die Systemleistung zu verbessern.
Volatility bietet das Plugin "svcscan" an, mit dem alle laufenden Dienste ausgegeben werden können.
Bei Anwendung auf den dritten RAM Dump wurde jedoch zu keinem Dienst eine PID angegeben, wordurch der Dienst mit den URL Artefakten nicht im RAM identifiziert werden konnte. 
% https://learn.microsoft.com/de-de/windows/application-management/svchost-service-refactoring
Stattdessen wurde der dritte Snapshot aufgetaut, um im laufenden Windowsbetrieb den Dienst mithilfe des Process Explorers zu identifizieren.
Wie in Abbildung X (TODO!) gezeigt, wurde anhand der PID $2252$ der Dienst "DNSCache" ermittelt.
\begin{figure}[h!]
	\centerline{\resizebox{\linewidth}{!}{\includegraphics{bilder/firefox-dnscache.png}}}
	\label{chart:final-criteria}  
	\caption{URL}
\end{figure}
Der DNSCache-Dienst unter Windows ist ein Teil des Betriebssystems, der für die Übersetzung von Domainnamen in IP-Adressen verantwortlich ist. Der DNSCache-Dienst speichert DNS-Anfragen und Antworten temporär, umd wiederholte DNS-Anfragen zu beschleunigen.
% https://learn.microsoft.com/en-us/answers/questions/47441/how-to-disable-windows-10-dns-cache-services
Nach Löschen des DNSCaches mit dem Kommandozeilenbefehl \texttt{ipconfig /flushdns}, dem Schließen aller Process Monitor Instanzen sowie Beenden des DNSCaches Services wurde erneut ein RAM-Dump durchgeführt. Dort wurden keine URL Artefakte mehr gefunden.

\paragraph*{Yararule Mail}
\begin{figure}[h!]
	\centerline{\resizebox{\linewidth}{!}{\includegraphics{bilder/volatility/firefox/mail.png}}}
	\label{chart:final-criteria}  
	\caption{Mail}
\end{figure}
Es konnten alle E-Mail Artefakte des Browsing Szenarios gefunden werden.
Die Artefakte befinden sich ausschließlich im zweiten Firefox RAM Dump, nach dem Browsing Szenario mit geöffnetem Browser.
Unter den gefundenen Artefakten befindet sich mit zwölf Vorkommen am häufigsten die Absenderadresse "computerforensikvl@gmail.com". Dieses Artefakt wurde als einziges Mail-Artefakt in anderen Prozesses außer Firefox gefunden.

Bemerkenswert ist, dass das Passwort des Google-Accounts, mit dem die E-Mails verschickt wurden, vier mal als Klartext im RAM gefunden wurden. Das Passwort wurde in je zwei Firefox Prozessen mit den PIDs 7420 und 8424 zwei mal gefunden. Tabelle X zeigt die virtuellen Speicheradressen der Artefakte aus der Yarascan Ausgabe.
\begin{table}[]
\resizebox{\linewidth}{!}{
\begin{tabular}{|c|c|c|ll}
\cline{1-3}
\textbf{Virtuelle Speicheradresse} & \textbf{PID} & \textbf{Byte-Offset in extrahierter Speicherseite} &  &  \\ \cline{1-3}
0xb9ce29180c8                      & 7420         & 0x11dd40c8                                         &  &  \\ \cline{1-3}
0x2859f4ffd4e0                     & 7420         & 0x12e234e0                                         &  &  \\ \cline{1-3}
0x24083b41858                      & 8424         & 0x583858                                           &  &  \\ \cline{1-3}
0x240840e5b08                      & 8424         & 0x96bb08                                           &  &  \\ \cline{1-3}
\end{tabular}
}
\end{table}

Zu diesen Artefakten wurde gemäß Methodik in Kapitel X der String Kontext -- die Zeichen vor und nach dem gefundenen Artefakt im Speicherbereich -- ermittelt. Dazu wurde gemäß Methodik in Kapitel X mithilfe des Volatility memmap-Plugins die Abbildung der virtuellen Speicheradressen auf den Byte-Offset in der extrahierten Speicherseite des Prozesses ermittelt. 
\begin{figure}[h!]
	\centerline{\resizebox{0.7\linewidth}{!}{\includegraphics{bilder/volatility/firefox/password_0xb9ce29180c8_7420.png}}}
	\label{chart:final-criteria}  
	\caption{Password in memory page of PID 7420 at Byte-Offset 0x11dd40c8}
\end{figure}
Wie in Abbildung X gezeigt, sind in der Speicherseite des Prozesses mit PID $7420$ konnte vor und nach dem gefundenen Passwort am Byte-Offset $0xb9ce29180c8$ neben der Gmail-Url "mail.google.com" Code-Fragmente der "Gecko-Engine" zu finden. Dieser Teil des Firefox Browsers ist für das Rendering von Webinhalten verantwortlich, einschließlich HTML, CSS, JavaScript und anderen Medienformaten wie Bildern, Audio und Video.
% https://wiki.mozilla.org/GeckoMediaPlugins
\begin{figure}[h!]
	\centerline{\resizebox{0.7\linewidth}{!}{\includegraphics{bilder/volatility/firefox/password_0x2859f4ffd4e0_7420.png}}}
	\label{chart:final-criteria}  
	\caption{Password in memory page of PID 7420 at Byte-Offset 0x12e234e0}
\end{figure}
In der gleichen Datei konnte nach dem gefundenen Passwort am Byte-Offset $0x24083b41858$ die Strings "Passwd" sowie "sessionrestore" (siehe Common Location "Sessionstore" in Kapitel X) identifiziert werden. 
Wie in den Abbildungen X und Y (TODO!) gezeigt, können in den Byte-Offsets der gefundenen Passwörter in der Speicherseite der PID 8424 konnten kein Kontext ermittelt werden. Im Gegensatz zur Speicherseite der PID 7420 wird das Passwort dort mit 2 Bytes pro Zeichen enkodiert. Das eine Unicode-Zeichenenkodierung vermuten.
\begin{figure}[h!]
	\centerline{\resizebox{0.7\linewidth}{!}{\includegraphics{bilder/volatility/firefox/password_0x24083b41858_8424.png}}}
	\label{chart:final-criteria}  
	\caption{Password in memory page of PID 8424 at Byte-Offset 0x583858}
\end{figure}
\begin{figure}[h!]
\centerline{\resizebox{0.7\linewidth}{!}{\includegraphics{bilder/volatility/firefox/password_0x240840e5b08_8424.png}}}
\label{chart:final-criteria}  
\caption{Password in memory page of PID 8424 at Byte-Offset 0x96bb08}
\end{figure}
				
\paragraph*{Yararule Image}
Das im Browsing Szenario geöffnete Donaukurier Logo wurde ausschließlich im zweiten RAM Dump in drei mal in Firefox Prozessen gefunden.
\begin{figure}[h!]
	\centerline{\resizebox{\linewidth}{!}{\includegraphics{bilder/volatility/firefox/image.png}}}
	\label{chart:final-criteria}  
	\caption{Image}
\end{figure}

Wie in Abbildung X zusammenfassend gezeigt wurden vor dem Browsing Szenario, keine private Browsing Artefakte im ersten RAM Dump gefunden.
Nach dem Browsing Szenario mit geöffnetem Browser konnten die meisten Artefakte identifiziert werden. Dabei wurden am häufigsten URL Artefakte in Firefox Prozessen gefunden. Zudem konnte hier das E-Mail Passwort im Klartext lokalisiert werden.
Nach Schließen des Browsers konnten im dritten Snapshots URLs im DNSCache Windows Service gefunden werden. Nach leeren des Caches und Beenden des DNSCache Services konnten wurden keine Artefakte gefunden.
\begin{figure}[h!]
	\centerline{\resizebox{\linewidth}{!}{\includegraphics{bilder/volatility/firefox/summary.png}}}
	\label{chart:final-criteria}  
	\caption{Summary}
\end{figure}


\subsection*{Registry}

Die Analyse der Registry zählt gemäß Methodik in Kapitel X sowohl zu den Common als auch Uncommon Locations. Weder in den Process Monitor "SetValue" Operations noch in den System- und User-Hives konnten PB Artefakte gefunden werden. Eine detaillierte Analyse dieser Common- und Uncommon Locations der Registry ist im Anhang X beschrieben.


*** TODO: Zusammenfassung Firefox ***


\newpage


% ######################################################################
% ######################################################################
% ######################################################################
% ######################################################################


\section{Tor}

In diesem Abschnitt werden die Ergebnisse der Datenanalyse der Common Locations, Uncommon Locations sowie der Registry für den Tor-Browser präsentiert.

\subsection*{Common Locations}

Als Erstes werden die Common Locations analysiert, um potenzielle Hinweise auf Internetaktivitäten des Browsing Szenarios zu finden. Bei der Untersuchung der gängigen Speicherorte wurde gemäß der im Kapitel X (TODO!) beschriebenen Methodik zwischen Schreibvorgängen in den Protokolldateien des Process Monitors und den SQLite-Datenbanken zur Verwaltung von Benutzerdaten unterschieden. Dabei konnten in keiner Datei PB Artefakte gefunden werden. Eine detaillierte Analyse der Process Monitor "WriteFile" Operations sowie SQLite-Datenbänken ist im Anhang X beschrieben.

\subsection*{Uncommon Locations}
Nachfolgend werden die Analyseergebnisse der Tor Uncommon Locations beschrieben.
Dazu werden die vollständigen Speicherabbilder nach PB Artefakten untersucht ohne das genaue Browserverhalten zu berücksichtigen. Stattdessen wird sich auf die Vollständigkeit der Funktionen der Forensik-Tools Autopsy und Volatility verlassen.

\subsubsection*{Analyse mit Autopsy}

Im ersten Schritt wird Autopsy als konkretes forensisches Werkzeug verwendet, statt nur nur zur Dateiextraktion, wie es bei den Common Locations der Fall war.

Eine Stichwortsuche in Autopsy in allen fünft Festplatten-Images nach PB Artefakten ergab keine Treffer.

Ebenso wurden in den von automatisch kategorisierten Dateien kein PB Artefakte gefunden. 
Im Anhang X ist eine detaillierte Analyse der kategorisierten Dateien beschrieben.


\subsubsection*{Analyse mit Volatility}

Nachfolgend werden die Ergebnisse der Analyse des RAMs mithilfe Volatility beschrieben. 

\paragraph*{Yararule HTML}
Wie bei Firefox in Kapitel X (TODO!) konnten keine HTML Artefakte im RAM gefunden werden. Deshalb wird diese Kategorie nicht aufgeführt.

\paragraph*{Yararule Keyword}
Wie in Abbildung X gezeigt, wurden ausschließlich während des Browsing-Szenarios (RAM-Dump 2) und nach Erstellen einer "Neuen Identität" (RAM Dump 3-1) Keyword Artefakte gefunden.
Nachdem eine "Neue Identität" erstellt wurde reduzierten sich die gefundenen Artefakte deutlich. 
Die Keyword-Artefakte wurden hauptsächlich in Firefox Prozessen gefunden. Kein Artefakt war im Tor.exe Prozess zu finden.
Mit 4833 Artefakten wurden am häufigsten "pfaffenhofen" nach dem Browsing Szenario im zweiten RAM-Dump gefunden. 
Nach dem Schließen des Tor-Browsers wurde keine Keyword Artefakte mehr im RAM identifiziert.
\begin{figure}[h!]
	\centerline{\resizebox{\linewidth}{!}{\includegraphics{bilder/volatility/tor/keywords.png}}}
	\label{chart:final-criteria}  
	\caption{Keywords}
\end{figure}

\paragraph*{Yararule URL}
Wie bei der "Keyword" Yararule wurden ausschließlich während des Browsing-Szenarios (RAM-Dump 2) und nach Zuweisung einer neuen Identität (RAM-Dump 3-1) URL Artefakte gefunden. Ebenso wurden im RAM Dump 3-1 bei deutlich weniger Artefakte URLs als in RAM Dump 2 gefunden. 
Für diese Yara-Regel wurden nach Firefox-Prozessen hauptsächlich Artefakt in Tor-Prozessen gefunden. Am wenigsten Artefakte waren in anderen Prozessen zu finden.
Auffällig ist, dass die URL "mallofamerica.com" 26.505 mal in RAM-Dump 2 gefunden wurde. Im Gegensatz dazu wurde "mooserliesl.de" nur 508 mal im zweiten RAM Dump  gefunden. Nach Schließen des Tor-Browsers wurden keine URL Artefakte mehr im RAM gefunden.
\begin{figure}[h!]
	\centerline{\resizebox{\linewidth}{!}{\includegraphics{bilder/volatility/tor/url.png}}}
	\label{chart:final-criteria}  
	\caption{URL}
\end{figure}

\paragraph*{Yararule Mail}
Nach dem Browsing-Szenario, vor Zuweisung einer "Neuen Identität" (RAM-Dump 2) konnten alle Mail Artefakte gefunden werden.
\begin{figure}[h!]
	\centerline{\resizebox{\linewidth}{!}{\includegraphics{bilder/volatility/tor/mail.png}}}
	\label{chart:final-criteria}  
	\caption{Mail}
\end{figure}
Wie in Abbildung X zu sehen ist, wurden die Artefakte ausschließlich in Firefox Prozess gefunden.
Nur die Absenderadresse "computerforensikvl@gmail.com" wurde nach Erstellen der neuen Identität in RAM-Dump 3-1 gefunden. Die Absenderadresse ist ebenso das am häufigsten gefundene Mail Artefakt.
Wie bei Firefox in Kapitel X (TODO!) wurde das Passwort als Klartext nach dem Browsing-Szenario im zweiten RAM-Dump gefunden.
Das Passwort wurde in zwei Mal im Firefox Prozess mit der PID 708 gefunden. Tabelle X zeigt die virtuellen Speicheradressen der Artefakte aus der Yarascan Ausgabe sowie deren Abbildung auf die mittels "memmap" identifizierten Byte-Offsets der extrahierten Speicherseiten.
\begin{table}[]
\resizebox{\linewidth}{!}{
\begin{tabular}{|c|c|c|ll}
\cline{1-3}
\textbf{Virtuelle Speicheradresse} & \textbf{PID} & \textbf{Byte-Offset in extrahierter Speicherseite} &  &  \\ \cline{1-3}
0x2b1e2c22318                      & 708         & 0xea0318                                         &  &  \\ \cline{1-3}
0x2b1e2ecb748                     & 708         & 0x10f7748                                         &  &  \\ \cline{1-3}
\end{tabular}
}
\end{table}
Bei Untersuchung des String-Kontexts wurden für das Passwort am Byte-Offset \texttt{0xea0318} wie in Abbildung X gezeigt keine Auffällige Artefakte entdeckt.
\begin{figure}[h!]
	\centerline{\resizebox{0.7\linewidth}{!}{\includegraphics{bilder/volatility/tor/password_0xea0318.png}}}
	\label{chart:final-criteria}  
	\caption{Password in memory page of PID 708 at Byte-Offset 0xea0318}
\end{figure}
Wie in Abbildung X gezeigt, wurde im Bereich des gefundenen Passworts am Byte-Offset \texttt{0x10f7748} der String "CSP\_ignoringSrcForStrictDynamic", dessen Bedeutung nicht bestimmt werden konnte.
Weiterhin wurde die Zeichenkette "invalidation/lcs/client" in der Nähe des Passworts gefunden. Dieser String wird in einem Firefox Bug-Ticket verwendet, welches vor 2017 geschlossen wurde. Der Bug betraf ein Speicher-Leck.
% https://bugzilla.mozilla.org/show_bug.cgi?id=1452114
\begin{figure}[h!]
	\centerline{\resizebox{0.7\linewidth}{!}{\includegraphics{bilder/volatility/tor/password_0x10f7748.png}}}
	\label{chart:final-criteria}  
	\caption{Password in memory page of PID 708 at Byte-Offset 0x10f7748}
\end{figure}
	
\paragraph*{Yararule Image}
Wie in Abbildung X dargestellt, wurde der Hexadezimal-Wert des Donaukurier-Logos  ein einziges Mal nach dem Browsing Szenario, vor Erstellen der "Neuen Identität" im zweiten RAM-Dump in einem Firefox Prozess gefunden.
\begin{figure}[h!]
	\centerline{\resizebox{\linewidth}{!}{\includegraphics{bilder/volatility/tor/image.png}}}
	\label{chart:final-criteria}  
	\caption{Image}
\end{figure}


Wie in Abbildung X zusammenfassend gezeigt ausschließlich nach dem Browsing-Szenario vor (RAM-Dump 2) und nach (RAM-Dump 3-1) Zuweisung einer "Neuen Identität" Brwosing Artefakte im Arbeitsspeicher gefunden.
Nach dem Browsing Szenario mit geöffnetem Browser konnten die meisten Artefakte identifiziert werden. Dabei wurden am häufigsten URL Artefakte in Firefox Prozessen gefunden. Zudem konnte hier das E-Mail Passwort im Klartext lokalisiert werden.
Nach Schließen des Browsers konnten im dritten Snapshots URLs im DNSCache Windows Service gefunden werden. Nach leeren des Caches und Beenden des DNSCache Services konnten wurden keine Artefakte gefunden.
Die Erstellung einer "Neuen Identität" reduzierte dabei deutlich die gefundenen Artefakte im RAM.
In keinem RAM-Speicherabbild konnten HTML-Fragmente der im Browsing-Szenario besuchten Seiten identifiziert werden. Weiterhin wurde zwei Mal das Passwort des Google-Accounts im Klartext gefunden.
\begin{figure}[h!]
	\centerline{\resizebox{\linewidth}{!}{\includegraphics{bilder/volatility/tor/summary.png}}}
	\label{chart:final-criteria}  
	\caption{Summary}
\end{figure}
%Literatur:
%o Autopsy: \cite{Muir.2019}
%	•	Configuration files, downloaded files, and browserrelated data are recoverable from the file system.
%	•	Significant data-leakage from the browsing session occurred: HTTP header information, titles of web pages and an instance of a URL were found in registry files, system files, and unallocated space.
%o RAM-Analyse nach \cite{Muir.2019}:
%	•	Live-Analyse identifiziert auch nach dem Schließen und Deinstallieren des Browsers und Abmelden des Benutzers Spuren von Tor-Prozessen, einschließlich des absoluten Pfads zur Browser-Executable, des Benutzernamens und des Geräts, von dem es ausgeführt wurde.
%	•	The data-leakage contained the German word for ’search’ in reference to a Google search. This hints at the locale of the Tor server used to exit the network (exit relay).
%
%o RAM-Analyse nach \cite{Hariharan.2022}:
%	o	process was found to be firefox.exe
%	o	pslist and pstree: parent process was shown 
%	o	Belkasoft Ram Capturer: retrieve information about facebook
%	o	Cmdline: file path of the browser “E:/TorBrowser/Browser/firefox.exe” + name of process tor.exe and firefox.exe
%	o	Dlllist: DLL files of the executable files were not captured
%	o	Netscan: tor.exe + obfs4proxy.exe -> showed “LISTENING” connections to nonstandardized ports as output.
%	Yarascan: was able to retrieve all the browsing sessions
%o RAM-Analyse nach \cite{Sajan.2021} mit Volatility
%	•	process list extracted from the memory
%	•	registry hives been extracted from the memory dump
%	•	threads were extracted: “D:/VolatilityWorkbench/volatility.exe”–plugins=”D:/VolatilityWorkbench/profiles” pslistfilename =”C:/Users/username/Desktop/tor.raw” –profile=Win10x64 17763 –kdbg=0xf807606ac5e0
%	•	Handles: resources used by the process 5672
%	•	Dlls: These dlls can be found from prefetch file --> Can be found in “prefetch” file -> Analyzed with “winprefetchview”
%	•	Places.sqlite: SQLite viewer has been used to recover bookmarks and frequently visited sites even after uninstalling the application
%	•	Visited Websites: Using keyword search in Dump’s Hex
%
%o Registry:
%	> Shellactivites (siehe Firefox) \cite{Muir.2019}: instance of a URL were found in registry file
%	> \cite{Nelson.2020} The userassist key is located in the NTUSER.dat hive of the
%		 -> Registry and indicates the execution path of the program, as well as the number of times the program was executed 

\subsection*{Registry}

Wie in der Methodik in Kapitel X beschrieben, teilt sich die Analyse der Registry sowohl Common als auch Uncommon Locations. Weder in den Process Monitor "SetValue" Operations noch über die Stringsuche in den System- und User-Hives konnten PB Artefakte gefunden werden. Eine detaillierte Analyse der Registry ist im Anhang X beschrieben. 

\section{Chrome}

\subsection*{Uncommon Locations}

o Autopsy Keyword-Suche: 
	> Chrome and Edge produced five artefacts as reported by both tools. (FTK, Autopsy) \cite{Gabet.2018}
		--> Artefakte werden nicht genannt!
	> only two temporary files (Figure 7) were recovered with Minitool Power Data Recovery but it was a dead end; Location: appdata/…/Chrome/…/ Preferences/RF1533fa.TMP \cite{Fayyad.2021}
	> pagefile.sys file showed no traces at all \cite{Said.2011}
	

\section{Brave}