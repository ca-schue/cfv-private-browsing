\chapter{Ergebnisse}

*** TODO ***

\section{Firefox}

Im nachfolgenden Abschnitt werden die Ergebnisse der Datenanalyse für den Webbrowser Firefox detailliert beschrieben. Die Analyse ist in drei Hauptkategorien unterteilt: Common Locations, Uncommon Locations und Registry.

\subsection*{Common Locations}

Zunächst werden die standardmäßigen Speicherorte für Browserartefakte nach potentiellen privaten Browsing Artefakten untersucht. Diese Common Locations beziehen sich ausschließlich auf Dateien, die auf die Festplatte geschrieben werden. In diesem Versuch wird gemäß Methodik in Kapitel X (TODO!) zwischen Schreiboperationen aus den Process Monitor Logfiles und SQLite Datenbänken zur Verwaltung von Nutzerdaten unterschieden.

\subsubsection*{Process Monitor WriteFile Operations}

Gemäß Versuchsdurchführung in Abbildung X (TODO!) wurden für Firefox mit dem Process Monitor Tool zwei Logfiles erstellt. Diese Dateien enthalten alle aufgezeichneten Prozessaktivitäten während und nach dem Browsing Szenario.
Zunächst werden die beiden Logfiles gemäß Methodik in Kapitel X (TODO!) in Excel aufbereitet. 
Im Anhang X (TODO!) ist dazu eine Tabelle mit allen in den gefilterten Logfiles identifizierten Dateien aufgeführt.
Dabei wurde für jede Datei vermerkt
ob und wie sie wiederherstellbar war, mit welchem Tool die Datei analysiert wurde und ob PB Artefakte enthalten sind.

Abbildung X (TODO!) zeigt diese Tabelle in reduzierter Darstellung.
Dazu wurden ausschließlich wiederherstellbare Dateien aufgeführt. 
Die Dateien wurden in die fünf Kategorien "Cache", "datareporting", "Sessionstore-Backup" und "Sonstige Dateien" eingeordnet.
Für jede Datei wurde vermerkt, ob in der entsprechenden Logfile PB Artefakte geschrieben wurden.
Dies trifft für keine der identifizierten Dateien zu.
\begin{figure}[h!]
	\resizebox{\linewidth}{!}{\includegraphics{bilder/firefox-tabelle-logfile1vlogfile2-reduced.png}}
%	\label{...}
	\caption{Tabelle mit wiederherstellbaren Dateien: Logfile 1 vs. Logfile 2}
\end{figure}

Bei detaillierter Untersuchung der Dateien, können zwei Pfade identifiziert werden, in die Firefox während des Versuchs Dateien schreibt. Nur die Dateien in der Cache Kategorie sind im Local Pfad gespeichert.
\begin{itemize}
\item[\textbf{Local}] \texttt{C:$\backslash$Users$\backslash$<User>$\backslash$AppData$\backslash$Local$\backslash$Mozilla$\backslash$Firefox$\backslash$Profiles$\backslash$<Profile>.default-release$\backslash$}
\item[\textbf{Roaming}] \texttt{C:$\backslash$Users$\backslash$<User>$\backslash$AppData$\backslash$Roaming$\backslash$Mozilla$\backslash$Firefox$\backslash$Profiles$\backslash$<Profile>.default-release$\backslash$}
\end{itemize}
In Tabelle X (TODO!) sind die Dateien je nach Speicherort "Local" (Hellblau) oder "Roaming" (Dunkelblau) entsprechend eingefärbt. 

\paragraph*{Cache}

Firefox verwendet den Cache, um Webseiten und deren Ressourcen temporär lokal zu speichern. Dadurch können wiederholte Anfragen an den Server vermieden und die Ladezeiten verringert werden. Die Inhalte dieser Dateien sind binär.
Die Dateien im Format \texttt{$\backslash$cache2$\backslash$entries$\backslash$<ID>} werden dem Cache zugeordnet und im Local Pfad gespeichert.
% https://www.techguy.org/threads/what-exactly-is-in-firefoxs-cache2-folder.1221567/
Wie in Kapitel X beschrieben, können diese Dateien mit dem Tool MZCacheView eingelesen werden.
Wie in Abbildung X gezeigt, konnten im Cache-Ordner im zweiten Snapshot drei JSON Dateien identifiziert werden. Dabei handelt es sich um Zertifikatsdateien, die von der "One Certificate Revocation List" stammen, ein Mechanismus von Firefox zur Überprüfung von Zertifikaten. In keinem der Zertifikate konnten mit HxD private Browsing Artefakte oder besuchte Seiten gefunden werden.
Weiterhin befindet sich im Cache das HTML-Dokument der Firefox Datenschutzseite, welche sich beim ersten Start des Browsers automatisch öffnete. % TODO: Siehe Kapitel X
Weitere Cache Dateien konnten in keinem Snapshot gefunden werden.
\begin{figure}[h!]
	\resizebox{\linewidth}{!}{\includegraphics{bilder/firefox-cache.png}}
%	\label{...}
	\caption{Tabelle mit wiederherstellbaren Dateien: Logfile 1 vs. Logfile 2}
\end{figure}
Die Indexdatei \texttt{$\backslash$cache2$\backslash$index} dient als Datenbank im Cache. Sie ermöglicht dem Firefox-Browser, schnell auf die zwischengespeicherten Ressourcen zuzugreifen und diese effizient zu verwalten. Sowohl mit HxD als auch dem Tool FirefoxCache2 konnten keine PB Artefakte identifiziert werden.
Schließlich enthält die Datei \texttt{$\backslash$jumpListCache$\backslash$ZKJGVJPzPe7w4w0KwEY0jw==.ico} ein $64x64$ Pixel großes Mozilla Logo. Dieses Logo ist keinem Schritt aus dem Browsing Szenario zuzuordnen


\paragraph*{Datareporting}
Dateien im Ordner \texttt{$\backslash$datareporting$\backslash$glean$\backslash$db} sind Teil des Glean-Systems, das für die Sammlung von Telemetriedaten und deren Übermittlung an Mozilla verwendet wird. 
% https://github.com/mozilla/glean
Die Datei \texttt{data.safe.bin} enthält verschlüsselte und anonyme Informationen über die Nutzung des Browsers. In HxD konnten keine keine PB Artefakte gefunden werden
Dateien im Foremat \texttt{$\backslash$datareporting$\backslash$glean$\backslash$db$\backslash$<Profilname>.new-profile.jsonlz4} speichern Informationen über das Firefox-Profil, das von Glean verwendet wird. Wie in Kapitel X beschrieben, lassen sich Dateien, im proprietären \textit{jsonlz4}-Format mit dem Tool dejsonlz4 dekomprimieren. Die entstandene JSON Datei wird mit dem Notepad++ JSON Plugin untersucht. Dabei konnten keine PB Artefakte gefunden werden.

\paragraph*{Sessionstore}
Die Datei \texttt{$\backslash$sessionstore-backups$\backslash$recovery.jsonlz4} enthält eine Sicherungskopie der vorherigen Sitzung. Sie wird erstellt, wenn der Firefox-Browser nach einem Absturz oder einem unerwarteten Beenden neu gestartet wird." % https://support.mozilla.org/de/questions/1221836
Jefferson Scher entwickelte ein Online-Tool zur Analyse von \textit{Sessionstore-Backup} Dateien.
% https://www.jeffersonscher.com/ffu/scrounger.html)
In der Sitzungswiederherstellung konnten wie in Abbildung X gezeigt lediglich die automatisch geöffnete Seite über Firefox Datenschutzhinweise identifiziert werden.
\begin{figure}[h!]
	\resizebox{\linewidth}{!}{\includegraphics{bilder/firefox-sessionstore.png}}
%	\label{...}
	\caption{Tabelle mit wiederherstellbaren Dateien: Logfile 1 vs. Logfile 2}
\end{figure}

\paragraph*{Sonstige Dateien}
In der Datei \texttt{prefs-1.js} werden benutzerspezifische Einstellungen und Konfigurationen für den Firefox-Browser gespeichert. Die Datei enthält Präferenzen des Benutzers in Form von JavaScript-Objekten. Es konnten mit HxD keine PB Artefakte gefunden werden.
% https://kb.mozillazine.org/Prefs.js_file
Schließlich speichert die Datei \texttt{xulstore.json} benutzerspezifische Anpassungen und Konfigurationen für den Firefox-Browser. In der Datei konnten mit Notepad++ keine PB Artefakte gefunden werden.
% https://support.mozilla.org/de/kb/firefox-support-troubleshooting-guide
	
\subsubsection*{SQLite Datenbänke}
Wie in Kapitel X (Methodik, TODO!) erwähnt, werden SQLite Datenbanken als Datenstrukturen für Nutzerdaten genauer untersucht. Mithilfe der Process Monitor Logfiles wurden die in Tabelle X dargestellten SQLite-Datenbanken für Firefox identifiziert:
\begin{table}[]
\resizebox{\linewidth}{!}{
\begin{tabular}{|l|l|lll}
\cline{1-2}
\textbf{Datenbank}                        & \textbf{Gespeicherte Daten}                                                                                              &  &  &  \\ \cline{1-2}
\textit{places.sqlite}                    & Informationen über Lesezeichen und Verlauf. Zu jeder besuchten Webseite: URL, Seitentitel, Zeitstempel des Besuchs etc.  &  &  &  \\ \cline{1-2}
\textit{cookies.sqlite}                   & Von besuchten Webseiten verwendete Cookies.                                                                              &  &  &  \\ \cline{1-2}
\textit{storage.sqlite}                   & Diverse Webdaten, z. B. Indexed-Datenbanken, Offline-Cache-Daten und andere lokale Speicherinformationen.                &  &  &  \\ \cline{1-2}
\textit{favicons.sqlite}                  & Enhtält Favicons (kleine Symbole in der Adressleiste) um besuchte Webseiten visuell zu identifizieren.                   &  &  &  \\ \cline{1-2}
\textit{webappsstore.sqlite}              & Speichert Informationen über installierte Webanwendungen im Firefox-Browser, z.B. Berechtigungen und Einstellungen.      &  &  &  \\ \cline{1-2}
\textit{1657114595AmcateirvtiSty.sqlite}  & Datenspeicher für Activity Stream, eine personalisierte Übersicht über Browser-Aktivitäten beim Öffnen eines neuen Tabs. &  &  &  \\ \cline{1-2}
\textit{3870112724rsegmnoittet-es.sqlite} & Datenspeicher für Remote Settings, eine zentrale Verwaltung von benutzerspezifischen Browsereinstellungen.               &  &  &  \\ \cline{1-2}
\end{tabular}
}
\end{table}

Jede dieser Datenbanken wurde in allen vier Snapshots miteinander verglichen. Die Dateiextraktion und Dateianalyse erfolgte analog zur Methodik in Kapitel X (TODO!).
Die Ergebnisse wurden in Tabelle X (TODO!) dargestellt.

\begin{figure}[h!]
	\centerline{\resizebox{\linewidth}{!}{\includegraphics{bilder/firefox-sqlite-table.png}}}
	\label{chart:final-criteria}  
	\caption{Comparison of found PB artifacts between RAM Dumps}
\end{figure}
Nach Browser-Installation (Snapshot 1) existierte noch keine der SQLite-Dateien.

Nach dem Browsing Szenario (Snapshot 2) wurde festgestellt, dass alle SQLite-Datenbanken 
initialisiert wurden, außer \texttt{webappsstore.sqlite}. Dabei wurden in \texttt{places.sqlite} die automatisch im normalen Modus geöffnete Datenschutzhinweise Seite eingetragen. 
In restlichen Datenbanken wurden leer initialisiert, nur die Spaltennamen wurden eingetragen.
Der Inhalt aller erstellten Datenbanken blieb nach Durchführung von PRAGMA WAL Checkpoints unverändert.

Nach Schließen des Browsers (Snapshot 3) wurden in \texttt{places.sqlite} die Indizes bei eingetragenen Seiten aktualisiert. Die SQLite-Datenbank \texttt{1657114595AmcateirvtiSty.sqlite} erhielt ein binäres Datenobjekt als Eintrag. Bei der Untersuchung mit HxD konnten keine Artefakte gefunden werden. Weiterhin wurde \texttt{webappsstore.sqlite} leer initialisiert. Die restlichen Daten blieben im Vergleich mit Snapshot 2 unverändert. Ebenfalls veränderte sich nicht der Inhalt nach Durchführung von PRAGMA WAL Checkpoints.

Nach herunterfahren der VM (Snapshot 4) gab es keine Änderungen in den SQLite Datenbanken, auch nach Durchführung der PRAGMA WAL Checkpoints.
	
Somit wurden in den SQLite Datenbanken von Firefox keine zurückverfolgbaren PB Artefakte im privaten Modus hinterlassen.


Mithilfe des Process Monitors wurde festgestellt, dass sowohl während des Browsing Szenarios (Logfile 1) als auch danach (Logfile 2) Inhalte in Dateien geschrieben wurden. Wie zusammenfassend in Abbildung X (TODO!) dargestellt, wurde mit Ausnahme der Datareporting Dateien gab es in Logfile 1 stets mehr oder genauso viele Schreiboperationen wie in Logfile 2.
Keine Schreiboperation hinterließ jedoch Private Browsing Artefakte.
\begin{figure}[h!]
	\centerline{\resizebox{\linewidth}{!}{\includegraphics{bilder/bar-chart-logfile1vs2-test.png}}}
	\label{chart:final-criteria}  
	\caption{Comparison of found PB artifacts between RAM Dumps}
\end{figure}


%Literatur:
%	o no traces were found in “common locations” \cite{Montasari.2015}
%		>  “places.sqlite”, “webappsstore. sqlite”, “sessionstore.bak”, “search.json” and “nssckbi.dll”
%	o	Safebrowsing: Alle Dateien in /safebrowsing-updating/ nicht relevant. Dort nur .vlpset und .sbstore Dateien. Speichern 256-Bit Hash von URLs, die auf SafeSearch Blacklist stehen 
%	o	Cache-Dateien: drei Caches: startupCache, jumpListCache (beide enthalten Binärdateien ohne Browsing Artefakte) und cache2 (können mit MozillaCacheView untersucht werden, enthalten keine Browsing Artefakte)
%	o	SQLite Datenbanken: Sqlite Dateien erst ohne WAL Dateien untersuchen, Danach mit sqlite3 Konsole: WAL in Datenbank schreiben mit: PRAGMA wal\_checkpoint; places.sqlite besonders relevant, da dort Browser in public Modus Browsing URLs verwaltet (Am besten hier vergleich mit Public Browsing machen)	
%		> \cite{Fayyad.2021} for Mozilla Firefox, 7 database files were recovered: cookies.sqlite-shm, places.sqlite-shm, prefs.js etc.
%		> \cite{Muir.2019} The two SQLite databases used by Firefox to track cookies and history (cookies.sqlite und places.sqlite) were both recoverable from the file system after deletion	
%		Ergebnisse stehen im Gegensatz zu \cite{Hedberg.2013} :
%			o	Chrome und Firefox: Einträge in places.sqlite + history.sqlite DB gefunden während PB! (Noch aktuell??)
%		Sonderfall: SQlite DB-Crash \cite{Hedberg.2013}
%			> WAL Files/Journal Files bei Crash gefunden -> Kann genutzt werden um zu beweisen, dass privater Browser genutzt wurde
%			> Daher: WAL Rollback mit sqlite3	
%	o	Jsonlz4 \& balkz4: Enthalten komprimierte Firefox-Sessions, jsonlz4 Dateien können mit Tool "entkomprimiert" werden: https://www.jeffersonscher.com/ffu/scrounger.html

\subsection*{Uncommon Locations}
= Verzeichnisse oder Orte im Dateisystem, die nicht den gängigen Speicherorten der Browser entsprechen und nur durch forensische Analyse entdeckt werden können.
- Möglich ohne vorheriges Wissen über das Browserverhalten oder Vorverarbeitung der Dateien. 
- Forensiker nutzen vordefinierte Funktionen von Forensik-Tools und verlassen sich darauf, dass die Speicherabbilder komplett untersucht werden
- wichtigster Unterschied zu "Common Locations": Umgekehrte Suchrichtung => PB Artefakt (String) -> Alle Daten nach PB Artefakt durchsuchen

Hier: Analyse mit Autopsy und Volatility

\subsubsection*{Analyse mit Autopsy}
Bei Common Locations (Kapitel X): Autopsy nur zur Dateiextraktion genutzt, hier: als konkretes forensisches Werkzeug

1. Stichwortsuche: In allen Snapshots keine Treffer, selbst innerhalb gelöschter Dateien (\$Carved)

2. Untersuchung der von Autopsy automatisch indexierte Dateien: 
In allen Fällen: keine Dateien gelöscht, nur über Zeitraum der Snapshots neue dazugekommen
- Web Bookmarks:
	\begin{figure}[h!]
		\centerline{\resizebox{\linewidth}{!}{\includegraphics{bilder/cfv_firefox_autopsy_web_bookmarks.png}}}
		\label{chart:final-criteria}  
		\caption{Autopsy Web Bookmarks}
	\end{figure}
	Snapshot 1:
		> Bing.url (Unter C:/User/Forensik/Favorites/Links) enthält Bing Startseite
	Snapshot 2:
		> 5 Einträge in places.sqlite: (Firefox Standardseiten -> Deckt sich mit Beobachtungen aus Process Monitor Analyse, siehe Kapitel X)
	Snapshot 3:
		> unverändert zu 2
	Snapshot 4:
		> unverändert zu 3
- Web Cookies:
	\begin{figure}[h!]
		\centerline{\resizebox{\linewidth}{!}{\includegraphics{bilder/cfv_firefox_autopsy_web_cookies.png}}}
		\label{chart:final-criteria}  
		\caption{Autopsy Web Cookies}
	\end{figure}
	Snapshot 1:
		> 10 Einträge in WebCacheV01.dat (= DB des Internet Explorers zum speichern von Browserdaten): Cookies für bing.com und live.com (= outlook)
	Snapshot 2:
		> unverändert zu 1
	Snapshot 3:
		> unverändert zu 2
	Snapshot 4:
		> unverändert zu 3
- Web History:
	\begin{figure}[h!]
		\centerline{\resizebox{\linewidth}{!}{\includegraphics{bilder/cfv_firefox_autopsy_web_history.png}}}
		\label{chart:final-criteria}  
		\caption{Autopsy Web History}
	\end{figure}
	Snapshot 1:
		> 2 Einträge in WebCacheV01.dat:
			- 2x live.com (= outlook)
	Snapshot 2:
		> 1 Eintrag in places.sqlite: % https://www.mozilla.org/privacy/firefox/
			-> Zurückzuführen auf Seite, die sich automatisch geöffnet hat, als Firefox gestartet (bevor privates Fenster geöffnet wurde)
		> 1 neuer Einträge in WebCacheV01.dat:
			- file:///Z:/Logfile\_1 (= Process Monitor Logfile, die in shared-Folder geladen wurde) -> Erklärung?
	Snapshot 3:
		> 1 neuer Eintrag in WebCacheV01.dat:
			- file:///Z:/Logfile\_2 (= Process Monitor Logfile, die in shared-Folder geladen wurde) -> Erklärung?
	Snapshot 4:
		> unverändert zu 3
- Web Categories:
	\begin{figure}[h!]
		\centerline{\resizebox{\linewidth}{!}{\includegraphics{bilder/cfv_firefox_autopsy_web_categories.png}}}
		\label{chart:final-criteria}  
		\caption{Autopsy Web Categories}
	\end{figure}
	Snapshot 1:
		> 2x WebCacheV01.dat aufgelistet => Mit HxD untersucht, keine PB Artefakte
	Snapshot 2:
		> unverändert zu 2
	Snapshot 3:
		> unverändert zu 3
	Snapshot 4:
		> unverändert zu 4
		
Zusammenfassung:
- keine PB Artefakte
- Keine neuen Erkenntnisse vgl. mit intensiver Analyse mittels Process Monitor in Kapitel X
- Eintrag von Datenschutzseite in places.sqlite wurde erkannt.

%Literatur:
%	o	Autopsy Keywortsuche: 
%		>	In alles Snapshots ergebnislos (keine Keyword-Hits
%		-->	In Literatur: Autoren fanden Ergebnisse in pagefile.sys 
%			> Autopsy: websites and some of the keywords found in hidden file called “pagefile.sys” \cite{Mahlous.2020}
%			o \cite{Montasari.2015} traces were found in: 
%				> However, on investigating the “pagefile.sys”, some entries were discovered
%				> Using the “data carving” technique, profile picture was recovered
%			o \cite{Said.2011} 
%				> Examining pagefile.sys showed some positive hits 			
%		--> Evtl. hier zeigen, was gefunden werden kann, wenn RAM reduziert
%		--> Aber auf Problem hinweisen, dass gefundener String in pagefile nicht direkt Browser zugeordnet werden kann
%		> \cite{Gabet.2018}	Firefox only produced three recoverable artefacts as reported by both tools (FTK, Autopsy) --> Artefakte werden nicht genannt!
%		> \cite{Muir.2019} Autopsy Keyword Suche nach Suchbegriffen: unallocated space
%		> Autopsy Carving Module (\$Carved): \cite{Muir.2019}
%			•	When searching for the string ’clot’ from the browsing protocol, six .dll, .edb and .reg files were discovered in unallocated space.
%			•	Further searching of unallocated space uncovered references to the Tor installation directory and the obfs4 bridging IP addresses
%			•	browsing data found in NTUSER.DAT was also replicated in unallocated space.
%	o	Autopsy PlugIns:
%		>	*** TODO: Hier Liste mit PlugIns ***

\subsubsection*{Analyse mit Volatility}
Vorgehen: Siehe "Methodik" Kapitel
Für jeden Yarascan-Treffer: Prozessnamen und ggf. String-Kontext ermittelt

HTML Artefakte wurden in keinem RAM Dump gefunden => Nicht aufgeführt

Yararule "Keyword":
	\begin{figure}[h!]
		\centerline{\resizebox{\linewidth}{!}{\includegraphics{bilder/volatility/firefox/keywords.png}}}
		\label{chart:final-criteria}  
		\caption{Keywords}
	\end{figure}
	Analyse:
		> Ausschließlich in RAM Dump 2 Keyword Artefakte gefunden
		> Hauptsächlich in Firefox Prozess
		> Mit 1301 Artefakten, am häufigsten "pfaffenhofen" vertreten. Vermutung: Evtl. weil Google Maps viele zusätzliche Artefakte lädt. 
		
Yararule "URL":
	\begin{figure}[h!]
		\centerline{\resizebox{\linewidth}{!}{\includegraphics{bilder/volatility/firefox/url.png}}}
		\label{chart:final-criteria}  
		\caption{URL}
	\end{figure}
	Analyse:
		> Wie bei anderen Kategorien: Die meisten Artefakte in RAM Dump 2, in Firefox Prozessen
		> mooserliesl tritt am wenigsten auf, donaukurier am meisten (vmtl. auf Öffnen von Bild zurückzuführen)
		> Hier bemerkenswert, dass in RAM Dump 3 Artefakte von allen vier URLs zu finden sind
		> Bei genauerer Analyse des Process Monitor Logfiles herausgefunden: Artefakte alle in svchost.exe Prozess gefunden
		> Deshalb RAM Dump erneut mit Volatility windows.svcscan Plugin untersucht:
			"The svcscan plugin allows the analyst to list out the services running. This plugin gives more detail to the running processes in the event that the analyst requires additional details such as the display name, binary path, or service type." % (https://www.oreilly.com/library/view/digital-forensics-and/9781787288683/9ab60586-2b04-45e0-b437-dbfe10ab3be8.xhtml)
		> Ausgabe aller im RAM gefundener Services
		> Problem: Volatility svcscan liefert keine PID zu laufenden Services
		> Deshalb: "White-Box" Analyse: Snapshot 3 erneut aufgetaut, danach mit Process Explorer PID X (TODO!) von SVChost Prozess gesucht, in dem PB Artefakte gefunden wurden
			Def. Process Explorer:
				"Prozess Explorer zeigt Ihnen Informationen darüber an, welche Handles und DLLs-Prozesse geöffnet oder geladen wurden." % https://learn.microsoft.com/de-de/sysinternals/downloads/process-explorer
				"Process Explorer, from Sysinternals, is a process management program that allows you to see the running processes on your computer and a great deal of information about each process. One of the nice features of Process Explorer is that it also gives you the ability to see what services a particular SVCHOST.EXE process is controlling."
		> Ergebnis: DNSCache Service mit PID X (TODO!) = DNScache Service
			TODO: Screenshot
		> Ausführung von ipconfig /displaydns liefert gecachte URLs
			TODO: Screenshot
		> Nach Löschen des DNSCaches mit ipconfig /flushdns + Schließen aller Process Monitor Instanzen + Beenden des DNSCaches Services + Erneuter RAM-Dump -> Keine PB Artefakte mehr gefunden!
Yararule "Mail":
	\begin{figure}[h!]
		\centerline{\resizebox{\linewidth}{!}{\includegraphics{bilder/volatility/firefox/mail.png}}}
		\label{chart:final-criteria}  
		\caption{Mail}
	\end{figure}
	Analyse:
		> Alle Mail Artefakte gefunden
		> Ausschließlich in RAM Dump 2 Mail Artefakte gefunden
		> Am häufigsten Absenderadresse "computerforensikvl@gmail.com" gefunden, als einziges Artefakt auch in anderen Prozessen gefunden.
		> Bemerkenswert: Passwort wurde 4x als Klartext im RAM gefunden!
			String Kontext:
				Offsets:		PIDs:
				0xb9ce29180c8	7420
				0x2859f4ffd4e0	7420
				0x24083b41858	8424
				0x240840e5b08	8424
				
			Memmap: Pid 7420
				virtual			physical	size	offset in file
				0xb9ce2918000	0xcb20a000	0x1000	0x11dd4000
					-> 0xb9ce29180c8 = 0x11dd40c8 
					\begin{figure}[h!]
						\centerline{\resizebox{0.7\linewidth}{!}{\includegraphics{bilder/volatility/firefox/password_0xb9ce29180c8_7420.png}}}
						\label{chart:final-criteria}  
						\caption{Password in memory page of PID 7420 at offset 0xb9ce29180c8}
					\end{figure}
				0x2859f4ffd000	0x96812000	0x1000	0x12e23000	Disabled				
					-> 0x2859f4ffd4e0 = 0x12e234e0
					\begin{figure}[h!]
						\centerline{\resizebox{0.7\linewidth}{!}{\includegraphics{bilder/volatility/firefox/password_0x2859f4ffd4e0_7420.png}}}
						\label{chart:final-criteria}  
						\caption{Password in memory page of PID 7420 at offset 0x2859f4ffd4e0}
					\end{figure}
			Memmap: Pid 8424
				virtual			physical	size	offset in file
				0x24083b41000	0xc1d52000	0x1000	0x583000	Disabled
					-> 0x24083b41858 = 0x583858
					\begin{figure}[h!]
						\centerline{\resizebox{0.7\linewidth}{!}{\includegraphics{bilder/volatility/firefox/password_0x24083b41858_8424.png}}}
						\label{chart:final-criteria}  
						\caption{Password in memory page of PID 8424 at offset 0x24083b41858}
					\end{figure}
				0x240840e5000	0x2d3fb000	0x1000	0x96b000	Disabled
					-> 0x240840e5b08 = 0x96bb08
					\begin{figure}[h!]
						\centerline{\resizebox{0.7\linewidth}{!}{\includegraphics{bilder/volatility/firefox/password_0x240840e5b08_8424.png}}}
						\label{chart:final-criteria}  
						\caption{Password in memory page of PID 8424 at offset 0x240840e5b08}
					\end{figure}
		> In PID 8424: 2 Bytes pro Character, bspw. Unicode
				
Yararule "Image":
	\begin{figure}[h!]
		\centerline{\resizebox{\linewidth}{!}{\includegraphics{bilder/volatility/firefox/image.png}}}
		\label{chart:final-criteria}  
		\caption{Image}
	\end{figure}
	Analyse:
		> Hex-Wert von Donaukurier Bild wurde im 2. RAM Dump in 3 Firefox Prozessen gefunden

Zusammenfassung = Stacked Bar Chart:
\begin{figure}[h!]
	\centerline{\resizebox{\linewidth}{!}{\includegraphics{bilder/volatility/firefox/summary.png}}}
	\label{chart:final-criteria}  
	\caption{Summary}
\end{figure}

TODO: Kreisdiagramme/Balkendiagramme mit Gesamtzahl an (Non-)Firefox Yarascan-Treffer erst im Vergleich mit Tor

\subsection*{Registry}
> Process Monitor: SetValue Operationen von Browser 
TODO: Logfile 1 vs 2?
	Kategorien Registry Keys:
	1) PreXULSkeletonUISettings:
		> Prefix: Absoluter Installationspfad von Firefox
		> Skeleton UI Einstellungen von Firefox % https://itigic.com/skeleton-ui-new-firefox-interface-to-start-up-much-faster/#google_vignette
			Definition:
				> Der "PreXULSkeletonUISettings" Registry Key enthielt Einstellungen für die Benutzeroberfläche (UI) des Firefox-Browsers, insbesondere für das sogenannte "Skeleton UI". Das Skeleton UI ist eine vereinfachte Benutzeroberfläche, die während des Ladens des Browsers angezeigt wird, bevor die vollständige Benutzeroberfläche geladen ist. Es besteht aus grundlegenden Steuerelementen und Elementen, die dem Benutzer die Interaktion ermöglichen, während der Rest der Benutzeroberfläche noch geladen wird.
				> Der "PreXULSkeletonUISettings"-Schlüssel enthielt Konfigurationsoptionen wie Farben, Positionen und andere Einstellungen für das Skeleton UI. Durch das Bearbeiten dieses Schlüssels konnten Benutzer die Darstellung des Skeleton UI anpassen. Es ist jedoch wichtig zu beachten, dass das Ändern der Registrierungseinträge ein fortgeschrittenes Verfahren ist und Fehler zu Problemen mit dem Browser führen kann.
			
		> Struktur der Keys: % HKCU\SOFTWARE\Mozilla\Firefox\PreXULSkeletonUISettings\C:\Program Files\Mozilla Firefox\firefox.exe|<UI Einstellung>
		> Unterschiedliche UI Einstellungen
			- % ScreenX (DWORD)
			- % ScreenY (DWORD)
			- % Width (DWORD)
			- % Height (DWORD)
			- % Maximized (DWORD)
			- % Flags (DWORD)
			- % CssToDevPixelScaling (REG_BINARY)
			- % UrlbarCSSSpan (REG_BINARY)
			- % SearchbarCSSSpan (REG_BINARY)
			- % SpringsCSSSpan (REG_BINARY)
		> keine PB Artefakte unter UI Einstellungen	
	2) Business Activity Monitoring % https://learn.microsoft.com/de-de/biztalk/core/business-activity-monitoring-bam
		> Quelle: % https://notes.qazeer.io/dfir/windows/_artefacts_overview
		> BAM is a mostly undocumented feature that controls the programs executed in the background. DAM is a feature for devices supporting the "Connected Standby" mode (i.e when a device is turned on, but its display will be turned off). As a result, the BAM registry keys will contain data on any devices, while DAM registry keys will only contain data on mobile devices.
		> The BAM registry key contains multiple subkeys under bam\\State\\UserSettings, with one subkey per user, identified with the user SID. While the key is in the SYSTEM registry hive, program executions can thus still be tied to a specific user using this SID.
		> Each user-specific key contains a list of executed programs, with their full path and timestamp of last execution.
		> If a file is deleted, the eventual associated entry in the BAM is deleted as well after the system reboot. Additionally, BAM entries older than 7 days are deleted upon system boot. The BAM thus provides limited information on historic execution of programs
		> No entries are created in the BAM keys for executables on removable media and/or on network shares.
		> Key: %  HKLM\System\CurrentControlSet\Services\bam\State\UserSettings\S-1-5-21-588412547-2749917301-3803556669-1001\\Device\HarddiskVolume2\Program Files\Mozilla Firefox\firefox.exe (REG_BINARY)

Quantitativ: (Diagramme)
	- Stacked Balkendiagramm jeweils für Logfile 1 und Logfile2: Anteil Kategorie 1 bzw.2 an allen Registry-Schreiboperationen
	\begin{figure}[h!]
		\centerline{\resizebox{\linewidth}{!}{\includegraphics{bilder/firefox-registry-stacked-bar-chart.png}}}
		\label{chart:final-criteria}  
		\caption{Comparison of found PB artifacts between RAM Dumps}
	\end{figure}
	
> Stringsuche in Registry Hives mit Registry Explorer (Siehe Liste)
	In allen Hives kein Treffer für alle Suchbegriffe

Literatur: 
	> angeblich in Shellactivities Ergebnisse. --> Nicht mehr vorhanden in aktueller Version (Verweis auf E-Mail)

%Literatur:
%	>	Auf Autor verweisen: angeblich in Shellactivities Ergebnisse. --> Nicht mehr vorhanden in aktueller Version (Verweis auf E-Mail)
%	>	Process Monitor/Regshot zeigen keine relevanten Key-Änderungen
%	> \cite{Muir.2019}: Autopsy Keyword Suche nach Suchbegriffen: Ergebnisse in \%SystemRoot\%Minidump NTUSER.DAT, ntuser.dat.LOG1 (a log of changes to NTUSER.DAT)
%	> Zentral: shellactivites Key:	NTUSER.DAT --> “shellactivities” key \cite{Muir.2019}
%	> \cite{Rochmadi.2017} Detection of registry changes helps to determine what the appropriate plugin is used to search for digital evidence using volatility memory forensic:
%	- RegQueryValue:	HKCU/Software/Microsoft/Windows/CurrentVersion/InternetSettings/Connections/DefaultConnectionSettings
%	- RegCloseValue: 	HKCU/Software/Microsoft/Windows/CurrentVersion/InternetSettings/Connections
%	- IRP\_MJ\_READ: C:/pagefile.sys

\newpage


% ######################################################################
% ######################################################################
% ######################################################################
% ######################################################################


\section{Tor}

\subsection*{White-Box Analyse/Common Locations}

Schreiboperationen mit Process Monitor verfolgen:

Im Anhang: Tabelle mit allen geschriebenen Dateien (markiert, wenn nicht mehr wiederherstellbar + markiert, wenn Datei "verändert" (siehe oben: temp, WAL))

Aux-Dateien, welche nicht mehr vorhanden waren, aber dafür "richtige" Dateien:

Ergebnis: Tabelle mit wiederherstellbaren Dateien: Logfile 1 vs. Logfile 2 + Tool mit dem Datei untersucht wurde
- Dateien, die in beiden Logfiles nicht wiederherstellbar 
\begin{figure}[h!]
	\resizebox{\linewidth}{!}{\includegraphics{bilder/tor-tabelle-logfile1vlogfile2vlogfile3-reduced.png}}
%	\label{...}
	\caption{Tabelle mit wiederherstellbaren Dateien: Logfile 1 vs. Logfile 2}
\end{figure}

Allgemein: Tor hat nur einen "Common Pfad"
-	% C:\Users\Forensik\Desktop\Tor Browser\Browser\TorBrowser\Data\Browser\
- Dateien tauchen in zwei unterschiedlichen Ordnern auf:
	- % C:\Users\Forensik\Desktop\Tor Browser\Browser\TorBrowser\Data\Browser\Caches\profile.default\ (Caches)
	- % C:\Users\Forensik\Desktop\Tor Browser\Browser\TorBrowser\Data\Browser\profile.default\ (Profile.Default)

- Alle Schreibopertationen von Prozess "firefox.exe" durchgeführt, nicht "tor.exe" 

=> Keine der Dateien enthält PB Artefakte, trotzdem nachfolgende genauere Betrachtung der wichtigsten Dateien im Zusammenhang des Tor Browsers

Kategorien der Logs:
- Cache: 
	> % \Caches\profile.default\startupCache\startupCache.8.little (Caches Folder)
	Zweck:
		"Die Datei "startupCache.8.little" ist eine interne Datei, die von Firefox und dem Tor Browser erstellt wird, um den Startvorgang des Browsers zu beschleunigen. Sie enthält im Wesentlichen eine Zwischenspeicherung von Daten, die beim Starten des Browsers benötigt werden.

		Diese Datei enthält Informationen über bereits geladene Browser-Komponenten wie JavaScript-Code, CSS-Dateien, Bilder und andere Ressourcen. Indem der Browser diese Informationen zwischenspeichert, kann er sie beim erneuten Starten des Browsers wiederverwenden, anstatt sie erneut herunterladen und verarbeiten zu müssen. Dadurch wird die Startzeit des Browsers verkürzt und die allgemeine Leistung verbessert." %https://wiki.mozilla.org/StartupCache
	Analyse:
		- Tool: HxD
		- kein PB Artefakte

- datareporting:
	> % *\datareporting\state.json 
	Zweck: 
		"Die Datei "state.json" im Ordner "/datareporting" enthält Informationen über den Zustand und die Konfiguration des Firefox- oder Tor Browsers. Diese Datei kann Daten über die Verwendung des Browsers, wie z.B. installierte Add-Ons, zuletzt besuchte Websites, Browser-Einstellungen und andere Informationen enthalten. Sie wird verwendet, um dem Browser bei Bedarf den Zustand und die Einstellungen wiederherzustellen."
		% https://github.com/mozilla/firefox-data-store-docs/blob/master/README.md
	Analyse:
		- Tool Notepad++ mit JSON Plugin
		- keine PB Artefakte

- Sonstige Dateien:
	> % \AlternateServices.txt
		Zweck:
			"enthält onion URLs,
			HTTP Alternative Services is a mechanism that allows servers to tell clients that the service they are accessing is available at another network location or over another protocol.
			This mapping can be stored in a file in the profile folder. This allows websites that do not support HTTPS to communicate in a secure way via port 443 (Opportunistic Encryption)."
			% https://support.mozilla.org/en-US/questions/1310302
	> % \extensions\staged\{73a6fe31-595d-460b-a920-fcc0f8843232}.xpi
		Zweck:
			"Ist "NoScript" Extension. Wenn in Firefox geöffnet, kann installiert werden"
		=> TODO: Screenshot, wenn in Firefox per "drag-and-drop" gezogen
	> % *\onion-aliases.json
		Zweck:
			Enthält SecureDrop Adressen: z.B. sueddeutsche.securedrop.tor.onion (z.B. %https://www.sueddeutsche.de/projekte/kontakt/#securedrop)
	> % \security_state\data.safe.bin
		Zweck: The file containing the updated security data % (https://bbs.archlinux.org/viewtopic.php?pid=1952286#p1952286)
	> % \SiteSecurityServiceState.txt
		Entielt früher private Browsing Artefakte (https://gitlab.torproject.org/tpo/applications/tor-browser/-/issues/18589), jetzt aber keine private Browsing Artefakte
	=> Keine der Dateien enthält PB Artefakte
				
- SQLite: 
	Aus Process Monitor Logfiles erkennbar: Tor verwaltet und beschreibt die exakt gleichen SQLite Datenbanken wie Firefox.

	Hier ebenfalls gesondert betrachtet: Fokus auf die Entwicklung von Dateiinhalt in allen Snapshots (1, 2, 3-1, 3 und 4) betrachtet
	
	Ergebnisse:
		\begin{figure}[h!]
			\centerline{\resizebox{\linewidth}{!}{\includegraphics{bilder/tor-sqlite-table.png}}}
			\label{chart:final-criteria}  
			\caption{Comparison of found PB artifacts between RAM Dumps}
		\end{figure}
		> Nach Browser-Installation noch keine SQLite-Datei angelegt (Snapshot 1)
		> Während Browsing Szenario alle DBs Initialisiert, außer "webappsstore.sqlite" (Snapshot 2)
			- Dabei wurden in places.sqlite automatisch .onion URLs geschreiben, die zu Tor Standardseiten führen, wie "The Tor Blog" oder "Tor Browser Manual" bzw. die Tor Spenden-Seite, obwohl keine dieser Seiten aufgerufen wurde
				TODO: Screenshot von URLs?
			- in Favicons.sqlite wurden die exakt gleichen Enträge geschrieben, mit dem Präfix "Fake-favicon-uri". Ein tatsächliches Icon wurde nicht in die DB geschrieben
			- remote settings Datenbank enthielt den gleichen Eintrag wie es bereits bei Firefox der Fall war. Keine PB Artefakte
			- Restliche Dateien ohne Inhalt, nur Spaltennamen
			- Nach WAL Checkpoints bleiben Dateien unverändert
		> Nach Zurücksetzen der Browser-Identität (Snapshot 3-1)
			- in places.sqlite: Indizes bei eingetragenen Seiten aktualisiert
			- restliche Dateien unverändert
		> Nach Schließen des Browsers (Snapshot 3)
			- in places.sqlite sowie favicons.sqlite: Indizes bei eingetragenen Seiten aktualisiert
			- restliche Dateien unverändert
			- nach WAL Checkpoints bleiben Dateien unverändert
		> Nach herunterfahren der VM (Snapshot 4)
			- Alle Dateien unverändert, auch nach WAL Checkpoint
	
- Zusammenfassung: in keiner Datei PB Artefakte


Quantitativ: (Diagramme)		
	> Balkendiagramm: Für jede Logfilekategorie: Anzahl Schreiboperationen Logfile 1 vs Logfile 2
	\begin{figure}[h!]
		\centerline{\resizebox{\linewidth}{!}{\includegraphics{bilder/tor-bar-chart-logfile1vs2cs3.png}}}
		\label{chart:final-criteria}  
		\caption{Comparison of found PB artifacts between RAM Dumps}
	\end{figure}

%Literatur:
%	o no traces were found in “common locations” \cite{Montasari.2015}
%		>  “places.sqlite”, “webappsstore. sqlite”, “sessionstore.bak”, “search.json” and “nssckbi.dll”
%	o	Safebrowsing: Alle Dateien in /safebrowsing-updating/ nicht relevant. Dort nur .vlpset und .sbstore Dateien. Speichern 256-Bit Hash von URLs, die auf SafeSearch Blacklist stehen 
%	o	Cache-Dateien: drei Caches: startupCache, jumpListCache (beide enthalten Binärdateien ohne Browsing Artefakte) und cache2 (können mit MozillaCacheView untersucht werden, enthalten keine Browsing Artefakte)
%	o	SQLite Datenbanken: Sqlite Dateien erst ohne WAL Dateien untersuchen, Danach mit sqlite3 Konsole: WAL in Datenbank schreiben mit: PRAGMA wal\_checkpoint; places.sqlite besonders relevant, da dort Browser in public Modus Browsing URLs verwaltet (Am besten hier vergleich mit Public Browsing machen)	
%		> \cite{Fayyad.2021} for Mozilla Firefox, 7 database files were recovered: cookies.sqlite-shm, places.sqlite-shm, prefs.js etc.
%		> \cite{Muir.2019} The two SQLite databases used by Firefox to track cookies and history (cookies.sqlite und places.sqlite) were both recoverable from the file system after deletion	
%		Ergebnisse stehen im Gegensatz zu \cite{Hedberg.2013} :
%			o	Chrome und Firefox: Einträge in places.sqlite + history.sqlite DB gefunden während PB! (Noch aktuell??)
%		Sonderfall: SQlite DB-Crash \cite{Hedberg.2013}
%			> WAL Files/Journal Files bei Crash gefunden -> Kann genutzt werden um zu beweisen, dass privater Browser genutzt wurde
%			> Daher: WAL Rollback mit sqlite3	
%	o	Jsonlz4 \& balkz4: Enthalten komprimierte Firefox-Sessions, jsonlz4 Dateien können mit Tool "entkomprimiert" werden: https://www.jeffersonscher.com/ffu/scrounger.html


\subsection*{Registry}
> Process Monitor: SetValue Operationen von Browser 
	Kategorien Registry Keys: Analog zu Firefox
	1) PreXULSkeletonUISettings:
		> Prefix: Absoluter Installationspfad von Firefox
		> Skeleton UI Einstellungen von Firefox % https://itigic.com/skeleton-ui-new-firefox-interface-to-start-up-much-faster/#google_vignette
			Definition:
				> Der "PreXULSkeletonUISettings" Registry Key enthielt Einstellungen für die Benutzeroberfläche (UI) des Firefox-Browsers, insbesondere für das sogenannte "Skeleton UI". Das Skeleton UI ist eine vereinfachte Benutzeroberfläche, die während des Ladens des Browsers angezeigt wird, bevor die vollständige Benutzeroberfläche geladen ist. Es besteht aus grundlegenden Steuerelementen und Elementen, die dem Benutzer die Interaktion ermöglichen, während der Rest der Benutzeroberfläche noch geladen wird.
				> Der "PreXULSkeletonUISettings"-Schlüssel enthielt Konfigurationsoptionen wie Farben, Positionen und andere Einstellungen für das Skeleton UI. Durch das Bearbeiten dieses Schlüssels konnten Benutzer die Darstellung des Skeleton UI anpassen. Es ist jedoch wichtig zu beachten, dass das Ändern der Registrierungseinträge ein fortgeschrittenes Verfahren ist und Fehler zu Problemen mit dem Browser führen kann.
			
		> Struktur der Keys: % HKCU\SOFTWARE\Mozilla\Firefox\PreXULSkeletonUISettings\C:\Program Files\Mozilla Firefox\firefox.exe|<UI Einstellung>
		> Unterschiedliche UI Einstellungen
			- % ScreenX (DWORD)
			- % ScreenY (DWORD)
			- % Width (DWORD)
			- % Height (DWORD)
			- % Maximized (DWORD)
			- % Flags (DWORD)
			- % CssToDevPixelScaling (REG_BINARY)
			- % UrlbarCSSSpan (REG_BINARY)
			- % SearchbarCSSSpan (REG_BINARY)
			- % SpringsCSSSpan (REG_BINARY)
		> keine PB Artefakte unter UI Einstellungen	
	2) Business Activity Monitoring % https://learn.microsoft.com/de-de/biztalk/core/business-activity-monitoring-bam
		> Quelle: % https://notes.qazeer.io/dfir/windows/_artefacts_overview
		> BAM is a mostly undocumented feature that controls the programs executed in the background. DAM is a feature for devices supporting the "Connected Standby" mode (i.e when a device is turned on, but its display will be turned off). As a result, the BAM registry keys will contain data on any devices, while DAM registry keys will only contain data on mobile devices.
		> The BAM registry key contains multiple subkeys under bam\\State\\UserSettings, with one subkey per user, identified with the user SID. While the key is in the SYSTEM registry hive, program executions can thus still be tied to a specific user using this SID.
		> Each user-specific key contains a list of executed programs, with their full path and timestamp of last execution.
		> If a file is deleted, the eventual associated entry in the BAM is deleted as well after the system reboot. Additionally, BAM entries older than 7 days are deleted upon system boot. The BAM thus provides limited information on historic execution of programs
		> No entries are created in the BAM keys for executables on removable media and/or on network shares.
		> Key: %  HKLM\System\CurrentControlSet\Services\bam\State\UserSettings\S-1-5-21-588412547-2749917301-3803556669-1001\\Device\HarddiskVolume2\Program Files\Mozilla Firefox\firefox.exe (REG_BINARY)

Quantitativ: (Diagramme)
	- Stacked Balkendiagramm jeweils für Logfile 1 und Logfile2: Anteil Kategorie 1 bzw.2 an allen Registry-Schreiboperationen
	\begin{figure}[h!]
		\centerline{\resizebox{\linewidth}{!}{\includegraphics{bilder/tor-registry-stacked-bar-chart.png}}}
		\label{chart:final-criteria}  
		\caption{Comparison of found PB artifacts between RAM Dumps}
	\end{figure}
	
> Stringsuche in Registry Hives mit Registry Explorer (Siehe Liste)
	In allen Hives kein Treffer für alle Suchbegriffe

Literatur: 
	> Wie bei Firefox: Shellactivities Key existiert nicht mehr --> Nicht mehr vorhanden in aktueller Version (Verweis auf E-Mail)

%Literatur:
%	>	Auf Autor verweisen: angeblich in Shellactivities Ergebnisse. --> Nicht mehr vorhanden in aktueller Version (Verweis auf E-Mail)
%	>	Process Monitor/Regshot zeigen keine relevanten Key-Änderungen
%	> \cite{Muir.2019}: Autopsy Keyword Suche nach Suchbegriffen: Ergebnisse in \%SystemRoot\%Minidump NTUSER.DAT, ntuser.dat.LOG1 (a log of changes to NTUSER.DAT)
%	> Zentral: shellactivites Key:	NTUSER.DAT --> “shellactivities” key \cite{Muir.2019}
%	> \cite{Rochmadi.2017} Detection of registry changes helps to determine what the appropriate plugin is used to search for digital evidence using volatility memory forensic:
%	- RegQueryValue:	HKCU/Software/Microsoft/Windows/CurrentVersion/InternetSettings/Connections/DefaultConnectionSettings
%	- RegCloseValue: 	HKCU/Software/Microsoft/Windows/CurrentVersion/InternetSettings/Connections
%	- IRP\_MJ\_READ: C:/pagefile.sys

\subsection*{Black-Box Analyse/Uncommon Locations}

\subsubsection*{Analyse mit Autopsy}
Bei White-Box Analyse/Common Locations: Autopsy nur zur Dateiextraktion genutzt, hier: als konkretes forensisches Werkzeug

Stichwortsuche:
- In allen Snapshots keine Treffer (auch innerhalb \$Carved)
- TODO: Pagefile gefunden?

Von Autopsy automatisch indexierte Dateien: 
In allen Fällen: keine Dateien gelöscht, nur über Zeitraum der Snapshots neue dazugekommen
- Web Bookmarks:
	\begin{figure}[h!]
		\centerline{\resizebox{\linewidth}{!}{\includegraphics{bilder/cfv_tor_autopsy_web_bookmarks.png}}}
		\label{chart:final-criteria}  
		\caption{Autopsy Web Bookmarks}
	\end{figure}
	Snapshot 1:
		> Bing.url (Unter C:/User/Forensik/Favorites/Links) enthält Bing Startseite
	Snapshot 2:
		> unverändert zu 1
	Snapshot 3-1:
		> unverändert zu 2
	Snapshot 3-2:	
		> unverändert zu 3-1
	Snapshot 4:
		> unverändert zu 3-2
- Web Cookies:
	\begin{figure}[h!]
		\centerline{\resizebox{\linewidth}{!}{\includegraphics{bilder/cfv_tor_autopsy_web_cookies.png}}}
		\label{chart:final-criteria}  
		\caption{Autopsy Web Cookies}
	\end{figure}
	Snapshot 1:
		> 9 Einträge in WebCacheV01.dat (= DB des Internet Explorers zum speichern von Browserdaten): Cookies für bing.com und live.com (= outlook)
	Snapshot 2:
		> unverändert zu 1
	Snapshot 3-1:
		> unverändert zu 2
	Snapshot 3-2:
		> unverändert zu 3-1
	Snapshot 4:
		> unverändert zu 3-2
- Web History:
	\begin{figure}[h!]
		\centerline{\resizebox{\linewidth}{!}{\includegraphics{bilder/cfv_tor_autopsy_web_history.png}}}
		\label{chart:final-criteria}  
		\caption{Autopsy Web History}
	\end{figure}
	Snapshot 1:
		> 2 Einträge in WebCacheV01.dat:
			- 2x live.com (= outlook)
	Snapshot 2:
		> 1 neuer Einträge in WebCacheV01.dat:
			- file:///Z:/Logfile\_1 (= Process Monitor Logfile, die in shared-Folder geladen wurde) -> Erklärung?
	Snapshot 3-1:
		> 1 neuer Eintrag in WebCacheV01.dat:
			- file:///Z:/Logfile\_2-1 (= Process Monitor Logfile, die in shared-Folder geladen wurde) -> Erklärung?
	Snapshot 3-2:
		> 1 neuer Eintrag in WebCacheV01.dat:
			- file:///Z:/Logfile\_2-2 (= Process Monitor Logfile, die in shared-Folder geladen wurde) -> Erklärung?
	Snapshot 4:
		> unverändert zu 3-2
- Web Categories:
	\begin{figure}[h!]
		\centerline{\resizebox{\linewidth}{!}{\includegraphics{bilder/cfv_tor_autopsy_web_categories.png}}}
		\label{chart:final-criteria}  
		\caption{Autopsy Web Categories}
	\end{figure}
	Snapshot 1:
		> 2x WebCacheV01.dat aufgelistet => Mit HxD untersucht, keine PB Artefakte
	Snapshot 2:
		> unverändert zu 2
	Snapshot 3-1:
		> unverändert zu 3
	Snapshot 3-2:
		> unverändert zu 3-1
	Snapshot 4:
		> unverändert zu 3-2
		
Zusammenfassung:
- keine PB Artefakte
- Keine neuen Erkenntnisse vgl. mit intensiver Analyse mittels Process Monitor in Kapitel X
- .onion URL Einträge in places.sql nicht erkannt

%Literatur:
%	o	Autopsy Keywortsuche: 
%		>	In alles Snapshots ergebnislos (keine Keyword-Hits
%		-->	In Literatur: Autoren fanden Ergebnisse in pagefile.sys 
%			> Autopsy: websites and some of the keywords found in hidden file called “pagefile.sys” \cite{Mahlous.2020}
%			o \cite{Montasari.2015} traces were found in: 
%				> However, on investigating the “pagefile.sys”, some entries were discovered
%				> Using the “data carving” technique, profile picture was recovered
%			o \cite{Said.2011} 
%				> Examining pagefile.sys showed some positive hits 			
%		--> Evtl. hier zeigen, was gefunden werden kann, wenn RAM reduziert
%		--> Aber auf Problem hinweisen, dass gefundener String in pagefile nicht direkt Browser zugeordnet werden kann
%		> \cite{Gabet.2018}	Firefox only produced three recoverable artefacts as reported by both tools (FTK, Autopsy) --> Artefakte werden nicht genannt!
%		> \cite{Muir.2019} Autopsy Keyword Suche nach Suchbegriffen: unallocated space
%		> Autopsy Carving Module (\$Carved): \cite{Muir.2019}
%			•	When searching for the string ’clot’ from the browsing protocol, six .dll, .edb and .reg files were discovered in unallocated space.
%			•	Further searching of unallocated space uncovered references to the Tor installation directory and the obfs4 bridging IP addresses
%			•	browsing data found in NTUSER.DAT was also replicated in unallocated space.
%	o	Autopsy PlugIns:
%		>	*** TODO: Hier Liste mit PlugIns ***

\subsubsection*{Analyse mit Volatility}
Vorgehen: Siehe "Methodik" Kapitel
	- Ausgangslage: Volatility Yarascan Treffer
	- Für jeden Treffer: virtueller Offset des Strings, PID, getriggerte Yararule, getriggerte Yara Component z(= Variablenname des gesuchten Strings), gefundener String
	- Neue Spalte: "Prozessname" -> zu jeder PID Prozessnamen
	- Ergebnisse Aufbereitet nach folgendem Schema:
		> Für jeden RAM Dump
		> Für jede Yararule
		> Für jede Component
		> Filter: Prozessname = Firefox -> Anzahl zählen
		> Filter: Prozessname = Alle Prozesse außer Firefox -> Anzahl zählen

Wie bei Firefox: HTML Artefakte wurden in keinem RAM Dump gefunden => Nicht aufgeführt

Yararule "Keyword":
	\begin{figure}[h!]
		\centerline{\resizebox{\linewidth}{!}{\includegraphics{bilder/volatility/tor/keywords.png}}}
		\label{chart:final-criteria}  
		\caption{Keywords}
	\end{figure}
	Analyse:
		> Ausschließlich in RAM Dump 2 und RAM Dump 3-1 Keyword Artefakte gefunden
		> In RAM Dump 3-1 bei jedem Keyword deutlich weniger Artefakte als in RAM Dump 2 => Identitäts-Reset reduziert Keyword Artefakte deutlich
		> Hauptsächlich in Firefox Prozess, kein Artefakt in Tor.exe Prozess
		> Mit 4833 Artefakten in RAM Dump 2 am häufigsten "pfaffenhofen" vertreten. Vermutung: Evtl. weil Google Maps viele zusätzliche Artefakte lädt. 
		> Nach Schließen von Tor Browser: keine Keyword Artefakte mehr in RAM
		
Yararule "URL":
	\begin{figure}[h!]
		\centerline{\resizebox{\linewidth}{!}{\includegraphics{bilder/volatility/tor/url.png}}}
		\label{chart:final-criteria}  
		\caption{URL}
	\end{figure}
	Analyse:
		> Wie bei Yararule "Keyword": Ausschließlich in RAM Dump 2 und RAM Dump 3-1 Keyword Artefakte gefunden
		> In RAM Dump 3-1 bei jedem Keyword deutlich weniger Artefakte als in RAM Dump 2 => Identitäts-Reset reduziert URL Artefakte deutlich
		> Hauptsächlich in Firefox Prozess, danach am häufigsten Tor.exe Prozess und am wenigsten Artefakte in anderen Prozessen
		> Bemerkenswert: "mallofamerica.com" ist mit 26.505 mal in RAM Dump 2 am häufigsten als Artefakt gefunden worden. Vergleich: "mooserliesl.de" wurde nur 508 mal in RAM Dump 2 gefunden
		> Nach Schließen von Tor Browser: keine URL Artefakte mehr in RAM
		
		> TODO: DNSCache?

Yararule "Mail":
	\begin{figure}[h!]
		\centerline{\resizebox{\linewidth}{!}{\includegraphics{bilder/volatility/tor/mail.png}}}
		\label{chart:final-criteria}  
		\caption{Mail}
	\end{figure}
	Analyse:
		> Alle Mail Artefakte gefunden
		> Artefakte ausschließlich in Firefox Prozess gefunden
		> Artefakte fast ausschließlich in RAM Dump 2 Mail gefunden
		> Nur die Absenderadresse "computerforensikvl@gmail.com" wurde nach Identitäts-Reset in RAM Dump 3-1 gefunden
		> Absenderadresse ist häufigstes Mail Artefakt
		> Bemerkenswert: Passwort wurde 2x als Klartext im RAM gefunden!
			String Kontext:
				Offsets:		PIDs:
				0xb9ce29180c8	7420
				0x2859f4ffd4e0	7420
				0x24083b41858	8424
				0x240840e5b08	8424
				
Yararule "Image":
	\begin{figure}[h!]
		\centerline{\resizebox{\linewidth}{!}{\includegraphics{bilder/volatility/tor/image.png}}}
		\label{chart:final-criteria}  
		\caption{Image}
	\end{figure}
	Analyse:
		> Hex-Wert von Donaukurier Bild wurde ein einzigees mal im 2. RAM Dump in einem Firefox Prozess gefunden
	

Zusammenfassung = Stacked Bar Chart:
\begin{figure}[h!]
	\centerline{\resizebox{\linewidth}{!}{\includegraphics{bilder/volatility/tor/summary.png}}}
	\label{chart:final-criteria}  
	\caption{Summary}
\end{figure}
- PB Artefakte ausschließlich in RAM Dump 2 und 3-1 gefunden
- Nach Identitäts-Reset deutlich weniger Artefakte in vorhanden
- Am meisten URL-Artefakte gefunden, wobei mallofamerica.com dominant 
- HTML Artefakte wurden in keinem RAM Dump gefunden


TODO: Kreisdiagramme/Balkendiagramme mit Gesamtzahl an (Non-)Firefox Yarascan-Treffer erst im Vergleich mit Tor

\subsection*{Uncommon Locations}

Literatur:
%o Autopsy: \cite{Muir.2019}
%	•	Configuration files, downloaded files, and browserrelated data are recoverable from the file system.
%	•	Significant data-leakage from the browsing session occurred: HTTP header information, titles of web pages and an instance of a URL were found in registry files, system files, and unallocated space.
%o RAM-Analyse nach \cite{Muir.2019}:
%	•	Live-Analyse identifiziert auch nach dem Schließen und Deinstallieren des Browsers und Abmelden des Benutzers Spuren von Tor-Prozessen, einschließlich des absoluten Pfads zur Browser-Executable, des Benutzernamens und des Geräts, von dem es ausgeführt wurde.
%	•	The data-leakage contained the German word for ’search’ in reference to a Google search. This hints at the locale of the Tor server used to exit the network (exit relay).
%
%o RAM-Analyse nach \cite{Hariharan.2022}:
%	o	process was found to be firefox.exe
%	o	pslist and pstree: parent process was shown 
%	o	Belkasoft Ram Capturer: retrieve information about facebook
%	o	Cmdline: file path of the browser “E:/TorBrowser/Browser/firefox.exe” + name of process tor.exe and firefox.exe
%	o	Dlllist: DLL files of the executable files were not captured
%	o	Netscan: tor.exe + obfs4proxy.exe -> showed “LISTENING” connections to nonstandardized ports as output.
%	Yarascan: was able to retrieve all the browsing sessions
%o RAM-Analyse nach \cite{Sajan.2021} mit Volatility
%	•	process list extracted from the memory
%	•	registry hives been extracted from the memory dump
%	•	threads were extracted: “D:/VolatilityWorkbench/volatility.exe”–plugins=”D:/VolatilityWorkbench/profiles” pslistfilename =”C:/Users/username/Desktop/tor.raw” –profile=Win10x64 17763 –kdbg=0xf807606ac5e0
%	•	Handles: resources used by the process 5672
%	•	Dlls: These dlls can be found from prefetch file --> Can be found in “prefetch” file -> Analyzed with “winprefetchview”
%	•	Places.sqlite: SQLite viewer has been used to recover bookmarks and frequently visited sites even after uninstalling the application
%	•	Visited Websites: Using keyword search in Dump’s Hex
%
%o Registry:
%	> Shellactivites (siehe Firefox) \cite{Muir.2019}: instance of a URL were found in registry file
%	> \cite{Nelson.2020} The userassist key is located in the NTUSER.dat hive of the
%		 -> Registry and indicates the execution path of the program, as well as the number of times the program was executed 




\section{Chrome}

\subsection*{Uncommon Locations}

o Autopsy Keyword-Suche: 
	> Chrome and Edge produced five artefacts as reported by both tools. (FTK, Autopsy) \cite{Gabet.2018}
		--> Artefakte werden nicht genannt!
	> only two temporary files (Figure 7) were recovered with Minitool Power Data Recovery but it was a dead end; Location: appdata/…/Chrome/…/ Preferences/RF1533fa.TMP \cite{Fayyad.2021}
	> pagefile.sys file showed no traces at all \cite{Said.2011}
	

\section{Brave}