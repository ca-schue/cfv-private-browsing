\chapter{Ergebnisse}

= LÄNGSTES/AUSFÜHRLICHSTES KAPITEL!!!

Für jedes Unterkapitel gilt: 
> Erst allgemeines Vorgehen/Methodik definieren
> Danach spezifisch für jeden Browser: Unterschied zwischen Snapshot-Zeitpunkten, insb. zwischen Live- und Dead-Forensik

\section{Firefox}

\subsection*{White-Box Analyse/Common Locations}

\subsubsection*{Schreiboperationen mit Process Monitor verfolgen}

Qualitative Analyse:
	o no traces were found in “common locations” \cite{Montasari.2015}
		>  “places.sqlite”, “webappsstore. sqlite”, “sessionstore.bak”, “search.json” and “nssckbi.dll”
	o	Safebrowsing: Alle Dateien in /safebrowsing-updating/ nicht relevant. Dort nur .vlpset und .sbstore Dateien. Speichern 256-Bit Hash von URLs, die auf SafeSearch Blacklist stehen 
		•	Logfile 1 vs Logfile 2
	o	Cache-Dateien: drei Caches: startupCache, jumpListCache (beide enthalten Binärdateien ohne Browsing Artefakte) und cache2 (können mit MozillaCacheView untersucht werden, enthalten keine Browsing Artefakte)
		•	Logfile 1 vs Logfile 2
	o	SQLite Datenbanken: Sqlite Dateien erst ohne WAL Dateien untersuchen, Danach mit sqlite3 Konsole: WAL in Datenbank schreiben mit: PRAGMA wal\_checkpoint; places.sqlite besonders relevant, da dort Browser in public Modus Browsing URLs verwaltet (Am besten hier vergleich mit Public Browsing machen)	
		> \cite{Fayyad.2021} for Mozilla Firefox, 7 database files were recovered: cookies.sqlite-shm, places.sqlite-shm, prefs.js etc.
		> \cite{Muir.2019} The two SQLite databases used by Firefox to track cookies and history (cookies.sqlite und places.sqlite) were both recoverable from the file system after deletion	
		Ergebnisse stehen im Gegensatz zu \cite{Hedberg.2013} :
			o	Chrome und Firefox: Einträge in places.sqlite + history.sqlite DB gefunden während PB! (Noch aktuell??)
		Sonderfall: SQlite DB-Crash \cite{Hedberg.2013}
			> WAL Files/Journal Files bei Crash gefunden -> Kann genutzt werden um zu beweisen, dass privater Browser genutzt wurde
			> Daher: WAL Rollback mit sqlite3		
		•	Logfile 1 vs Logfile 2
	o	Jsonlz4 \& balkz4: Enthalten komprimierte Firefox-Sessions, jsonlz4 Dateien können mit Tool "entkomprimiert" werden: https://www.jeffersonscher.com/ffu/scrounger.html
		•	Logfile 1 vs Logfile 2
	o	JSON: ???
		•	Logfile 1 vs Logfile 2
	o	Glean: Enthalten Tracking Daten; Pageload insb. interessant
		•	Logfile 1 vs Logfile 2
	o	Sonstige Dateien: ???
		•	Logfile 1 vs Logfile 2

Quantitative Zusammenfassung
	•	Anzahl geschriebene Daten nach Dateiendung
	•	Anzahl geschriebene Dateien nach Kategorie (Safebrowsing, Cache, etc.)
	•	Anteil nicht-gelöschter Dateien am Ende von Logfile 1 (Snapshot 2)
	•	Anteil nicht-gelöschter Dateien am Ende von Logfile 2 (Snapshot 3)

\subsubsection*{SQLite-Datenbänke}
	TODO

\subsubsection*{Registry}
>	Auf Autor verweisen: angeblich in Shellactivities Ergebnisse. --> Nicht mehr vorhanden in aktueller Version (Verweis auf E-Mail)
>	Process Monitor/Regshot zeigen keine relevanten Key-Änderungen
> \cite{Muir.2019}: Autopsy Keyword Suche nach Suchbegriffen: Ergebnisse in \%SystemRoot\%Minidump NTUSER.DAT, ntuser.dat.LOG1 (a log of changes to NTUSER.DAT)
> Zentral: shellactivites Key:	NTUSER.DAT --> “shellactivities” key \cite{Muir.2019}
> \cite{Rochmadi.2017} Detection of registry changes helps to determine what the appropriate plugin is used to search for digital evidence using volatility memory forensic:
- RegQueryValue:	HKCU/Software/Microsoft/Windows/CurrentVersion/InternetSettings/Connections/DefaultConnectionSettings
- RegCloseValue: 	HKCU/Software/Microsoft/Windows/CurrentVersion/InternetSettings/Connections
- IRP\_MJ\_READ: C:/pagefile.sys


\subsection*{Black-Box Analyse/Uncommon Locations}

\subsubsection*{Analyse mit Autopsy}
Qualitative Analyse:
	o	Autopsy Keywortsuche: 
		>	In alles Snapshots ergebnislos (keine Keyword-Hits
		-->	In Literatur: Autoren fanden Ergebnisse in pagefile.sys 
			> Autopsy: websites and some of the keywords found in hidden file called “pagefile.sys” \cite{Mahlous.2020}
			o \cite{Montasari.2015} traces were found in: 
				> However, on investigating the “pagefile.sys”, some entries were discovered
				> Using the “data carving” technique, profile picture was recovered
			o \cite{Said.2011} 
				> Examining pagefile.sys showed some positive hits 			
		--> Evtl. hier zeigen, was gefunden werden kann, wenn RAM reduziert
		--> Aber auf Problem hinweisen, dass gefundener String in pagefile nicht direkt Browser zugeordnet werden kann
		> \cite{Gabet.2018}	Firefox only produced three recoverable artefacts as reported by both tools (FTK, Autopsy) --> Artefakte werden nicht genannt!
		> \cite{Muir.2019} Autopsy Keyword Suche nach Suchbegriffen: unallocated space
		> Autopsy Carving Module (\$Carved): \cite{Muir.2019}
			•	When searching for the string ’clot’ from the browsing protocol, six .dll, .edb and .reg files were discovered in unallocated space.
			•	Further searching of unallocated space uncovered references to the Tor installation directory and the obfs4 bridging IP addresses
			•	browsing data found in NTUSER.DAT was also replicated in unallocated space.
	o	Autopsy PlugIns:
		>	TODO:

\subsubsection*{Analyse mit Volatility}
o	RAM Yarascan Treffer
	>	Dump 1 vs 2 vs 3 
	>	Im 3. Dump svchost: Evtl. mit Process Explorer VM Snapshot klonen, “auftauen” und zeigen, dass DNS-Cache die Daten speichert. 
	-->	Hier DNSlist zeigen
	-->	Evtl. 4. Dump nach Beenden/Deaktivieren von DNS-Cache zeigen, dass keine Yarascan Ergebnisse mehr vorhanden

	> Diagramme:
		- 

		
\subsubsection*{Quantitative Zusammenfassung}
>	Keyword-Hits Autopsy vs. RAM (pro Zeitpunkt)
>	Anzahl private Browsing Artefakte pro Keyword (pro Zeitpunkt)


\section{Tor}

\subsection*{Uncommon Locations}

\subsubsection*{Qualitative Analyse}

o Autopsy: \cite{Muir.2019}
	•	Configuration files, downloaded files, and browserrelated data are recoverable from the file system.
	•	Significant data-leakage from the browsing session occurred: HTTP header information, titles of web pages and an instance of a URL were found in registry files, system files, and unallocated space.



o RAM-Analyse nach \cite{Muir.2019}:
	•	Live-Analyse identifiziert auch nach dem Schließen und Deinstallieren des Browsers und Abmelden des Benutzers Spuren von Tor-Prozessen, einschließlich des absoluten Pfads zur Browser-Executable, des Benutzernamens und des Geräts, von dem es ausgeführt wurde.
	•	The data-leakage contained the German word for ’search’ in reference to a Google search. This hints at the locale of the Tor server used to exit the network (exit relay).

o RAM-Analyse nach \cite{Hariharan.2022}:
	o	process was found to be firefox.exe
	o	pslist and pstree: parent process was shown 
	o	Belkasoft Ram Capturer: retrieve information about facebook
	o	Cmdline: file path of the browser “E:/TorBrowser/Browser/firefox.exe” + name of process tor.exe and firefox.exe
	o	Dlllist: DLL files of the executable files were not captured
	o	Netscan: tor.exe + obfs4proxy.exe -> showed “LISTENING” connections to nonstandardized ports as output.
	Yarascan: was able to retrieve all the browsing sessions
o RAM-Analyse nach \cite{Sajan.2021} mit Volatility
	•	process list extracted from the memory
	•	registry hives been extracted from the memory dump
	•	threads were extracted: “D:/VolatilityWorkbench/volatility.exe”–plugins=”D:/VolatilityWorkbench/profiles” pslistfilename =”C:/Users/username/Desktop/tor.raw” –profile=Win10x64 17763 –kdbg=0xf807606ac5e0
	•	Handles: resources used by the process 5672
	•	Dlls: These dlls can be found from prefetch file --> Can be found in “prefetch” file -> Analyzed with “winprefetchview”
	•	Places.sqlite: SQLite viewer has been used to recover bookmarks and frequently visited sites even after uninstalling the application
	•	Visited Websites: Using keyword search in Dump’s Hex

o Registry:
	> Shellactivites (siehe Firefox) \cite{Muir.2019}: instance of a URL were found in registry file
	> \cite{Nelson.2020} The userassist key is located in the NTUSER.dat hive of the
		 -> Registry and indicates the execution path of the program, as well as the number of times the program was executed 

\subsubsection*{Quantitative Zusammenfassung}



\section{Chrome}

\subsection*{Uncommon Locations}

o Autopsy Keyword-Suche: 
	> Chrome and Edge produced five artefacts as reported by both tools. (FTK, Autopsy) \cite{Gabet.2018}
		--> Artefakte werden nicht genannt!
	> only two temporary files (Figure 7) were recovered with Minitool Power Data Recovery but it was a dead end; Location: appdata/…/Chrome/…/ Preferences/RF1533fa.TMP \cite{Fayyad.2021}
	> pagefile.sys file showed no traces at all \cite{Said.2011}
	

\section{Brave}