\chapter*{Erklärung}
\addcontentsline{toc}{chapter}{Erklärung}

Wir erklären hiermit ehrenwörtlich, dass wir die Arbeit selbständig verfasst, noch nicht anderweitig für Prüfungszwecke vorgelegt, keine anderen als die angegebenen Quellen oder Hilfsmittel benützt sowie wörtliche und sinngemäße Zitate als solche gekennzeichnet haben.

\vspace{2cm}

Ingolstadt, den 20.06.2023 		\hfill Ingolstadt, den 20.06.2023			\\

\vspace{0.8cm}
.............................. 	\hfill .................................	\\
Christoph Sell 					\hfill Carl Schünemann						\\


\chapter*{Bearbeiter der jeweiligen Kapitel}
\addcontentsline{toc}{chapter}{Bearbeiter der jeweiligen Kapitel}

Die Aufteilung bei der Ausarbeitung der Kapitel war dabei wie folgt:

\begin{tabular}{ll}
	Kapitel 1:	& Carl Schünemann (komplett)																																	\tabularnewline
	Kapitel 2:	& Christoph Sell (komplett)																																		\tabularnewline
	Kapitel 3:	& Carl Schünemann (komplett)																																	\tabularnewline
	Kapitel 4:	& Carl Schünemann (komplett)																																	\tabularnewline
	Kapitel 5:	& Carl Schünemann Kapitel \ref{section:ergebnisse-firefox} und Kapitel \ref{section:ergebnisse-tor}  (Firefox und Tor) und 										\tabularnewline
				& Christoph Sell  Kapitel \ref{chap:ergebnisse-chrome} und Kapitel \ref{chap:ergebnisse-brave} (Chrome und Brave) 												\tabularnewline
	Kapitel 6:	& Carl Schünemann (komplett)																																	\tabularnewline
	Kapitel 7:	& Christoph Sell (komplett)																																		\tabularnewline
	Kapitel 8:	& Christoph Sell (komplett)																																		\tabularnewline
	Kapitel 9:	& Carl Schünemann Kapitel \ref{appendix:yara-regeln}, Kapitel \ref{section:appendix-firefox} und Kapitel \ref{section:appendix-tor} (Firefox und Tor) und 		\tabularnewline
				& Christoph Sell Kapitel \ref{chap:anhang-chrome} und Kapitel \ref{chap:anhang-brave} (Chrome und Brave) 														\tabularnewline
	\end{tabular}


%EOF