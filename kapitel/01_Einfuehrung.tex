\chapter{Einleitung}
Webbrowser speichern Informationen wie den Verlauf von besuchten Websites, Suchbegriffe, Passwörter, Cookies und andere Nutzeraktivitäten. 
Um die Privatsphäre von Benutzern zu schützen, wurde der sogenannte \textit{private Modus} für Browser entwickelt.
Bei den meisten Browsern ist dieser Schutz auf den lokalen Rechner beschränkt. \cite{Rochmadi.2017} Um die Privatsphäre im gesamten Internet zu schützen werden zusätzliche Maßnahmen, wie beispielsweise VPNs empfohlen. \cite{Perdices.2023}
%Trotzdem wurde im Jahr 2016 die Verwendung eines privaten Browsing-Fensters als die weltweit beliebteste Form der Online-Datenschutzmaßnahme festgestellt [1].

Es gibt unterschiedliche Nutzermodelle für private Browsing-Modi. Einersets verwenden Privatpersonen diese Technologien, um ihre Privatsphäre zu schützen und ihren lokalen digitalen Fußabdruck im Internet zu regulieren \cite{Horsman.2019}. Darüber hinaus nutzen einige Personen private Browsing-Modi, um persönliche Informationen vor Betrügern im Internet zu schützen oder spezifische Websites, wie beispielsweise Erwachsenen- oder Geschenk-Websites, diskret zu besuchen \cite{Aggarwal.2010}. Auf der anderen Seite nutzen kriminelle Nutzer private Browsing-Modi, um Online-Straftaten zu verschleiern und digitale Beweise in kriminellen Fällen zu minimieren oder zu verhindern \cite{Montasari.2015, Rochmadi.2017}. Des Weiteren gibt es staatlich unterdrückte Nutzer, wie beispielsweise Journalisten in autokratischen Staaten, die private Browsing-Modi nutzen, um ihre freie Meinungsäußerung ohne Repressionen zu ermöglichen \cite{Rathod.2017}. Jedes dieser Nutzermodelle hat seine eigenen Motivationen und Gegenspieler.

Entwickler von privaten Browsing-Modi stehen deshalb vor einem Dilemma, da sie entscheiden müssen, wer zu welchem Grad geschützt werden soll. Beispielsweise strebt der Tor-Browser an, Menschenrechte und Freiheiten zu fördern. % https://www.torproject.org/de/download/
Jedoch erschweren seine Funktionalitäten forensische Ermittlungen bei Ermittlungen zu kriminellen Nutzern \cite{Muir.2019, Rathod.2017}.

Unabhängig davon, wer private Browsing-Modi nutzt, haben alle Stakeholder Interesse daran, ob und welche Spuren hinterlassen werden. In der Literatur werden stets neue Schwachstellen identifiziert, durch die private Browsing-Daten "lecken" \cite{Satvat.2014}.
Der Umfang dieser Seminararbeit umfasst eine Untersuchung der privaten Browsing-Modi von vier Webbrowsern: Mozilla Firefox, Tor-Browser, Google Chrome und Brave \cite{Montasari.2015}. Es wird untersucht, ob und welche Spuren von diesen Browsern in ihren privaten Modi auf den lokalen Rechnern hinterlassen werden.

%Immer mehr Web-Browser Nutzer:\cite{Izzati.2022}
%	•	Im Jahr 2019 gab es laut [13] fast 4,5 Milliarden Internetnutzer.
%	•	Webbrowser werden täglich genutzt, um verschiedene Online-Aktivitäten durchzuführen. \cite{Montasari.2015}
%Defintion Private Browsing:
%	•	Webbrowser speichern normalerweise Informationen wie URL-Verlauf, Suchbegriffe, Passwörter, Cookies und andere Nutzeraktivitäten. \cite{Rochmadi.2017}
%	•	Aus Sicherheitsgründen wurden einige Funktionen von Webbrowsern entwickelt, um den privaten Modus zu ermöglichen. \cite{Rochmadi.2017}
%	•	Private Browsing-Modi wurden entwickelt, um Benutzern das Surfen im Internet zu ermöglichen, ohne Spuren zu hinterlassen. \cite{Montasari.2015}
%	- Ziel des privaten Modus = Schutz der Privatsphäre des Nutzers !!!!
%Steigende Beliebtheit private Browsing: \cite{Horsman.2019}
%	•	Die Verwendung von PB wurde als die beliebteste Form der Online-Privatsphäre weltweit identifiziert.
%	•	Im Jahr 2016 wurde die Verwendung eines PB-Fensters als die weltweit beliebteste Form der Online-Datenschutzmaßnahme identifiziert [1]. Allein in den USA nutzen Berichten zufolge rund 33 \% der Nutzer ein PB-Fenster, wobei über 70 \% zugeben, ihren Internetverlauf zu löschen [2].
%
%Unterschiedliche Nutzermodelle:
%1) "Normale Nutzung": Es gibt einige Statistiken zur Nutzung des privaten Modus in Browsern
%	> Nutzer: Privatpersonen
%	> Motivation:
%		- Sensibilität und Öffentlichkeit für den Schutz der Privatsphäre und die Regulierung des eigenen digitalen Fußabdrucks im Internet werden PB-Technologien wahrscheinlich häufiger auf den Geräten der Nutzer eingesetzt. \cite{Horsman.2019}
%		- Schutz persönlicher Informationen vor Betrügern im Internet % https://www.statista.com/statistics/1384224/uk-reasons-private-browsing/
%		- Besuch von Erwachsenen- und Geschenk-Websites (\cite{Aggarwal.2010})
%	> Gegenspieler: Kriminelle/andere Privatpersonen
%2) "kriminelle Nutzer" => Private Browsing als Anti-Forensik (von öffentlichen Umfragen nicht erfasst) 
%	> Nutzer: Kriminelle
%	> Motivation: 
%		- Verschleiern von Online-Straftaten \cite{Montasari.2015}
%		- digitale Beweise in kriminellen Fällen zu minimieren oder zu verhindern. \cite{Rochmadi.2017}
%	> Gegenspieler = Stafverfolgungsbehörden, Forensiker
%3) "Staatlich unterdrückte Nutzer"
%	> Nutzer: z.B. unterdrückte Journalisten in autokratischen Staaten
%	> Motivation:
%		- freie Meinungsäußerung ohne Repressionen 	
%		- Nutzung als demokratisches Mittel
%	> Gegenspieler = Staatliche Institutionen
%==> Entwickler von PB Modi stehen vor einem Dilemma: Wer wird geschützt?	
%- Z.B. 
%	> Einerseits möchte Tor-Browser Menschenrechte und Freiheiten Schützen: 
%	"Menschenrechte und Freiheiten durch die Entwicklung und Verbreitung von Open Source Anonymitäts- und Privatsphäre-Technologien zu fördern, ihre ungehinderte Verfügbarkeit zu unterstützen und ihr Verständnis in Wissenschaft und der Allgemeinheit zu vergrößern."
%	% https://www.torproject.org/de/download/
%	> Erschwert jedoch forensische Ermittlungen bei kriminellen Nutzern \cite{Muir.2019} \cite{Rathod.2017}
%
%Unabhängig davon wer PB Nutzt: Alle Stakeholder haben Interesse daran, ob und welche PB hinterlassen werden.
%	•	In Literatur immer wieder Schwachstellen identifiziert, durch die private Browsingdaten “lecken” von denen viele zuvor nicht bekannt waren. \cite{Satvat.2014}
%	•	Schwachstellen stets schnell von Browser-Entwicklern geschlossen \cite{Satvat.2014}
%
%Umfang dieser Arbeit:
%	•	Versuch für vier Webbrowser Mozilla Firefox, Tor-Browser, Google Chrome und Brave durchgeführt, um zu untersuchen, ob ihre privaten Modi Spuren auf dem lokalen Rechner hinterlassen. \cite{Montasari.2015}



