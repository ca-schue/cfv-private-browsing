\chapter{Einleitung}
Webbrowser speichern Informationen wie den Verlauf von besuchten Websites, Suchbegriffe, Passwörter, Cookies und andere Nutzeraktivitäten. 
Um die Privatsphäre von Benutzern zu schützen, wurde der sogenannte \textit{private Modus} für Browser entwickelt.
Bei den meisten Browsern ist dieser Schutz auf den lokalen Rechner beschränkt. \cite{Rochmadi.2017} Um die Privatsphäre im gesamten Internet zu schützen, werden zusätzliche Maßnahmen, wie beispielsweise VPNs empfohlen. \cite{Perdices.2023}

Es gibt unterschiedliche Nutzermodelle für private Browsing-Modi. Einersets verwenden Privatpersonen diese Technologien, um ihre Privatsphäre zu schützen und ihren lokalen digitalen Fußabdruck im Internet zu regulieren \cite{Horsman.2019}. Darüber hinaus nutzen einige Personen private Browsing-Modi, um persönliche Informationen vor Betrügern im Internet zu schützen oder spezifische Websites, wie beispielsweise Erwachsenen- oder Geschenk-Websites, diskret zu besuchen \cite{Aggarwal.2010}. Auf der anderen Seite nutzen kriminelle Nutzer private Browsing-Modi, um Online-Straftaten zu verschleiern und digitale Beweise in kriminellen Fällen zu minimieren oder zu verhindern \cite{Montasari.2015, Rochmadi.2017}. Des Weiteren gibt es staatlich unterdrückte Nutzer, wie beispielsweise Journalisten in autokratischen Staaten, die private Browsing-Modi nutzen, um einer freien Pressearbeit ohne Repressionen nachzugehen \cite{Rathod.2017}. Jedes dieser Nutzermodelle hat seine eigenen Motivationen und Gegenspieler.

Entwickler von privaten Browsing-Modi stehen deshalb vor einem Dilemma, da sie entscheiden müssen, wer zu welchem Grad geschützt werden soll. Beispielsweise strebt der Tor-Browser an, Menschenrechte und die Freiheiten des Individuums zu fördern. \cite{Tor.24.05.2023}
Jedoch erschweren seine Funktionalitäten forensische Ermittlungen zu kriminellen Nutzern \cite{Muir.2019, Rathod.2017}.

Unabhängig davon, wer private Browsing-Modi nutzt, haben alle Stakeholder Interesse daran zu erfahren, ob und welche Spuren hinterlassen werden. In der Literatur werden stets neue Schwachstellen identifiziert, durch die private Browsing-Daten \glqq{}lecken\grqq{} \cite{Satvat.2014}.
Im Rahmen dieser Seminararbeit werden die privaten Browser die privaten Browsing-Modi von vier Webbrowsern untersucht: Mozilla Firefox, Tor-Browser, Google Chrome und Brave \cite{Montasari.2015}. Es wird analysiert, ob und welche Spuren von diesen Browsern in ihren privaten Modi auf den lokalen Rechnern hinterlassen werden.
